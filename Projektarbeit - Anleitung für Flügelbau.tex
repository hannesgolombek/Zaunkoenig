Der Bau des Flügels erfolgt in drei Abschnitten. Zuerst soll der Steg gebaut werden, danach die Flügelschale und abschließend erfolgt die Verklebung der Bauteile.
Es werden beide Negativformen der Profilform zur Verfügung gestellt, verbaut werden die Gewebe Interglas 90070 und 92145, das Harz ... inkl. Härter, Mikroballons zu Andicken der Mumpe, Messing und Kiefernholz. Zur Positionierung von Bauteilen und eigenem Formenbau sollen Aluminiumprofile genutzt werden.

\begin{enumerate}
	\item \textbf{Holm:}
	\begin{enumerate}
		\item Jeweils zwei Aluminiumprofile werden als Formwände für die Holmgurte in beiden Profilformen positioniert. Dadurch wird die Breite garantiert, während die Profilform die Wölbung schafft.
		\item Anschließend werden 9 Lagen Interglas 92145 in diesen Formen laminiert. 
		\item Nach dem Aushärten werden die Gurte und die Aluminiumprofile aus der Profilform entnommen. 
		\item Die Gurte werden an den Enden auf die passende Länge geschliffen.
		\item Der Schaumstoff wird zugeschnitten und die Abstufung geschliffen.
		\item Es werden Lagen Interglas 90700 auf einer ebenen Fläche laminiert. Dabei muss auf die Abstufung von 12 auf 2 und auf die 45°-Ausrichtung geachtet werden. Danach wird der Schaum positioniert und anschließend wieder 12 bis 2 Lagen in umgekehrter Reihenfolge der Abstufung laminiert.
		\item Nach dem Härten des Steg-Sandwichs wird dieser auf das Endmaß geschliffen.
		\item Mittels kleiner Holz-Klebewinkel wird der Steg auf einem Holmgurt positioniert und mit Harz verklebt.
		\item Nun werden Die Holzklötze für die Verstiftungen , die Wurzlrippe und die Endrippe eingeklebt. An den verbleibenden Kanten der Verklebung des Steges mit dem Gurt wird mit Mumpe die nötige Klebefläche geschaffen.
		\item Sobald die Mumpe gehärtet ist, wird der andere Gurt auf den Steg geklebt und die Klebekante wird ebenfalls mit Mumpe ausgefüllt.
		\item Abschließend werden in den Steg an den entsprechenden Stellen Löcher gebohrt und die Messinghülsen eingesetzt.
	\end{enumerate}
	\item \textbf{Flügelschale:}
	\begin{enumerate}
		\item Anhand der Profilformen wird der Schaumstoff zugeschnitten und die Schrägen werden geschliffen.
		\item In beide Profilschalen wird die äußere Lage Interglas 90070 laminiert, anschließend wird der Schaumstoff positioniert und die innere Lage folgt.
	\end{enumerate}
	\item \textbf{Verklebung:}
	\begin{enumerate}
		\item Nun wird der Holm inkl. der Rippen in die untere Schale mit Mumpe geklebt.
		\item Darauf wird die obere Flügelschale verklebt, dabei wird diese sowohl mit dem Hol, als auch mit der unteren Schale verklebt.
		\item Die Klebekanten beider Flügelschalen werden in Form geschliffen.
		\item Der Flügel ist nun fertiggestellt, nun könnten dieser gespachtelt und lackiert werden. Darauf wird jedoch aus Gewichtsgründen verzichtet. Um dabei nicht die äußere Gewebelage zu beschädigen, wäre es dafür angebracht, eine Schutz-Gewebelage zusätzlich in dne Lagenaufbau einzuplanen.
	\end{enumerate}
\end{enumerate}

\noindent Der Flügel wird im Handlaminat-Verfahren hergestellt. Für die Gewichtseinsparung wären Harzinfusionen oder Vakuumverfahren möglich, dieses würde aber nicht den zusätzlichen Aufwand rechtfertigen.