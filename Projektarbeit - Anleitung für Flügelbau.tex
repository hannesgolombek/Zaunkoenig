Nach der erfolgten Auslegung, Dimensionierung und Konstruktion des Tragflügels erfolgt, die Entwicklungsphase abschließend, der Bau eines Prototyps. Versuchsbauten werden, neben der Erprobung der Fertigung und Funktion, hauptsächlich für Strukturtests zum Nachweis der erfolgten Rechnungen genutzt. Um den Modellflügel des Zaunkönigs fertigen zu können, wird im Folgenden eine grob-strukturierte Anleitung gestellt. Dabei werden jedoch nicht alle notwendigen Nebenschritte genannt, sondern von erfahrenen Handwerker als bekannt vorausgesetzt. Ebenfalls wird nicht auf eine vorherige Fertigung kleinerer Teile eingegangen.\\

\noindent Der Bau des Flügels erfolgt in drei Abschnitten. Zuerst soll der Holm gebaut werden, danach die Flügelschale und abschließend erfolgt die Verklebung der Bauteile.
Es werden beide Negativformen der Profilform zur Verfügung gestellt, verbaut werden die Gewebe Interglas 90070 (bidirektionial) und 92145 (annähernd unidirektional), das Epoxidharz L385 inkl. Härter H386, Aerosil oder Baumwollflocken zum Andicken des Harze (ugs. Mumpe), Messing und Buchenholz. Zur Positionierung von Bauteilen und zum eigenen Formenbau sollen Strangprofile genutzt werden. 

\begin{enumerate}
	\item \textbf{Holm:}
	\begin{enumerate}
		\item Jeweils zwei Aluminiumprofile werden als Formwände für die Holmgurte in beiden Profilformen positioniert. Dadurch wird die Breite garantiert, während die Profilform die Wölbung schafft. Die Flügelform wird durch gefräste Profile verlängert, um den Holmstummel ebenfalls in gewölbter Geometrie bauen zu können. 
		\item Anschließend werden je 9 Lagen Interglas 92145 in diesen Formen laminiert. (Belegung nach Kapitel  4.4) 
		\item Nach dem Aushärten werden sowohl die Gurte als auch die Aluminiumprofile wieder aus der Profilform entnommen. 
		\item Die Gurte werden an den Enden auf die passende Länge geschliffen.
		\item Der Schaumstoff wird zugeschnitten und die Abstufung geschliffen.
		\item Der Schaumstoff wird nun mit angedicktem Harz (Harz vermischt mit Microballons) \glqq abgespachtelt\grqq, um ein Vollsaugen der Poren mit nicht-gehärtetem Harz zu vermeiden. Der folgende Schritt muss so zeitnah erfolgen, dass die Härtung noch nicht erfolgt ist. Dadurch wird eine Nass-Nass-Verklebung des Schaumstoffs mit Geweben garantiert.
		\item Es werden Lagen Interglas 90700 auf einer ebenen Fläche laminiert. Dabei muss auf die Abstufung von 12 auf 2 Gewebelagen und auf die 45°-Ausrichtung geachtet werden. Danach wird der Schaum positioniert und anschließend wieder 12 bis 2 Lagen in umgekehrter Reihenfolge der Abstufung belegt (Belegung nach Kapitel 4.4).
		\item Nach dem Härten des Steg-Sandwichs wird dieser auf das Endmaß geschliffen.
		\item Mittels kleiner Holz-Klebewinkel wird der Steg auf einem Holmgurt positioniert und mit Harz verklebt.
		\item Nun werden die Holzklötze für die Bolzenaufnahmen , die Wurzlrippe und die Endrippe eingeklebt. An den verbleibenden Kanten der Verklebung des Steges mit dem Gurt wird mit Mumpe die nötige Klebefläche geschaffen.
		\item Sobald die Mumpe gehärtet ist, wird der andere Gurt auf den Steg, die Rippen und die Holzklötze geklebt und die Klebekanten werden ebenfalls mit Mumpe ausgefüllt.
		\item Abschließend werden in den Steg an den entsprechenden Stellen Löcher gebohrt und die Messinghülsen eingesetzt. Dabei kann die exakte Ausrichtung der Buchsen durch eine gleichzeitige Verbindung des Teststands mit Bolzen/Stiften erfolgen.
	\end{enumerate}
	\item \textbf{Flügelschale:}
	\begin{enumerate}
		\item Anhand der Profilformen wird der Schaumstoff zugeschnitten und die Schrägen werden geschliffen.
		\item Der Schaumstoff wird auch hierbei mit angedicktem Harz \glqq abgespachtelt\grqq, um ein Vollsaugen der Poren mit nicht-gehärtetem Harz zu vermeiden. Der folgende Schritt muss wiederum so zeitnah erfolgen, dass die Härtung noch nicht erfolgt ist. Dadurch wird eine Nass-Nass-Verklebung des Schaumstoffs mit Geweben garantiert.
		\item In beide Profilschalen wird die äußere Lage Interglas 90070 laminiert, anschließend wird der Schaumstoff positioniert und die innere Lage folgt (Belegung nach Kapitel 4.4).
	\end{enumerate}
	\item \textbf{Verklebung:}
	\begin{enumerate}
		\item Nun wird der Holm inkl. der Rippen in die untere Schale mit Mumpe geklebt. Entsprechende Hohlräume sollten mit Mumpe aufgefüllt werden, um eine durchgehend stoffschlüssige Verbindung zu schaffen.
		\item Darauf wird die obere Flügelschale verklebt, dabei wird diese sowohl mit dem Holm, als auch mit der unteren Schale verklebt wird. Da es sich hierbei um eine Blind-Verklebung handelt, ist es zu Empfehlen, etwas überschüssig Mumpe aufzubringen.
		\item Die äußeren Klebekanten beider Flügelschalen werden in Form geschliffen.
		\item Der Flügel ist nun fertiggestellt, abschließend könnte dieser gespachtelt und lackiert werden. Darauf wird jedoch aus Gewichtsgründen verzichtet. Um dabei nicht die äußere Gewebelage zu beschädigen, wäre es dafür angebracht, eine Schutz-Gewebelage zusätzlich in den Lagenaufbau einzuplanen.
	\end{enumerate}
\end{enumerate}

\noindent Als grundlegendes Fertigungsverfahren steht das Handlaminier-Verfahren zur Verfügung. Neben diesem steht außerdem das Vakuumpressen und die Vakuuminjektion zur Auswahl. Ersteres Herstellungsverfahren beruht darauf, dass Gewebe- bzw. Gelegelagen  manuell mit Pinseln und Schaum-Rollen mit Harz getränkt, per Hand ausgerichtet und anschließend mit diesen Werkzeugen entlüftet werden. Dieses bringt einen niedrigen Faservolumengehalt, jedoch auch eine individuellere Gestaltung eines Bauteils in offenen Formen mit sich. \\
Beim Vakuumpressen handelt sich um eine Weiterentwicklung dieses Verfahrens. Nach dem eigentlichen Laminieren wird das Bauteil zusätzlich mit einer Folie abgedichtet, sodass die eingeschlossene Luft entzogen werden kann und die Folie somit einen zusätzlichen Druck auf das Bauteil ausübt. Dadurch kann überschüssiges Harz von einem aufgebrachtem Fließ aufgenommen werden, sodass der Faservolumenanteil steigt. Es muss jedoch in Erfahrung gebracht werden, ob bei Sandwich-Laminaten der Stützstoff dem Flächendruck standhalten kann.\\
Der Faservolumenanteil kann weiter gesteigert werden, in dem das Harz nicht während der Belegung eingebracht, sondern nach dem Vakuumieren mittels Unterdruck in das Gewebe gesaugt wird. Hierbei ist jedoch Erfahrung nötig, ob das Harz die geforderte Zähigkeit aufweist, um alle Fasern tränken zu können. Der hohe Faservolumengehalt (ca. 50\%) ist jedoch auch mit einem höherem Aufwand verbunden.\\

\noindent Es erscheint sinnvoll, den Tragflügel im Handlaminat herzustellen. Für die Gewichtseinsparung und Steifigkeitserhöhung (Mischugsregel) durch einen höheren Faservolumengehalt wären die beiden anderen Verfahren vorteilhafter, dieses würde aber nicht den zusätzlichen Aufwand und die nicht zwingend erforderlichen Gewichtsersparnisse rechtfertigen. Außerdem können so kurzfristig kleinere Anpassung der Formgestaltung während der Fertigung einfach durchgeführt werden, was sich für einen Prototypen als unabdingbar gestaltet. Außerdem werden durch die Wahl der Fertigung ein geringerer Erfahrungsschatz und weniger Halb- und Werkzeuge für die Fertigung benötigt.