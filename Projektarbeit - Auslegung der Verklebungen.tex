

\noindent Als letzte analytische Auslegung des Holms sollen zwei bedeutende Klebeflächen des Stegs zu einer Rippe bzw. zu den Gurten berechnet nach Kapitel \ref{Verklebung} werden.
\subsubsection{Klebeverbindung Steg - Gurt}
Die Klebverbindung wird ähnlich der VDI 2013 ausgelegt, sodass nur die Abtriebskraft des Holms und die übertragene Querkraft den Schubfluss für die Belastung definieren \cite{item23}\cite{item5}:
\begin{equation}
	p=\sqrt{p_{A}^{2}+p_{s}^{2}}
\end{equation}
Die Länge ergibt sich aus 
\begin{equation}
	l=\frac{p}{\tau_{zul}}
\end{equation}
wobei die zulässige Klebeschubspannung
\begin{equation}
	\tau_{zul}=10 Pa
\end{equation}
nach Angaben der Aufgabenstellung der Schubspannung für Mumpe entsprechen soll. Die Verklebung muss wegen kleiner Bauteilmaße \glqq Trocken-Trocken\grqq\: erfolgen.\\

\noindent Bereich $I$ und $II$ werden, ähnlich der Beulberechnung, zusammen mit den kritischsten aller Werte ausgelegt , sodass 
\begin{equation}\label{Klebebreite}
	l=\frac{\sqrt{(4282,26\frac{N}{m})^{2}+(159330,16\frac{N}{m})^{2}}}{10^{7}\frac{N}{m^{2}}}=15,9mm
\end{equation}
als Klebebreite benötigt werden. Für den Bereich $III$ ergibt sich
\begin{equation}
	l=\frac{\sqrt{(4282,26\frac{N}{m})^{2}+(15665,14\frac{N}{m})^{2}}}{10^{7}\frac{N}{m^{2}}}=1,62mm
\end{equation}
Beide Klebebreiten passen auf die verbleibenden inneren Holmgurtflächen und sollen durch Mumpe ohne zusätzliche Gewebelagen realisiert werden. Die Querschnittsfläche der Mumpe sollte somit einem gleichseitigem Dreieck nachempfunden werden.

\subsubsection{Klebeverbindung Holm - Rippen}
Die vergrößerten Klebeflächen der Rippen gegenüber dem Holm sollen durch Holzklötze ermöglicht werden, die auf beide Seiten des Stegs zwischen die Holmgurte geklebt werden. Die Breite dieser Klötze errechnet sich aus 
\begin{equation}
	A=b\cdot h=\frac{F}{\tau_{zul}}
\end{equation}
mit der maximal abgesetzten Kraft von $F=500N$.
Somit haben die Holzklötze eine Breite von 
\begin{equation}
	b=1,56mm
\end{equation}
bei einer Höhe von $35,8mm-2\cdot 1,941mm$.