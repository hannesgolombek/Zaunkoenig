Der Holm, bzw. der Tragflügel generell, wird aus verschiedenen Bauteilen gefertigt, die miteinander verbunden werden müssen. Unterschieden werden kann beim Fügen zwischen kraftschlüssiger Verbindung (z.B. Schrauben), formschlüssiger Verbindung (z.B. Schnapphaken) und stoffschlüssiger Verbindung (z.B. Löten oder Kleben). FVK werden meistens miteinander verklebt, dabei muss kann man auf drei Varianten zurückgreifen:
\begin{itemize}
	\item Nass-Nass-Verklebung: Zwei Laminate werden im nicht ausgehärteten Zustand, also während ihrer Fertigung, gefügt.
	\item Trocken-Nass-Verklebung: Obwohl ein Laminat schon gehärtet ist, kann es dennoch mit einem zweiten, nicht gehärteten Laminat gefügt werden. Bei dem gehärteten Material muss eine geforderte Rauheit garantiert werden.
	\item Trocken-Trocken-Verklebung: Um hierbei eine Verbindung zweier gehärteten Laminate zu schaffen, wird ein Fügestoff genutzt. Dieser wird durch das verwendete Harzsystem mit einer Beimischung von Aerosiloder Baumwollflocken zur Andickung gebildet. Beide Bauteile müssen eine Mindestrauheit in der Klebefläche aufweisen können.
\end{itemize}

\noindent In der fogenden Dimensionierungsmethode für Klebeflächen wird lediglich auf Schubspannungen in dieser Flächen-Ebene eingegangen. Schälen oder senkrechte Spannungen dazu sind somit ausgeschlossen. Außerdem soll nur mit einfachen Klebeverbindungen von aufeinander liegendne Bauteilen gerechnet werden, da ein Anschrägen bzw. Schäften an Bauteilen solch kleiner Abmessungen teilweise ausgeschlossen werden muss.\\

\noindent Als letzte analytische Auslegung des Holms sollen zwei bedeutende Klebeflächen berechnet werden.
\subsubsection{Klebeverbindung Steg - Gurt (T.B.)}
Die Klebverbindung wird ähnlich der VDI 2013 ausgelegt, sodass nur die Abtriebskraft des Holms und die übertragene Querkraft den Schubfluss für die Belastung definieren:
\begin{equation}
	p=\sqrt{p_{A}^{2}+p_{s}^{2}}
\end{equation}
Die Länge ergibt sich aus 
\begin{equation}
	l=\frac{p}{\tau_{zul}}
\end{equation}
wobei die zulässige Klebeschubspannung
\begin{equation}
	\tau_{zul}=10 Pa
\end{equation}
nach Angaben der Aufgabenstellung der Schubspannung für Mumpe entsprechen soll. Die Verklebung muss wegen kleiner Bauteilmaße \glqq Trocken-Trocken\grqq erfolgen.\\

\noindent Bereich $I$ und $II$ werden, ähnlich der Beulberechnung, zusammen ausgelegt mit den kritischsten Werten, sodass 
\begin{equation}
	l=\frac{\sqrt{(4282,26\frac{N}{m})^{2}+(159330,16\frac{N}{m})^{2}}}{10^{7}\frac{N}{m^{2}}}=15,9mm
\end{equation}
als Klebebreite benötigt werden. Für den Bereich $III$ ergibt sich
\begin{equation}
	l=\frac{\sqrt{(4282,26\frac{N}{m})^{2}+(15665,14\frac{N}{m})^{2}}}{10^{7}\frac{N}{m^{2}}}=1,62mm
\end{equation}
Beide Klebebreiten passen auf die verbleibende innere Holmgurtflächen und sollen durch Mumpe ohne zusätzliche Gewebelagen realisiert werden. Die Querschnittsfläche der Mumpe sollte somit einem gleichseitigem Dreieck entsprechen.

\subsubsection{Klebeverbindung Holm - Rippen (T.B.)}
Die vergrößerten Klebeflächen der Rippen gegenüber dem Holm sollen durch Holzklötze ermöglicht werden, die auf beide Seiten des Stegs zwischen die Holmgurte geklebt werden. Die Breite dieser Klötze errechnet sich aus 
\begin{equation}
	A=\frac{F}{\tau_{zul}}
\end{equation}
mit der maximal abgesetzten Kraft von $F=500N$.
Somit haben die Holzklötze eine Breite von 
\begin{equation}
	b=1,56mm
\end{equation}
bei einer Höhe von $35,8mm-2\cdot 1,941mm$.