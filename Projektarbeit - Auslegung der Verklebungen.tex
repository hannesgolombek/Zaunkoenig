Als letzte analytische Auslegung sollen die Klebeflächen berechnet werden.
\subsection{Klebeverbindung Steg - Gurt (T.B.)}
Die Klebverbindung wird ähnlich nach [VDI 2013]  ausgelegt, sodass nur die Abtriebskraft des Holms und die übertragene Querkraft den Schubfluss für die Belastung definieren:
\begin{equation}
	p=\sqrt{p_{A}^{2}+p_{s}^{2}}
\end{equation}
Die Länge ergibt sich aus 
\begin{equation}
	l=\frac{p}{\tau_{zul}}
\end{equation}
, wobei 
\begin{equation}
	\tau_{zul}=7MPa
\end{equation}
nach [Kennwerte Idaflieg, ...] beträgt.\\

\noindent Bereich $I$ und $II$ werden, ähnlich der Beulberechnung, zusammen ausgelegt mit den kritischsten Werten, sodass 
\begin{equation}
	l=\frac{\sqrt{(4282,26\frac{N}{m})^{2}+(159330,16\frac{N}{m})^{2}}}{7\frac{N}{m}}=22,8mm
\end{equation}
als Klebebreite benötigt werden. Für den Bereich $III$ ergibt sich
\begin{equation}
	l=\frac{\sqrt{(4282,26\frac{N}{m})^{2}+(15665,14\frac{N}{m})^{2}}}{7\frac{N}{m}}=2,32mm
\end{equation}
. Beide Klebebreiten passen auf die verbleibende innere Holmgurtflächen und sollen durch Mumpe ohne zusätzliche Gewebelagen realisiert werden.

\subsection{Klebeverbindung Holm - Rippen (T.B.)}
Die vergrößerte Klebeflächen der Rippen gegenüber dem Holm soll durch Holzklötze ermöglicht werden, die neben dem Steg auf beiden Seiten dessen innerhalb der Holmgurte geklebt werden sollen. Die Breite dieser Klötze errechnet sich aus 
\begin{equation}
	A=\frac{F}{\tau_{zul}}
\end{equation}
mit der maximal abgesetzten Kraft von $F=500N$.
Somit haben die Holzklötze eine Breite von 
\begin{equation}
	b=1,12mm
\end{equation}
bei einer Höhe von $35,8mm-2\cdot 1,41mm$.