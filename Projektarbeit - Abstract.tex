\large{\textbf{Übersicht (O.S.)}}~\\
In dieser Projektarbeit wird ein Halbflügelmodell des Flugzeugs "Zaunkönig" im Maßstab 1:4,7 in der Glasfaserverbund-Holm-Bauweise konstruiert. Der Flügel wird dabei soweit es geht mit analytischen und ergänzend mit numerischen Methoden auf Festigkeit und Biegesteifigkeit ausgelegt. Hierfür werden die Programme ABAQUS und $ eLamX^{2} $ zur Hilfe genommen. Für einen simulierten Testdurchlauf, bei dem eine exzentrische Last aufgebracht wird, wird die Masse, Bruchlast, Schubmittelpunkt und sowohl Absenkung als auch Verdrehung an der Flügelvorderkante. Unter besonderer Rücksichtnahme des gewichtsnormalisierten Festigkeitskriteriums wird der Flügel in Bezug auf Vergleichsdaten bewertet und die errechneten Werte kritisch analysiert.
\newpage

\large{\textbf{Abstract (O.S.)}}~\\
In this project work a model halve wing of the aircraft “Zaunkönig” will be constructed scaled down 1:4.7 as a glass fiber spar structure. The wing will be construed for strength and bending stiffness using analytical methods as much as possible, completed by numerical methods. The Programs ABAQUS and $ eLamX^{2} $ will be used in assistance. For a simulated test with an eccentric force the mass, breaking load, shear center and as well lowering as torsion of the tip of the wing will be determined. Under special consideration of the weight normalized strength criterion the wing will be rated in relation to comparison data and the calculated values critically analyzed. 