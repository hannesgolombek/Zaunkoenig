Übersicht (O.S.)
In dieser Projektarbeit wird ein Halbflügelmodell des Flugzeugs "Zaunkönig" im Maßstab 1:4,7 in der Glasfaserverbund-Holm-Bauweise konstruiert. Der Flügel wird dabei so weit es geht mit analytischen und ergänzend mit numerischen Methoden auf Festigkeit und Biegesteifigkeit ausgelegt. Hierfür werden die Programme ABAQUS und $ eLamX^{2} $ zur Hilfe genommen. Für einen simulierten Testdurchlauf, bei dem eine exzentrische Last aufgebracht wird, wird die Masse, Bruchlast, Schubmittelpunkt und sowohl Absenkung als auch Verdrehung an der Flügelvorderkante. Unter besonderer Rücksichtnahme des gewichtsnormalisierten Festigkeitskriteriums wird der Flügel im Bezug auf Vergleichsdaten bewertet und die errechneten Werte kritisch analysiert.
\newpage

Abstract (O.S.)
