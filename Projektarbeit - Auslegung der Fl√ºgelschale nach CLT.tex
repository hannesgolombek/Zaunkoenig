\subsubsection{$eLamX^{2}$}

Auch hier wird die Rechnung nach Handbuch-Methoden mittels $eLamX^{2}$ überprüft (siehe Kapitel \ref{elamx}) \cite{item22}\cite{item3}. \\

\noindent Der Lagenaufbau ähnelt dem des Steges, lediglich die Anzahl der Lagen wird auf zwei reduziert und die Dicke des Schaumstoffs erhöht (siehe Abbidung \ref{fig:Lagenaufbau Haut}). Auch hierbei ergibt sich eine Sicherheit der Festigkeit $>1$, wie der Abbildung \ref{fig:Berechnung Haut} im  Kapitel \ref{Abbildungen} zu entnehmen ist. Ebenfalls werden die Ingenieurskonstanten der Flügelhaut mit dieser Software ermittelt, wie in Abbildung \ref{fig:Ingenieurskonstanten Haut} dargestellt wird.\\

\noindent Ähnlich dem Kapitel \ref{elamx} werden die Abbildungen im Anhang dargestellt, obwohl auch hier vorerst die wichtigsten Erkentnisse vorab vorgestellt werden:

\begin{longtable}{lllll}
	
	Laminat&Lagenaufbau&$\hat{E}_{x}$&$\hat{E}_{y}$&$\hat{G}_{xy}$\\
	\hline\hline
	Flügelschale&$1\cdot[45^{\circ}](sym.)$&6639,8 MPa&6639,8 MPa&8577,8 MPa\\
	\hline
	&$\nu_{xy}$&$\nu_{yx}$&Sicherheit&Versagensart\\
	&&&nach Puck&\\
	\hhline{~====}
	&0,673&0,673&3,995&Zwischenfaserbruch\\
\end{longtable}