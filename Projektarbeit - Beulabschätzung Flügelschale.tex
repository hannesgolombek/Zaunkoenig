Nachdem die Flügelhaut auf Festigkeit ausgelegt worden ist, muss überprüft werden, ob der Effekt des Beulens auftritt.\\

\noindent Dazu werden folgende Annahmen getroffen:

\begin{enumerate}
	\item Die Berechnung erfolgt nach [1]
	\item Es wird angenommen, dass die orthotropen Gewebe hinreichend mit den Gleichungen für isotropes Material berechnet werden können. Diese Annahme wird mit erfolgten Zulassungen für Segelflugzeuge anhand dieser Formeln begründet.
	\item Die Flügelschale wird als ebener, unendlicher Streifen betrachtet. Die tatsächliche Krümmung und die Knicke erhöhen die Stabilität deutlich, sodass durch diese konservative Rechnung die Sicherheit gewährleistet bleibt.
	\item Es tritt lediglich die ermittelte Schubspannung auf. In der Realität auftretende Zug- und Druckspannungen, hervorgerufen durch aufgeprägte Dehnungen des Holms und durch die Prüfkraft, werden in der Rechnung vernachlässigt.
	\item Die Flügelschale wird als an beiden Rippen fest gelagert betrachtet. Eine Verformung der Rippen wird außer Acht gelassen.
	\item Die Flügelnase und Klappenkante werden als stützende Lager angenommen, da diese einen engen Krümmungsradius aufweisen, welcher stabile und steife Strukturen hervorbringt.
	\item Der Holmsteg wird ebenfalls an seinen beiden Enden, verbunden mit der Flügelschale über die Holmgurte, als freies Lager angenommen.
	\item Dadurch ergibt sich die maximale Breite eines Streifens zu $124,5 mm$ - von der Klappenkante zum oberen Steg-Lager.
\end{enumerate}

\noindent Die Berechnung erfolgt nach den gleichen Formeln der Berechnung des Kapitels \ref{Beulsicherheit Steg}.\\

\noindent Das Seitenverhältnis beträgt für die maximale Streifenbreite
\begin{equation}
	\frac{b}{a}=\frac{124,5 mm}{773 mm}=0,16
\end{equation}
und somit wird der Beulfaktor für Schub zu 
\begin{equation}
	k=0,5
\end{equation} ermittelt. Auch hier beträgt die Steifigkeitserhöhung $\kappa=3$ und der Schubmodul $G_{xy}=8577,8 MPa$. Dadurch lässt sich die zulässige Schubspannung ermitteln:
\begin{equation}
	\tau_{krit}=3\cdot 5\cdot 8577,8 MPa\cdot\biggl(\frac{4\cdot 0,078mm + 3mm}{124,5 mm}\biggr)^{2} =91,06 MPa
\end{equation}
Die Schaumdicke wird durch konstruktive Vorgaben auf $3 mm$ gesetzt. Dia maximal auftretende Schubspannung beträgt
\begin{equation}
	\tau_{max}=42,346 MPa
\end{equation}
(im Bereich der unteren Steg-Lagerung), sodass die Sicherheit gegen Beulen 
\begin{equation}
	j=2,15
\end{equation}
beträgt. Dieses stelltjedoch eine theoretische Sicherheit dar, da die maximale Länge und die größte Schubspannung nicht im gleichen Bau-Abschnitt liegen. Die Sicherheit ist in der Realität somit höher, zusätzlich wird sie durch den gekrümmten Verlauf erhöht.