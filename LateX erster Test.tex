\documentclass[a4paper,12p]{article}
\begin {document}
\tableofcontents
\section{Einleitung}

\subsection{GIT and it´s friends}
Dieser Abschnitt dient der alleinigen Kontrolle der Funktionsfähigkeit unseres lieben Git Hub Repos. Sollten gleich Probleme beim Pushen auftreten, mag das daran liegen, dass jedesmal eine pdf-Datei mit hochgeladen werden soll. Mal schauen...

\subsection{Projektbeschreibung}
Der Zaunkönig ist ein in den frühen 1940er Jahren entstandenes Flugzeug, das unter der Leitung von Hermann Winter an der TU-Braunschweig konstruiert wurde. (Quelle: Wikipedia (bessere Quelle suchen)) Da der Zaunkönig damals vornehmlich aus Holz gebaut wurde soll jetzt ein neuer Flügel im Maßstab 1:4,7 aus Glasfaser Kunststoffverbund (GFK) konstruiert werden. Der Flügel muss gewisse Anforderungen erfüllen, die im nachfolgenden definiert werden.\\
Bei dem Flügel handelt es sich um einen Rechteckflügel und ist an den Punkten l=1/4 l und l=1/2 l über Verstrebungen mit dem Rumpf verbunden. Diese Streben sollen in der neuen Konstruktion nicht vorhanden sein. Der Flügel soll im Rumpf verstiftet werden, wobei die Torsionsbelastung durch Querkraftbolzen aufgenommen wird. Insgesamt darf der Flügel das Gewicht von 0,750 kg nicht überschreiten.\\
Um die strukturmechanischen Anforderungen zu erfüllen wird der Flügel auf seine Steifigkeit und Festigkeit geprüft. Die Steifigkeit ist hinreichen, wenn der Flügel bei einer senkrechten Belastung von 100 N am L/4- Punkt, eine Durchbiegung von z=22mm nicht überschreitet. Außerdem darf der Flügel bei einer Prüfkraft von 500 N nicht brechen. Die Haut muss so ausgelegt sein, dass kein Beulen auftritt. Zusätzlich müssen der Torsionswinkel und Schubmittelpunkt berechnet werden.

\subsection{Motivation}
Zunächst ist zu klären, warum es überhaupt sinnvoll ist für diesen Flügel GFK zu verwenden. In der Luftfahrt wird immer nach Wegen gesucht das Gewicht zu minimieren, um die Wirtschaftlichkeit von Flugobjekten zu maximieren. Faser-Kunststoffverbunde (FKV) mit ihrer hohen spezifischen Festigkeit stellen hierbei einen idealen Kandidaten dar. Zusätzlich bieten FKV einfache Formgebung für komplexe aerodynamische Profile und auch die Korrosionsbeständigkeit ist höher, als bei konventionellen Werkstoffen. Als ein großer Nachteil ist hier jedoch der hohe Preis zu nennen, der jedoch in unserem Fall keine große Rolle spielt, da wir nur ein Modell entwerfen und der Flügel nicht für hohe Stückzahlen konstruiert wird. Glasfasern sind im Vergleich zu Kohlenstofffasern die günstigere Variante, aber auf Glasfasern wird in Kapitel 2.1 noch mal genauer eingegangen.
\subsection{Herangehensweise}
Im ersten Kapitel werden zunächst verschieden Lösungsvarianten und allgemeine Informationen zum Thema GFK Materialien vorgestellt. Daraufhin wird eine Lösungsvariante festgelegt. Um die Anforderungen zu erfüllen werden in Kapitel 3 Handbuchmethoden verwendet um eine Grobauslegung des Flügels durchzuführen. In Kapitel 4 werden diese dann mit Hilfe der Finiten Elemente Methode verifiziert. In Kapitel 5 werden die Ergebnisse mit den Konstruktionen vorheriger studentischer Arbeiten verglichen.

\section{Theorie und Grundlagen}
\subsection{GFK}
\subsubsection{Fasern}
Glasfasern gelten als älteste synthetische Faserart und wurde schon vor 3500 Jahren verwendet. Heute werden Glasfasern aus größtenteils SiO2 und Metalloxiden hergestellt. Die Bestandteile werden bei ca. 1400°C aufgeschmolzen und durch kleine Düsen im Boden des Kessels als dünne Fäden ausgelassen. Die Fäden werden aufgewickelt und zu größeren Fasern verwebt. \cite {item3}\\
Die hohe Festigkeit der Glasfaser beruht auf den kovalenten Bindungen von Silicium (Si) und Sauerstoff (O) Atome. Zugesetzte Metalloxide verhindern eine Ausbildung eines geordneten Gefüges und erhöhen somit zusätzlich die Festigkeit. Die Fasern können in Längsrichtung sehr hohe Kräfte aufnehmen, jedoch nicht in Querrichtung. Dadurch werden sie in eine Matrix eingegossen, die die Querkräfte aufnimmt und die Faser vor dem Knicken schützt. Glasfasern lassen sich auch um enge Radien sehr gut drapieren und sind durch ihre einfache Herstellungsweise sehr preiswert. \cite{item4}\\
Durch die zuvor erläuterten Eigenschaften sind Glasfasern sehr gut für dieses Projekt geeignet, für einen größeren Flügel wäre jedoch der Elastizitätsmodul zu gering und es müsste auf andere Fasern, wie zum Beispiel Kohlefasern zurückgegriffen werden. \\
Für die Konstruktion des Flügels stehen die Glasfasern Interglas 90070 und Interglas 92145 des Herstellers Interglas Technologies zur Verfügung. 

\subsubsection{Matrix}
\subsection{Bauweisen} 
\textit{Ole} \\
In der Aufgabenstellung wird gefordert, dass der Flügel in der Holm-Bauweise konstruiert wird. Ein Holm besteht aus zwei parallelen Gurten, die durch einen oder mehrere Stege miteinander verbunden werden. Dabei bieten sich unterschiedliche Möglichkeiten (s. Abb1). Neben der Festigkeit ist die Steifigkeit die einzige Strukturmechanische Anforderung. Somit lässt sich das Problem als Kragbalken betrachten, der bei der vorgegebenen Prüflast (FPrüf = 100 N) am freien Ende die vorgegebene Durchbiegung (z100N = 22 mm) einhält. Das entstehende Biegemoment wird hauptsächlich von den Gurten getragen, weswegen man sich bei der Wahl des Steges auf andere Kriterien konzentrieren kann. Da keine maximale Torsion vorgegeben ist und die Torsionssteifigkeit fast ausschließlich von der Haut bewirkt wird, führen mehrere Stege, wie man sie bei einem geschlossenen Profil hat, nur zu unerwünschter Gewichtszunahme. Nach diesen Überlegungen haben wir uns für den I-Holm entschieden, da dieser bei einfacher Fertigung die gewünschten Eigenschaften mit sich bringt.\\
Das aerodynamische Profil des Flügels wird durch Schalenbauweise erreicht. Hierbei wir eine dünne Haut nur an kritischen Stellen mit der Sandwichbauweise beziehungsweise Rippen an den kritischen Stellen verstärkt, um Beulen zu verhindern. Die Schale trägt dabei so gut wie gar nicht die Last des Flügels, jedoch ist sie für die Torsionssteifigkeit entscheidend.


\section{Berechnungen}

\section{Ausdetaillierung und Verifizierung mittels FEM }

\section{Auswertung und Vergleich}

\section{Zusammenfassung und Ausblick}
\begin{thebibliography}{99}          
	\bibitem{item1}
	H. Hertel :
	\textit { Leichtbau:Bauelemente,Bemessungen und Konstruktionen von Flugzeugen u.a. }.
	Springer Verlag Berlin/Göttingen/Heidelberg , 1960.
	
	\bibitem{item2}
	Mises, R. V. :
	\textit { Fluglehre: Vorträge über Theorie und Berechnung der Flugzeuge in elementarer Darstellung }.
	Springer Verlag , 1936.
	
	\bibitem{item3}
	Helmut Schürmann :
	\textit { Konstruieren mit Faser-Kunststoff-Verbunden }.
	Springer Verlag , 2005.
	
	\bibitem{item4}
	Elmar Witten :
	\textit { Handbuch Faserverbundkunststoffe, Composites: Grundlagen, Verarbeitung, Anwendungen }.
	Springer Vieweg , 2014.
	
	
	



	
	
\end{thebibliography}






	
\end{document}
