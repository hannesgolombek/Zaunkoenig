\subsubsection{Beulsicherheit der Gurte (T.B.)}
Nachdem die Holmgurte auf Festigkeit und Steifigkeit ausgelegt worden sind, muss überprüft werden, ob der Effekt des Beulens, ein  Versagen aufgrund mangelnder Stabilität, auftritt.\\

\noindent Dazu werden folgende Annahmen getroffen:

\begin{enumerate}
	\item Die Berechnung erfolgt nach [1]
	\item Es wird angenommen, dass die orthotropen Gewebe hinreichend mit den Gleichungen für isotropes Material berechnet werden können. Diese Annahme wird mit erfolgten Zulassungen für Segelflugzeuge anhand dieser Formeln begründet.
	\item Die Gurte werden als ebene, unendlich lange Streifen betrachtet. Die tatsächliche Krümmung dieser in x-Richtung beeinflusst die Beulsicherheit positiv.
	\item Da die Mitte in $x$-Richtung der Holmgurte mit dem Holmsteg verklebt ist, kann diese Klebelinie als freie Lagerung gesehen werden. Somit halbiert sich die angenommene Holmgurtbreite.
	\item Die äußeren Kanten in $x$-Richtung sind frei und nicht gelagert.
	\item Die äußeren Kanten in $y$-Richtung werden an den jeweiligen Rippen gestützt.
	\item Der Druckgurt wird nur durch Druckspannungen beansprucht. Die Schubspannungen werden durch das hohe Verhältnis von Länge zu  Höhe vernachlässigt.
	\item Als größtmögliche Länge bei höchster Biegespannung wird $l_{3}$ bestimmt.
\end{enumerate}
Das Seitenverhältnis beträgt 
\begin{equation}
	\frac{b}{a}=\frac{\frac{28 mm}{2}}{773 mm}=0,018 \approx 0
\end{equation}
\noindent Nach [Hertel, Abbildung 84] ergibt sich:
\begin{equation}
	k_{d}=0,4
\end{equation}
und somit die kritische Spannung mit $E_{xx}=31580MPa$, $d=1,941mm$ und $b=14mm$:
\begin{equation}
	\sigma_{krit,d}=k_{d}\cdot E_{xx}\cdot\biggl(\frac{d}{b}\biggr)^{2}\\
	= 242,82 MPa
\end{equation}
Im Vergleich zu der tatsächlich maximal auftretenden Randfaserspannung der Gurte ergibt sich die Sicherheit gegen Beulen zu 
\begin{equation}
	j_{Gurt}=\frac{\sigma_{krit,d}}{\sigma_{b}}=\frac{242,81 MPa}{224,96 MPa}=1,08
\end{equation}


\subsubsection{Beulsicherheit des Steges (T.B.)}
\label{Beulsicherheit Steg}
Ebenfalls muss der Holmsteg nach der Auslegung hinsichtlich der Sicherheit gegen Beulen überprüft werden. Folgende Annahmen werden dafür getroffen:\\
\begin{enumerate}
	\item Die Berechnung erfolgt, wie bei der Berechnung der Holmgurte, nach [1].
	\item Es wird angenommen, dass die orthotropen Gewebe hinreichend mit den Gleichnugen für isotropes Material berechnet werden können, Diese Entscheidung wir ebenfalls mit erfolgten Zulassungen für Segelflugzeuge begründet.
	\item Der Steg wird als ebener, unendlich langer Streifen betrachtet.
	\item Die Verklebung des Steges wird als gestützte, gelenkige Lagerung an allen vier Kanten angenommen.
	\item Der Steg wird durch Biegung und Schubspannung beansprucht.
\end{enumerate}
Für den Steg müssen alle drei Bereiche der Holmauslegung auf die Beulsicherheit geprüft werden. \\
\noindent Im Folgenden wir die Beulsicherheit des Bereichs $I$ berechnet:\\
\noindent Das Seitenverhältnis ergibt sich zu 
\begin{equation}
	\frac{b}{a}=\frac{35,8 mm-2\cdot 1,941 mm}{76 mm}=0,042
\end{equation}
Dadurch lässt sich mit 
\begin{equation}
	\frac{\sigma_{max}}{\sigma_{min}}=-1
\end{equation}
(symmetrisch) nach [Hertel, Abbildung 85] der Beulfaktor ermitteln zu
\begin{equation}
	k_{b} = 21,8
\end{equation}
Da das Dickenverhältnis von Steglagen zu Schaumkern sehr klein ausgelegt werden soll, wird
\begin{equation}
	\kappa = 1
\end{equation}
definiert. Dadurch ergibt sich die kritische Biegespannung mit $E_{xx}=6639,8MPa$, $d=1,882mm$ und $b=35,8mm-2\cdot 1,941mm$ zu
\begin{equation}
	\sigma_{krit,B}=\kappa\cdot\ k_{b}\cdot E_{xx}\cdot\Bigl(\frac{d}{b}\Bigr)^{2}=503,24MPa
\end{equation}
Das Verhältnis der Biegespannung zur kritischen Biegespannung ist
\begin{equation}
	j_{1}=\frac{200,56 Mpa}{503,24MPa}=0,398
\end{equation}. Für den Schub wird nach Hertel der Beulfaktor zu
\begin{equation}
	k_{s}=5,5
\end{equation}
Damit wir die kritische Schubspannung mit $G_{xy}=8577,8MPa$zu 
\begin{equation}
	\tau_{krit}=\kappa\cdot k\cdot G_{xy}\cdot\Bigl(\frac{1,88 2mm}{35,8 mm - 2\cdot 1,941  mm}\Bigr)^{2}=164,02 MPa
\end{equation}
Die tatsächlich auftretende Schubspannung beträgt 
\begin{equation}
	\tau=\frac{3}{2}\cdot \frac{5085,5 N}{1,882 mm\cdot(35,8 mm-2\cdot 1,941 mm)}=126,99 MPa
\end{equation}
sodass das Verhältnis der Schubspannung zur kritischen 
\begin{equation}
	j_{2}=\frac{126,99 MPa}{164,02 MPa}= 0,774
\end{equation}
ergibt. Die Gesamtsicherheit beträgt nach [Hertel einfügen]
\begin{equation}
	j=\sqrt{\frac{1}{j_{1}^{2}+j_{2}^{2}}}=1,148
\end{equation}
Somit kann rückgeschlossen werden, dass dieser Bereich des Holmsteges schon ohne Schaumkern sicher gegen Beulen ist.\\

\noindent Nun wird der Bereich $II$ betrachtet:\\
Da keine innere Querkraft herrscht, kann die Sicherheit durch Biegung außer Acht gelassen werden. Die Sicherheit gegen Beulen ist demnach nur von dem Schub abhängig.
Das Seitenverhältnis beträgt 
\begin{equation}
	\frac{b}{a}=\frac{35,8mm - 2\cdot 1,941mm}{37mm}=0,863
\end{equation}
Damit ergibt sich der Beulfaktor zu 
\begin{equation}
	k_{s}=6,8
\end{equation}
und weiterhin wird mit 
\begin{equation}
	\kappa = 1 
\end{equation}
gerechnet. Damit lässt sich 
\begin{equation}
	 \tau_{krit} = k_{s}\cdot \kappa\cdot G_{xy}\cdot \Bigl(\frac{1,882mm}{35,8mm-2\cdot 1,941mm}\Bigr)^{2}= 202,79 MPa
\end{equation}
berechnen. Mit dem gleichen maximalen Schub
\begin{equation}
	\tau=126,99 MPa
\end{equation}
 wie in Bereich I kann somit die Sicherheit zu 
 \begin{equation}
 	j=\frac{202,79MPa}{126,99MPa}=1,59
 \end{equation}
bestimmt werden. Auch dieser Stegbereich $II$ ist gegen Beulen sicher.\\

\noindent Abschließend wird der verbliebene Bereich $III$ überprüft:\\
\noindent Erneut wird das Seitenverhältnis ermittelt zu 
\begin{equation}
	\frac{b}{a}=\frac{35,8mm - 2\cdot 1,941mm}{773mm}=0,041\approx 0
\end{equation}
Dadurch ist
\begin{equation}
	k_{d}=21,8
\end{equation}
nach \cite{item1} für 
\begin{equation}
	\frac{\sigma_{max}}{\sigma_{min}}=-1
\end{equation} 
(symmetrisch) und
\begin{equation}
	k_{s} = 4,8
\end{equation}
Für die Belastung auf Druck durch Biegung wirkt maximal die Schubspannung
\begin{equation}
	\sigma_{b} = \frac{500N\cdot 0,773m}{3,075406\cdot 10^{-8}m^{4}}\cdot\frac{0,0358m - 2\cdot 1,941\cdot 10^{-3}m}{2}=200,56 MPa
\end{equation}
für den Schub wirkt die maximale Schubspannung von
\begin{equation}
	\tau_{max}=\frac{3}{2}\frac{500N}{0,313mm\cdot(35,8mm-2\cdot 1,941mm)}=75,07MPa
\end{equation}
Die Sicherheit gegen Beulen berechnet sich nun zu 
\begin{equation}
	j=\sqrt{\frac{1}{(\frac{\sigma_{v}}{\sigma_{krit}})^{2}+(\frac{\tau}{\tau_{krit}})^{2}}}
\end{equation}
mit 
\begin{equation}
	\sigma_{krit}=k\cdot k\cdot E_{xx}\cdot \Bigl(\frac{0,313mm+x}{35,8mm -2\cdot 1,941mm}\Bigr)^{2}
\end{equation}
und
\begin{equation}
	\tau_{krit} = \kappa\cdot k_{s}\cdot G_{xy}\cdot\Bigl(\frac{0,313mm + x}{35,8mm -2\cdot 1,941mm}\Bigr)^{2}
\end{equation}
Dieses mal kann $\kappa=3$ genutzt werden, sofern eine ausreichende Schaumstoff-Dicke $x$ auftritt, dessen Lösung analytisch herausgefunden wird. Dabei wird eine Mindestdicke von 
\begin{equation}
	x=0,569mm
\end{equation}
ermittelt, allerdings soll ein Schaum der Dicke $x=2mm$ verbaut werden, um das exakte Anpassen der Bauteilmaße während der Fertigung zu vereinfachen. Damit ergibt sich eine Beulsicherheit im Bereich $III$ von
\begin{equation}
	j=6,88
\end{equation}
Damit die äußeren Steglagen eine konstante Stegdicke bilden, soll in den Bereichen $I$ und $II$ eine Schaumdicke von $x=0,431mm$ verbaut werden. Der geschliffene Schaum erhöht zudem in diesen Bereichen die Beulsicherheit, obwohl kein zusätzlicher Schaum benötigt wäre. Der Übergang der beiden Schaumdicken soll zudem nicht direkt an der Wurzelrippe beginnen, sondern erst $23mm$ zur Flügelspitze versetzt erfolgen, um direkt an dem Kraftangriffspunk $C$ die Sicherheit ebenfalls zu erhöhen.
