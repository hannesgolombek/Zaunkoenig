\subsubsection{Einführung in Stabilitäts-Überprüfungen}
Neben der Festigkeit und Steifigkeit ist es bei einer Konstruktion dünnwandiger Bauteile notwendig, sie auf Versagen durch mangelnde statische Stabilität zu überprüfen. Stabilität bedeutet, dass ein System bestimmte maximale Lasten erträgt, es bei schon kleinen Erhöhungen jedoch zu drastischen Systemänderungen, wie z.B. in der Tragfähigkeit, kommen wird. Bemerkbar ist dieses auch in Veränderungen der Bauteilgeometrie \cite{item16}.\\

\noindent Typische Versagensarten aufgrund zu hoher Druckbeanspruchungen bilden dabei das Knicken (eindimensionale Formänderung) und das Beulen (zweidimensionale Formänderung). Bei Letzterem haben auch Schubbeanspruchungen Einfluss darauf. Dabei entstehen wellenförmige Ausprägungen des Bauteils (in entsprechend vielen Dimensionen). Sie können mit unendlich vielen Eigenwerten beliebig viele Ausprägungen, einer Sinus-Funktion mit unterschiedlichen Frequenzen entsprechend, im Bauteil bilden. Die kritische Last wird für eine einfache Ausprägung definiert, da dieser Fall zuerst eintreten wird. Die Verformungen können auch plastisch auftreten, sodass nach Entlastung des Bauteils irreparable Schäden bleiben. Inwiefern ein Bauteil im geknickten bzw. gebeulten Zustand seine Eigenschaften beibehält, lässt sich kaum vorhersagen \cite{item16}.\\

\noindent Die Beulsicherheit der Tragflügel-Bauteile erfolgt nach Hertel \cite{item1}. Dabei handelt es sich um Theorien isotroper Stoffe. Dennoch werden sie hier für FVK angewendet, da im Orthotropie-Achsensystem gleiche Verhaltensweisen wie isotroper Stoffe gelten. Die Orthotropie ist in nachfolgenden Bauteilen hinreichend durch gewählte Belastungsrichtungen und Lagenaufbauten gegeben. Diese Annahme hat sich bei bisherigen Flugzeug-Konstruktionen und deren Zulassungen bewährt \cite{item21}.\\

\noindent In der Hautebene eines dünnwandigen Bauteils wird zwischen beulkritischen Spannungen unterschieden, die aus reiner Biegung, Druck und Biegung, oder reinem Schub (Indizes $b$/$d$/$s$) resultieren. Sofern Normal- und Schubspannungen gleichzeitig vorliegen, kann lediglich eine kombinierte Belastung als über- oder unterkritisch eingestuft werden. Die Spannungen und Beulausprägungen sind von folgenden Parametern abhängig \cite{item1}:
\begin{itemize}
	\item Seitenverhältnis $\frac{a}{b}$. Für einen Betrag gleich Null wird das Hautfeld \glqq Streifen\grqq\: genannt. 
	\item Randbedingungen an den Rändern des Hautfelds. Ränder können frei (alle Freiheitsgrade), gestützt (rotatorischer Freiheitsgrade) und fest (keine Freiheitsgrade )gelagert werden.
	\item relative Hautstärke $\frac{s}{b}$.
	\item Elastizitätsmodul $E$.
\end{itemize}
Dabei stehen $a$ und $b$ für die lange bzw. die kurze Seite des Hautfelds und $s$ für dessen Hautstärke. Mit dem Seitenverhältnis und den Randbedingungen ergibt sich ein Beulfaktor $k$, der, je nach vorherrschender Beanspruchung, in den Abbildungen \ref{fig: Hertel_Druck}, \ref{fig: Hertel_Biegung} und \ref{fig: Hertel_Schub}  ermittelt wird. \\
Die beulkritischen Normal- und Schubspannungen lassen sich nach
\begin{equation}
	\sigma_{krit}=k\cdot E\cdot\Big(\frac{s}{b}\Big)^{2}
\end{equation}
\begin{equation}
	\tau_{krit}=k\cdot E\cdot\Big(\frac{s}{b}\Big)^{2}
\end{equation}
berechnen. Die Sicherheit gegen Beulen ergibt sich aus dem Verhältnis zur tatsächlich anliegenden Spannung $\sigma_{max}$ bzw. $\tau_{max}$. Falls beide Spannungstypen gleichzeitig anliegen sollten, kann mit der Gleichung
\begin{equation}
	\Big(\frac{\tau}{\tau_{krit}}\Big)^{2}+\Big(\frac{\sigma}{\sigma_{krit}}\Big)^{2}=1^{2}=j^{2}
\end{equation}
in Abbildung \ref{fig: Hertel_Sicherheit} herausgefunden werden, ob es sich um einen überkritischen Zustand handelt \cite{item1}.\\

\noindent Generell werden Krümmungen und Versteifungen nachfolgend vernachlässigt. Allerdings sind für Sandwich-Bauteile Anpassungen durchzuführen:\\
Die Dicke $d$ beinhaltet neben den FVK-Lagen die Schaumstoffdicke. Dadurch, dass diese Lagen nun an den Außenseiten konzentriert sind, kann das Trägheitsmoment mit einem Faktor $\kappa$ bis auf das Dreifache maximal erhöht werden. Dieser wird jedoch wieder minimiert, sofern der Lagenaufbau unsymmetrisch und somit das Dickenverhältnis ungleich eins ist. Exakte Beträge können der Abbildung \ref{fig: Hertel_Sandwich} entnommen werden. Die beulkritischen Spannungen berechnen sich somit nach \cite{item1} zu
\begin{equation}
	\sigma_{krit}=\kappa\cdot k\cdot E\cdot\Big(\frac{d}{b}\Big)^{2}
\end{equation}
\begin{equation}
	\tau_{krit}=\kappa\cdot k\cdot E\cdot\Big(\frac{d}{b}\Big)^{2}
\end{equation}

\subsubsection{Beulsicherheit der Gurte}
Für die Auslegung der Holmgurte werden folgende Annahmen getroffen:
\begin{enumerate}
	\item Die Gurte werden als ebene unendlich lange Streifen betrachtet.
	\item Da die Mitte der Holmgurte in $x$-Richtung mit dem Holmsteg verklebt ist, kann diese Klebelinie als freie Lagerung gesehen werden. Somit halbiert sich die angenommene Holmgurtbreite.
	\item Die äußeren Kanten in $x$-Richtung sind frei und nicht gelagert.
	\item Die äußeren Kanten in $y$-Richtung werden an den jeweiligen Rippen gestützt.
	\item Der Druckgurt wird nur durch Druckspannungen beansprucht. Die Schubspannungen werden durch das hohe Verhältnis von Länge zu  Höhe vernachlässigt. Der auf Zug beanspruchte Gurt wird nicht beulen.
	\item Als größtmögliche Länge bei höchster Biegespannung wird $l_{3}$ bestimmt.
\end{enumerate}
Das Seitenverhältnis beträgt 
\begin{equation}
	\frac{b}{a}=\frac{\frac{28 mm}{2}}{773 mm}=0,018 \approx 0
\end{equation}
\noindent Es ergibt sich:
\begin{equation}
	k_{d}=0,4
\end{equation}
und somit die kritische Spannung mit $\hat{E}_{xx}=31580MPa$, $d=1,941mm$ und $b=14mm$:
\begin{equation}
	\sigma_{krit,d}=k_{d}\cdot \hat{E}_{xx}\cdot\biggl(\frac{d}{b}\biggr)^{2}\\
	= 242,82 MPa
\end{equation}
Im Vergleich zu der tatsächlich maximal auftretenden Randfaserspannung der Gurte (siehe Kapitel \ref{NachrechnungGurte} ) ergibt sich die Sicherheit gegen Beulen zu 
\begin{equation}
	j_{Gurt}=\frac{\sigma_{krit,d}}{\sigma_{b}}=\frac{242,81 MPa}{224,96 MPa}=1,08
\end{equation}


\subsubsection{Beulsicherheit des Steges}
\label{Beulsicherheit Steg}
Ebenfalls muss der Holmsteg nach der Auslegung hinsichtlich der Sicherheit gegen Beulen überprüft werden. Folgende Annahmen werden dafür getroffen:\\
\begin{enumerate}
	\item Der Steg wird teilweise für einzelne lange Bereiche als nahezu ebener unendlich langer Streifen betrachtet.
	\item Die Verklebung des Steges wird als gestützte gelenkige Lagerung an allen vier Kanten angenommen.
	\item Der Steg wird durch Biegung und Schubspannung beansprucht.
\end{enumerate}
Für den Steg müssen alle drei Bereiche der Holmauslegung auf die Beulsicherheit geprüft werden. \\

\noindent Im Folgenden wir die Beulsicherheit des Bereichs $I$ berechnet:\\
\noindent Das Seitenverhältnis ergibt sich zu 
\begin{equation}
	\frac{b}{a}=\frac{35,8 mm-2\cdot 1,941 mm}{76 mm}=0,042
\end{equation}
Dadurch lässt sich mit 
\begin{equation}
	\frac{\sigma_{max}}{\sigma_{min}}=-1
\end{equation}
(symmetrische Spannungsverteilung über Höhe) der Beulfaktor ermitteln zu
\begin{equation}
	k_{b} = 21,8
\end{equation}
Da das Dickenverhältnis von Steglagen zu Schaumkern sehr klein ausgelegt werden soll, wird sicherheitshalber
\begin{equation}
	\kappa = 1
\end{equation}
definiert. Dadurch ergibt sich die kritische Biegespannung mit $\hat{E}_{xx}=6639,8MPa$, $d=1,882mm$ und $b=35,8mm-2\cdot 1,941mm$ zu
\begin{equation}
	\sigma_{krit,B}=\kappa\cdot\ k_{b}\cdot \hat{E}_{xx}\cdot\Bigl(\frac{d}{b}\Bigr)^{2}=503,24MPa
\end{equation}
Mit
\begin{equation}
	\sigma_{b}=\frac{M_{b}\cdot {h_{i}}}{\tilde{I_{x}}\cdot 2}=\frac{0,773m\cdot500N\cdot (35,8mm - 2\cdot 1,941mm)}{2\cdot 3,075406\cdot 10^{-8} m^{4} }=200,56 MPa
\end{equation}
ist das Verhältnis der vorhandenen Biegespannung zur kritischen 
\begin{equation}
	j_{1}=\frac{503,24MPa}{200,56 Mpa}=2,512.
\end{equation}
 Für den Schub wird der Beulfaktor zu
\begin{equation}
	k_{s}=5,5
\end{equation}
Damit wir die kritische Schubspannung mit $\hat{G}_{xy}=8577,8MPa$ zu 
\begin{equation}
	\tau_{krit}=\kappa\cdot k_{s}\cdot \hat{G}_{xy}\cdot\Bigl(\frac{1,88 2mm}{35,8 mm - 2\cdot 1,941  mm}\Bigr)^{2}=164,02 MPa
\end{equation}
Die tatsächlich auftretende Schubspannung beträgt 
\begin{equation}
	\tau= \frac{3}{2}\cdot\frac{Q_{max}}{h_{i}\cdot s} =\frac{3}{2}\cdot \frac{5085,5 N}{1,882 mm\cdot(35,8 mm-2\cdot 1,941 mm)}=126,99 MPa
\end{equation}
sodass das Verhältnis der vorhandenen Schubspannung zur kritischen 
\begin{equation}
	j_{2}=\frac{164,02 MPa}{126,99 MPa}= 1,292
\end{equation}
ergibt. Die Gesamtsicherheit beträgt
\begin{equation}
	j=\sqrt{\frac{1}{\left(\frac{1}{j_{1}}\right)^{2}+\left(\frac{1}{j_{2}}\right)^{2}}}=1,148
\end{equation}
Somit kann rückgeschlossen werden, dass dieser Bereich des Holmsteges schon ohne zusätzlichen Schaumkern sicher gegen Beulen ist.\\

\noindent Nun wird der Bereich $II$ betrachtet:\\
Da keine innere Querkraft herrscht, kann die Sicherheit durch Biegung außer Acht gelassen werden. Die Sicherheit gegen Beulen ist demnach nur von dem Schub abhängig.
Das Seitenverhältnis beträgt 
\begin{equation}
	\frac{b}{a}=\frac{35,8mm - 2\cdot 1,941mm}{37mm}=0,863
\end{equation}
Damit ergibt sich der Beulfaktor zu 
\begin{equation}
	k_{s}=6,8
\end{equation}
und weiterhin wird mit 
\begin{equation}
	\kappa = 1 
\end{equation}
gerechnet. Damit lässt sich 
\begin{equation}
	 \tau_{krit} = k_{s}\cdot \kappa\cdot \hat{G}_{xy}\cdot \Bigl(\frac{1,882mm}{35,8mm-2\cdot 1,941mm}\Bigr)^{2}= 202,79 MPa
\end{equation}
berechnen. Mit dem gleichen maximalen Schub
\begin{equation}
	\tau=126,99 MPa
\end{equation}
 wie in Bereich I kann somit die Sicherheit zu 
 \begin{equation}
 	j=\frac{202,79MPa}{126,99MPa}=1,59
 \end{equation}
bestimmt werden. Auch dieser Stegbereich $II$ ist ohne Schaumstoff gegen Beulen sicher.\\

\noindent Abschließend wird der verbliebene Bereich $III$ überprüft:\\
\noindent Erneut wird das Seitenverhältnis ermittelt zu 
\begin{equation}
	\frac{b}{a}=\frac{35,8mm - 2\cdot 1,941mm}{773mm}=0,041\approx 0
\end{equation}
Dadurch ist
\begin{equation}
	k_{d}=21,8
\end{equation}
für 
\begin{equation}
	\frac{\sigma_{max}}{\sigma_{min}}=-1
\end{equation} 
(symmetrische Spannungsverteilung über die Steghöhe) und
\begin{equation}
	k_{s} = 4,8
\end{equation}
Für die Belastung auf Druck durch Biegung wirkt maximal die Spannung
\begin{equation}
	\sigma_{b} = \frac{500N\cdot 0,773m}{3,075406\cdot 10^{-8}m^{4}}\cdot\frac{0,0358m - 2\cdot 1,941\cdot 10^{-3}m}{2}=200,56 MPa
\end{equation}
Für den Schub wirkt die maximale Schubspannung von
\begin{equation}
	\tau_{max}=\frac{3}{2}\frac{500N}{0,313mm\cdot(35,8mm-2\cdot 1,941mm)}=75,07MPa
\end{equation}
Die Sicherheit gegen Beulen berechnet sich nun zu 
\begin{equation}
	j=\sqrt{\frac{1}{(\frac{\sigma}{\sigma_{krit}})^{2}+(\frac{\tau}{\tau_{krit}})^{2}}}
\end{equation}
mit 
\begin{equation}
	\sigma_{krit}=\kappa\cdot k\cdot \hat{E}_{xx}\cdot \Bigl(\frac{0,313mm+x}{35,8mm -2\cdot 1,941mm}\Bigr)^{2}
\end{equation}
und
\begin{equation}
	\tau_{krit} = \kappa\cdot k_{s}\cdot \hat{G}_{xy}\cdot\Bigl(\frac{0,313mm + x}{35,8mm -2\cdot 1,941mm}\Bigr)^{2}
\end{equation}
Dieses mal kann $\kappa=3$ genutzt werden, sofern eine ausreichende Schaumstoff-Dicke $x$ auftritt, dessen Lösung analytisch herausgefunden wird. Dabei wird eine Mindestdicke von 
\begin{equation}
	x=0,569mm
\end{equation}
ermittelt. Allerdings soll ein Schaum der Dicke $x=2mm$ verbaut werden, um das exakte Anpassen der Bauteilmaße während der Fertigung zu vereinfachen. Damit ergibt sich eine Beulsicherheit im Bereich $III$ von
\begin{equation}
	j=6,88
\end{equation}
Damit die äußeren Steg-Gewebelagen eine konstante Stegdicke bilden, soll in den Bereichen $I$ und $II$ dennoch eine Schaumdicke von $x=0,431mm$ verbaut werden. Der in Form geschliffene Schaum erhöht zudem in diesen Bereichen die Beulsicherheit, obwohl kein zusätzlicher Schaum benötigt wird. Der Übergang der beiden Schaumdicken soll zudem nicht direkt an der Wurzelrippe beginnen, sondern erst $23mm$ zur Flügelspitze versetzt erfolgen, um direkt an dem Kraftangriffspunkt $C$ die Sicherheit zusätzlich zu erhöhen.
