\subsection{Beulsicherheit der Gurte (T.B.)}
Nachdem die Holmgurte auf festigkeit und Steifigkeit ausgelegt worden sind, muss überprüft werden, ob der Effekt des Beulens auftritt.\\

\noindent Dazu werden folgende Annahmen getroffen:

\begin{enumerate}
	\item Die Berechnung erfolgt nach Hertel, "LEichtbau - Bauelemente, Bemessungen und Konstruktion von Flugzeugen und anderen Leichtbauwerken", Springer-Verlag, 1980, Kapitel 3.2
	\item Es wird angenommen, dass die orthotropen Gewebe hinreichend mit den Gleichungen für isotropes Material gerechnet werden kann. Diese Entscheidung wird mit erfolgten Zulassungen für Segelflugzeuge begründet anhand dieser Formeln.
	\item Die Gurte werden als ebene, unendlich lange Streifen betrachtet. Die tatsächliche Krümmung dieser beeinflusst die Beulsicherheit positiv.
	\item Da die Mitte (in $x$-Richtung) der Holmgurte mit dem Holmsteg verklebt ist, kann diese Klebelinie als freie Lagerung gesehen werden. Somit halbiert sich die angenommene Holmgurtbreite.
	\item Die äußeren Kanten (in $x$-Richtung) sind frei, nicht gelagert.
	\item Die äußeren Kanten (in $y$-Richtung) werde als an den jeweiligen Rippen gestützt.
	\item Der Gurt wird nur durch Druckspannungen beansprucht. Die Schubspannungen werden durch das hohe Verhältnis von Länge zu  Höhe vernachlässigt.
	\item Als größt mögliche Länge bei höchster Biegespannung wird $l_{3}$ bestimmt.
\end{enumerate}

\subsection{Beulsicherheit des Steges (T.B.)}