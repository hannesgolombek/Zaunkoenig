\subsection{Gewichtsnormalisiertes Festigkeitskriterium (O.S.)}
Um eine Vergleichbarkeit zwischen verschiedenen Flügeln zu schaffen wird die gewichtsnormalisierte Festigkeit
\begin{equation}
\label{GewichtF}
	P=\frac{m_{\mathrm{Belastung,max}}}{m_{\mathrm{Fluegel}}}
\end{equation}
definiert. Es wird also die Gewichtskraft als Masse $m_{\mathrm{Belastung,max}}$, die der Flügel im Testaufbau maximal aushält, ins Verhältnis mit der Flügelmasse $m_{\mathrm{Fluegel}}$ gesetzt. Auch wenn das Ziel war, einen möglichst leichten Flügel, der den in der Aufgabenstellung formulierten Anforderungen entspricht, zu konstruieren, wird somit berücksichtigt, dass mehr Material zwar eine höhere Masse mit sich bringt, aber er wahrscheinlich auch größere Lasten aushalten kann. Üblicher Weise würde man die Bruchlast im Teststand ermitteln, aber auf Grund der COVID-19 Pandemie ist es uns nicht möglich den Flügel zu bauen, geschweige denn zu testen.\\
	
\noindent Eine Möglichkeit wäre es $500$ Newton als Bruchlast anzunehmen, da wir im Rahmen der Projektarbeit nachgewiesen haben, dass unsere Konstruktion dies aushält. Da jedoch immer eher vorsichtige Annahmen getroffen und die errechneten Werte dann meist noch aufgerundet wurden, wäre die gewichtsnormalisierten Festigkeit weit unter dem voraussichtlich im Teststand ermittelten Wert. Somit wäre der Aussagewert sehr gering.
Eine bessere Möglichkeit bietet hier das FE-Modell. Dies ist zwar auch nicht perfekt, da zum einen die Einspannung auf Grund der Holzblöcke nicht perfekt realistisch modelliert werden kann. Die kritischste Stelle liegt nach Überlegungen und durch ABAQUS bestätigt am Ansatz des Holmstummels, also dem Flügelende, dass am Rumpf anliegt, bzw. an der Stelle, wo der GFK im Steg dicker wird. Diese Stellen liegen weit genug von der Einspannung entfernt, sodass die Werte als plausibel angenommen werden können. Zum anderen ist durch die Studentenversion von ABAQUS der Idealisierungsgrad auf $1000$ Knoten und somit auch die Genauigkeit der Ergebnisse beschränkt. Trotzdem ist dies mit den zur Verfügung stehenden Mitteln der beste Ansatz um die Bruchlast zu ermitteln, da hier als Einziges der Flügel als Ganzes modelliert wird.\\

\noindent Da nicht bekannt ist, in welchem Bereich das Material zuerst versagt, wird in Gurt, Steg und Schale jeweils der Ort der höchsten Belastung einzeln betrachtet. Da nur ein einheitlicher Wert für die Spannung der Elemente gegeben wird, diese in Realität aber nicht über alle Schichten konstant ist, wird die Dehnung betrachtet, da sie für alle identisch ist. Wird nun die Dehnung in die beiden Achsenrichtungen und der Schubwinkel in $elamX^{2}$ eingegeben, erhält man sofort die Sicherheit gegen den Bruch. Die Ergebnisse sind in Abbildung \ref{sicher-steg} bis \ref{sicher-gurt} zu erkennen, wobei es zwei verschiedene Werte für den Holm gibt, da er auf der einen Seite auf Druck und auf der anderen auf Zug beansprucht wird. Der untere Gurt hat demnach die geringste Sicherheit von $1,31$. Da die Belastungen im Flügel linear mit der anliegenden Kraft steigen, lässt sich der Faktor direkt auf die aufgebrachte Last von $500\mathrm{N}$ übertragen. Somit ergibt sich eine Bruchlast von 650N oder mit einer Erdbeschleunigung von $9,81\frac{\mathrm{m}}{\mathrm{s^2}}$
$$m_{\mathrm{Belastung,max}} = 66,26\mathrm{kg}. $$ Eine erneute FEM-Berechnung mit der gesteigerten Kraft liefert neue Dehnungen, die im Gurt wie erwartet eine Sicherheit von ungefähr $1$ ergeben. Da der kleinste Beulfaktor deutlich höher als diese Sicherheit ist, kommt es auch nicht bei der erhöhten Last zum Beulen.
Für das gewichtsnormalisierte Festigkeitskriterium ergibt sich nun also mit $m_{\mathrm{Fluegel}}$ aus der Massenabschätzung ungefähr ein Wert von
$$P = 181,04.$$
\subsection{Diskussion der Ergebnisse (H.G.)}
Die Ergebnisse der Messwerte sollen nun mit den Werten anderer Gruppen verglichen werden. Dafür wurde uns eine Excel-Tabelle eines vor einigen Jahren durchgeführten Versuchs zur Verfügung gestellt, sowie die Daten einer Parallelgruppe.\\
\noindent
Ein Aufgabenkriterium ist die Absenkung der Flügelspitze bei einer Belastung von $100N$. Unser Flügel erreichte dabei eine Absenkung von $17,34mm$. Aus den Daten der uns zur Verfügung stehenden Testreihe geht eine Absenkung von 40,833 $mm$ bei einer Last von $100N$ hervor.\\
\noindent
Ein weiteres Vergleichskriterium ist die Bruchlast, die bei unserem Flügel $ 650N $ beträgt. Die Bruchlast der Testreihe beträgt nur $434,63 N$. Da jedoch nicht nur die maximal ertragbare Last, sondern auch das Gewicht eine große Rolle spielt, wird das gewichtsnormalisierte Festigkeitskriterium herangezogen (Glg. \ref{GewichtF}). Für unseren Flügel ergibt sich dort ein Wert von $P=181,04$. Für den Vergleichsflügel ergibt sich mit $m_{Last,max}=44,305kg$ und $m_{ges}=0,699kg$ ein Wert von $P=66,38$.\\
\noindent
Zusätzlich kann der maximal auftretende Torsionswinkel verglichen werden. Unser Flügel weist bei einer Belastung von $ 650N $ einen Torsionswinkel von 3,01 Grad auf. Der Winkel des Vergleichsobjektes beträgt bei $434,63 N$  1,5 Grad. Diese Werte sind aufgrund der verschiedenen Kräfte jedoch nur schwer vergleichbar.\\
Allgemein lässt sich sagen, dass unser Flügel in fast allen Kategorien besser abschneidet. Jedoch kann dies auch durch zwangsläufige Ungenauigkeiten in der Herstellung verursacht sein, die in unserem theoretischen Modell nicht berücksichtigt werden können.\\
\noindent
Etwas aussagekräftiger ist der Vergleich mit der Parallelgruppe, die die gleiche Aufgabenstellung bekommen hat.\\
\noindent
Die Absenkung bei einer Last von 100N beträgt dort $Z_{100N}=12,91mm$. Ein Bruch ergibt sich voraussichtlich bei $500N$, wodurch sich zusammen mit der Masse von $m=0,44375kg$ ein Wert von $P=112,68$ berechnen lässt. Alles in Allem kann man direkt erkennen, dass dieser Flügel zwar eine höhere Steifigkeit, aber eine niedrigere Festigkeit aufweist.\\

\begin{center}
\begin{tabular}[h]{l|c|c|c}
Kategorie&Eigene Ergebnisse&Parallelgruppe&Praxisgruppe\\
\hline
Absenkung $100N$&17,34$mm$&12,91$mm$&40,83$mm$\\
Bruchlast&650$N$&500$N$&434,63$N$\\
Masse&366$g$&443,75$g$&699$g$\\
$P$&181,04&112,68&66,38\\
max. Torsionswinkel&$3,01^{\circ}$& - &$1,5^{\circ}$
\end{tabular}
\end{center}


\subsection{Ausblick auf Optimierungen der Berechnungen (H.G.)}
Da für die Auslegung viele Annahmen und Vereinfachungen getroffen wurden, lassen sich einige Optimierungsmöglichkeiten für das Projekt nennen.\\
\noindent
Zunächst ließe sich die Dicke der Gurte über die Holmlänge als nicht konstant annehmen um zusätzlich das Gewicht zu verringern. Da dies jedoch schwer bis nicht analytisch zu lösen wäre und einen signifikanten Mehraufwand für eine relativ kleine Gewichtsänderung mit sich gebracht hätte, wurde in dieser Arbeit davon abgesehen.\\
\noindent
Die Wurzelrippe wurde mit einer definierten Dicke als gegeben angesehen. Hier wäre ebenfalls eine Möglichkeit, um evtl. Gewicht einzusparen und die Absenkung zusätzlich zu verringern. Da dies jedoch zusätzliche Einarbeitung in die Holztechnik und deutlich mehr Zeitaufwand mit sich bringen würde, wurde hier davon abgesehen.\\
\noindent
Die Berechnung des FKV wurde mittels der Mischungsregel durchgeführt. In alternativer Literatur wurde dies beispielsweise nach Puck berechnet. Dadurch könnten sich genauere Materialkennwerte ergeben, was zu einer möglichen Verringerung der GFK-Schichten und somit des Gewichts führen könnte. Dies würde aber wiederum zu einer deutlich aufwändigeren Rechnung führen, deren Ertrag zu gering ist, um dies zu rechtfertigen.\\
\noindent
In mehreren Berechnungen wurden schwächere Annahmen getroffen, als die Realität widerspiegelt. So wurde in der Auslegung des Holms und der Haut die Steifigkeit des Schaums nicht mit einberechnet, wodurch etwas Leichtbaupotential verschenkt wurden.\\
\noindent
Es wurde in vielen Fällen mit einer Sicherheit gerechnet. Da in den Annahmen und in den Materialkennwerten schon eine gewisse Ungenauigkeit und diese oft kleiner als in der Realität sind, können die Sicherheitswerte herabgesetzt, oder sogar gleich 1 gesetzt werden.\\
\noindent
Die FEM-Modellierung ist ebenfalls in vielerlei Hinsicht optimierbar. Durch die Studentenversion kann das Bauteil nicht so genau vernetzt werden, wie es für eine genau Auswertung nötig wäre. Zusätzlich sind die Holzblöcke am Holmstummel nicht modelliert worden, da dies sehr aufwändig gewesen wäre.\\
\noindent