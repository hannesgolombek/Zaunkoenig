\begin{longtable}[l]{l}
\onehalfspacing
\textbf{Lateinische Buchstaben}
\end{longtable}
\begin{longtable}[l]{ll}
\onehalfspacing
$ A $&Festlager\\
&in Bezug auf Wirkebene (hoch gestellt)\\
&Fläche\\
&Scheibensteifigkeit\\
$ B $&Loslager\\
$ C $&Krafteinleitung der Querkraftbolzen\\
$ E $&Elastizitätsmodul (allgemein)\\
$ F $&Kraft (allgemein)\\
$ G $&Schubmodul (allgemein)\\
$ I $&Flächenträgheitsmoment (allgemein)\\
$ K $&Dimensionierungskennwert nach \cite{item5},\\
 &Elementsteifigkeitsmatrix (zweifach unterstrichen)\\
$ Q $&Querkraft (allgemein)\\
&Steifigkeit Einzelschicht\\
$ R $&Integrationskonstante (allgemein)\\
&Festigkeit\\
$ P $&Gewichtsnormalisierte Festigkeit\\
$ S $&Statisches Moment\\
$ T $&Bei Torsion\\
& \\
$ a $&Lange Seite des Streifens nach \cite{item1}\\
$ b $&Kurze Seite des Streifens nach \cite{item1}\\
&Biegung\\
$ d$& Druck\\
$ f $&Anstrengung (mit E als Index)\\
$ h $&Höhenabmessung des Holmes (allgemein)\\
$ j $&Sicherheitsfaktor\\
$ l $&Längenabmessung des Holmes (allgemein)\\
$ n $&Lagenanzahl\\
&Normalkraftfluss\\
$ p $&Steigungs-/Neigungsparameter\\
$ q $&Schubfluss\\
$ \bar{q} $&Flächengewicht nach \cite{item5}\\
$ s $&Dicke des Streifens nach \cite{item1}\\
&Laufvariable in der y-z-Ebene\\
& Schub\\
$ t $&Dicke des Verbunds nach \cite{item3}\\
$ w $&Absenkung des Balkens unter Prüfkraft\\
$ x $&x-Koordinate in Flugzeuglängsrichtung\\
$ y $&y-Koordinate in Spannweitenrichtung, positiv Richtung von Rumpf zur Endrippe\\
$ z $&z-Koordinate der Rechtssystems\\
\end{longtable}
\begin{longtable}[l]{l}
	\textbf{Griechische Buchstaben}\\
\end{longtable}
\begin{longtable}[l]{ll}
\onehalfspacing
$\alpha$&Drehwinkel Fasern\\
$\beta$&Drehwinkel Fasern\\
$ \epsilon $&Dehnung (allgemein)\\
$ \eta $&Schwächungsfaktor\\
$ \vartheta $&Verwindung\\
$ \theta $ &Winkel der Bruchebene\\
$ \varphi $&Faservolumengehalt\\
&Winkel zwischen Hauptachsen- und Schwerpunkt-Koordinatensystem\\
$ \phi $&Orientierungsabweichung, Drillwinkel\\
$ \kappa $&Steifigkeitserhöhung des Sandwich nach \cite{item1}\\
$ \rho $&Dichte (allgemein)\\
$ \sigma $&Zug-/Druckspannung (allgemein)\\
$ \tau $&Schubspannung (allgemein)\\
\end{longtable}
\begin{longtable}[l]{l}
\onehalfspacing
\textbf{Indizes}\\
\end{longtable}
\begin{longtable}[l]{ll}
\onehalfspacing
$ 11 $&Faserhauptrichtung\\
$ 22 $&Fasernebenrichtung\\
$ 12 $& Richtung \glqq Wirkung (1) - Ursache (2)\grqq\\
$ \parallel $&Parallel zur Faser\\
$ \perp $&Senkrecht zur Faser\\
$ \perp\parallel $& Belastungsrichtung \glqq Wirkung ($\perp$) - Ursache ($\parallel$)\grqq\: ($bei\: eLamX^{2}$ umgekehrt)\\
$ \# $&Unter 45° zur Faserrichtung\\
$ + $&Bei Zugbeanspruchung\\
$ - $&Bei Druckbeanspruchung\\
$ \bar{()} $&Bei Schwerpunkt-Koordinaten\\
$ \hat{()} $&Bei Hauptachsen-Koordinaten\\
$ I $ &Holmbereich zwischen A und B\\
$ II $ &Holmbereich zwischen B und C\\
$ III $ &Holmbereich zwischen C und Flügelspitze\\
$ f $&Faser\\
$ m $&Matrix\\
$ max$& maximal\\
$ min$& mindestens\\
$ pruef$& Prüfkraft\\
$ Q $&Kraft der Querkraftbolzen
$ zul $&zulässig\\
\end{longtable}
\begin{longtable}[l]{l}
\onehalfspacing
\textbf{Abkürzungen}
\end{longtable}
\begin{longtable}[l]{ll}
\onehalfspacing
$ BD $& bidirektional\\
$Fb$&Faserbruch\\
$FKV$&Faser-Kunststoff-Verbund\\
$GFK$& Glasfaser-Kunststoffverbund\\
 $ KOS$& Koordinatensystem\\
$MSV$&Mehrschichtverbund\\
$UD$&unidirektional\\
$Zfb$&Zwischenfaserbruch\\
&\\
$H.G.$& verfasst von Hannes Golombek\\
$H.K.$& verfasst von Henri Kammler\\
$O.S.$& verfasst von Ole Scholz\\
$T.B.$& verfasst von Tristan Brack\\
\end{longtable}

