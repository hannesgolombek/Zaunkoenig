\textbf{Bezeichnungen}
\begin{table}[h]

\begin{tabular}{ll}
	$ A $&Festlager, (hoch gestellt) in Bezug auf Wirkebene, (mit 0 als Index) umschlossene Fläche\\
	$ B $&Loslager, Bei Biegung\\
	$ C $&Krafteinleitung der Querkraftbolzen\\
	$ E $&Elastizitätsmodul (allgemein)\\
	$ F $&Kraft (allgemein)\\
	FKV&Faser-Kunststoff-Verbund\\
	$ G $&Schubmodul (allgemein)\\
	GFK& Glasfaser-Kunstoffverbund\\
	$ I $&Flächenträgheitsmoment (allgemein)\\
	$ K $&Dimensionierungskennwert nach [QUELLE VDI einfügen], (zweifach unterstrichen) Elementsteifigkeistmatrix\\
	$ Q $&Querkraft (allgemein)\\
	$ R $&Integrationskonstante (allgemein), Festigkeit\\
	$ P $&Gewichtsnormalisierte Festigkeit\\
	$ S $&Statisches Moment\\
	$ T $&Bei Torsion\\
	UD&unidirektional\\
	Fb&Faserbruch\\
	Zfb&Zwischenfaserbruch\\
	
	
\end{tabular}
\end{table}

\begin{table}[h]
	\begin{tabular}{ll}
		$ a $&Lange Seite des Streifens nach \cite{item1}\\
		$ b $&Kurze Seite des Streifens nach \cite{item1}, bei Biegung\\
		$ f $&(mit E als Index) Anstrengung\\
		$ h $&Höhenabmessung des Holmes (allgemein)\\
		$ j $&Sicherheitsfaktor\\
		$ l $&Längenabmessung des Holmes (allgemein)\\
		$ n $&Lagenanzahl, Normalkraftfluss\\
		$ p $&Steigungs-/Neigungsparameter\\
		$ q $&Schubfluss\\
		$ \bar{q} $&Flächengewicht nach [Quelle VDI einfügen]\\
		$ s $&Dicke des Streifens nach \cite{item1}, Laufvariable in der	y-z-Ebene\\
		$ t $&Dicke des Verbunds nach \cite{item3}\\
		$ w $&Absenkung des Balkens unter Prüfkraft\\
		$ x $&x-Koordinate in Flugzeuglängsrichtung\\
		$ y $&y-Koordinate in Holmrichtung, positiv Richtung Endrippe\\
		$ z $&z-Koordinate der Rechtssystems\\
		
	\end{tabular}
\end{table}

\begin{table}[h]
	\begin{tabular}{ll}
		$ \epsilon $&Dehnung (allgemein)\\
		$ \eta $&Schwächungsfaktor\\
		$ \vartheta $&Verdrillung\\
		$ \theta $ &Winkel der Bruchebene\\
		$ \varphi $&Faservolumengehalt, Winkel zwischen Hauptachsen- und Schwerpunkt-Koordinatensystem\\
		$ \phi $&Orientierungsabweichung\\
		$ \kappa $&Steifigkeitserhöhung des Sandwich nach \cite{item1}\\
		$ \rho $&Dichte (allgemein)\\
		$ \sigma $&Zug-/Druckspannung (allgemein)\\
		$ \tau $&Schubspannung (allgemein)\\
	\end{tabular}
\end{table}

\begin{table}[h]
	\begin{tabular}{ll}
		$ 11 $&Faserhauptrichtung\\
		$ 22 $&Fasernebenrichtung\\
		$ 12 $&Wie heißt diese Richtung ?????\\
		$ \parallel $&Parallel zur Faser\\
		$ \perp $&Senkrecht zur Faser\\
		$ \# $&Unter 45° zur Faserrichtung\\
		$ + $&Bei Zugbeanspruchung\\
		$ - $&Bei Druckbeanspruchung\\
		$ \bar{ } $&Bei Schwerpunkt-Koordinaten\\
		$ \hat{ } $&Bei Hauptachsen-Koordinaten\\
		$ I $ &Holmbereich zwischen A und B\\
		$ II $ &Holmbereich zwischen B und C\\
		$ III $ &Holmbereich von C bis Flügelspitze\\
		$ zul $&zulässig
%		1 bis 10? & Bereiche Schubmittelpunktberechnung????
		
	\end{tabular}
\end{table}

