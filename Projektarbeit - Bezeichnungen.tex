\textbf{Lat. Großbuchstaben:}\\
\begin{table}

\begin{tabular}{ll}
	$ A $&Festlager\\
	$ B $&Loslager\\
	$ C $&Krafteinleitung der Querkraftbolzen\\
	$ E $&Elastizitätsmodul (allgemein)\\
	$ F $&Kraft (allgemein)\\
	$ G $&Schubmodul (allgemein)\\
	$ I $&Flächenträgheitsmoment (allgemein)\\
	$ K $&Dimensionierungskennwert der VDI 2013\\
	$ Q $&Querkraft (allgemein)\\
	$ R $&Integrationskonstante (allgemein)\\
	
\end{tabular}
\end{table}

\begin{table}
	\begin{tabular}{ll}
		$ a $&Lange Seite des Streifens nach \cite{item1}\\
		$ b $&Kurze Seite des Streifens nach \cite{item1}\\
		$ h $&Höhenabmessung des Holmes (allgemein)\\
		$ j $&Sicherheitsfaktor\\
		$ l $&Längenabmessung des Holmes (allgemein)\\
		$ n $&Lagenanzahl\\
		$ q $&Schubfluss\\
		$ \bar{q} $&Flächengewicht nach VDI 2013\\
		$ s $&Dicke des Streifens nach \cite{item1}\\
		$ t $&Dicke des Verbunds nach \cite{item3}\\
		$ w $&Absenkung des Balkens unter Prüfkraft\\
		$ x $&x-Koordinat in Flugzeuglängsrichtung\\
		$ y $&y-Koordinate in Holmrichtung, positiv Richtung Endrippe\\
		$ z $&z-Koordinate der Rechtssystems\\
		
	\end{tabular}
\end{table}

\begin{table}
	\begin{tabular}{ll}
		$ \epsilon $&Dehnung (allgemein)\\
		$ \phi $&Faservolumengehalt\\
		$ \kappa $&Dickenverhältnis des Sandwich nach \cite{item1}\\
		$ \rho $&Dichte (allgemein)\\
		$ \sigma $&Zug-/Druckspannung (allgemein)\\
		$ \tau $&Schubspannung (allgemein)\\
	\end{tabular}
\end{table}

\begin{table}
	\begin{tabular}{ll}
		$ 11 $&Faserhauptrichtung\\
		$ 22 $&Fasernebenrichtung\\
		$ \parallel $&Parallel zur Faser\\
		$ \perp $&Senkrecht zur Faser\\
		$ \# $Unter 45° zur Faserrichtung\\
		$ + $Bei Zugbeanspruchung\\
		$ - $Bei Druckbeanspruchung\\
		 
	\end{tabular}
\end{table}


\noindent\textbf{Zahlen:}\\
$0$\quad\quad von Holmstummelspitze bis A\
$1$\quad\quad von A bis B\\
$2$\quad\quad von B bis C\\
$3$\quad\quad von C bis Flügelspitze\\
$11$\quad\quad Faserhauptrichtung\\
$22$\quad\quad Fasernebenrichtung\\
\\
\noindent\textbf{Sonderzeichen:}\\
$\parallel$\quad\quad parallel\\
$\perp$\quad\quad orthogonal\\
$\#$\quad\quad unter 45°\\
$+$\quad\quad Zug\\
$-$\quad\quad Druck\\
$I$\quad\quad Bereich von A bis B\\
$II$\quad\quad Bereich von B bis C\\
$III$\quad\quad Bereich von C bis Flügelspitze\\
\\
\noindent\textbf{Wörter:}\\
$pruef$\quad\\
$max$\quad\\
$min$\quad\\
$Gurt$\quad\\
$Steg$\quad\\
$krit$\quad\\