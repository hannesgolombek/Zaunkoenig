\subsection{Warum FEM?}
Nachdem der Flügel nun analytisch ausgelegt wurde, stellt sich die Frage, ob eine numerische Herangehensweise hier überhaupt noch sinnvoll ist. In den vorherigen Kapiteln wurde viele Vereinfachungen angenommen, um die Berechnungen mit verhältnismäßigen Aufwand zu bewältigen. Je komplexer ein Beiteil ist, desto unwirtschaftlicher wird es, dies in seiner Fülle analytisch zu berechnen oder gar unmöglich, wenn keine geschlossenen Lösungen bekannt sind. Im Leichtbau werden diese Details benutzt, um Beiteile an de Stellen zu verstärken, wo besonders große Lasten auftreten (z.B. Rippen). Wenn diese nun für eine einfachere Berechnung wegfallen, muss das Restliche Bauteil robuster ausgelegt werden, was zu einer vermeidbaren Gewichtszunahme führt. Auch wenn es sich bei numerischen Lösungen nur um Annäherungen an den wahren Zustand handelt, kann durch einen hohen Vernetzungsgrad ein präziseres Ergebnis erzielt werden als das vereinfachte Analytische. Somit übernimmt die nummerische Berechnung auch eine Kontrollfunktion.
\subsection{Wie funktioniert FEM?}
\subsubsection{Schwache Lösung der Elastostatik}
Für die Berechnung der Elastostatik sind die Gleichgewichtsbedingung (\ref{GGW}),  Verzerrungs-Verschiebungsbedingung (\ref{VVB}) und das Stoffgesetz (\ref{SG}) auch bei der Finiten Elemente Methode (FEM) ausschlaggebend.
\begin{equation}\label{GGW}
	\underline{0} = \underline{X} + \underline{\underline{E}}^T \sigma 
\end{equation}
\begin{equation}\label{VVB}
	\underline{\epsilon} = \underline{\underline{D}} \underline{u}
\end{equation}
\begin{equation}\label{SG}
	\underline{\sigma} = \underline{\underline{E}} \underline{\epsilon}
\end{equation}
Wobei \underline{$X$} der Vektor der Volumenkräfte, \underline{\underline{$E$}} die Steifigkeitsmatrix, \underline{$\epsilon$} der Verzerrungsvektor, \underline{$u$} der Verschiebungsvektor und \underline{\underline{$D$}} die Operatormatrix ist.\\
Um einer aufwendigen Bestimmung der analytischen Lösung zu entgehen, bedient sich die FEM an dem Prinzip der \textit{schwachen Lösung}. Hierbei hat man eine Differenzialgleichung, die in dem betrachteten Gebiet gleich null ist. Für die Elastostatik kann man hierbei die Gleichgewichtsbedingung verwenden. Diese kann man mit $\delta\underline{u}$ multiplizieren und über das Gebiet integrieren, sodass man
\begin{equation}
	\int_{V}^{}\delta\underline{u}^T\underline{\underline{D}}^T\underline{\sigma}\,dV + \int_{V}^{}\delta\underline{u}^T\underline{X}\,dV = 0
\end{equation}
erhält. Umgeformt ergibt sich das zu
\begin{equation}\label{sV}
	\int_{V}^{}\delta\underline{u}^T\underline{\underline{D}}^T\underline{\underline{E}}\underline{\underline{D}}\underline{u}\,dV = \int_{O_{p}}^{}\delta\underline{u}^T\underline{p}\,dO_p + \int_{V}^{}\delta\underline{u}^T\underline{X}\,dV
\end{equation}
Wobei die Terme auf der rechten Seite den Lasten entsprechen, die auf das Volumen $V$ und die mit $p$ belastete Oberfläche $O_{p}$ wirken. Die schwer zu lösende Differenzialgleichung hat sich nun schon zu einem Integrationsproblem vereinfacht. Diese Gleichung ist noch ganz allgemein fürs Kontinuum gültig. Im nächsten Schritt wird das Gebiet in eine finite Menge von Elementen zerteilt.
\subsubsection{Diskretisierung}