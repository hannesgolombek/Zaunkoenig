\subsection{Vorteile der Finiten Element Methode (O.S.)}
Nachdem der Flügel nun analytisch ausgelegt wurde, stellt sich die Frage, ob eine numerische Herangehensweise hier überhaupt noch sinnvoll ist. In den vorherigen Kapiteln wurden viele Vereinfachungen angenommen, um die Berechnungen mit verhältnismäßigen Aufwand zu bewältigen. Je komplexer ein Bauteil ist, desto unwirtschaftlicher wird es, dieses in seiner Fülle analytisch zu berechnen oder gar unmöglich, wenn keine geschlossenen Lösungen bekannt sind. Im Leichtbau werden diese Details benutzt, um Bauteile an den Stellen zu verstärken, wo besonders große Lasten auftreten (z.B. Rippen). Wenn diese nun für eine einfachere Berechnung wegfallen, muss das restliche Bauteil robuster ausgelegt werden, was zu einer vermeidbaren Gewichtszunahme führt. Auch wenn es sich bei numerischen Lösungen nur um Annäherungen an den wahren Zustand handelt, kann durch einen hohen Vernetzungsgrad ein präziseres Ergebnis erzielt werden als das vereinfachte analytische. Somit übernimmt die numerische Berechnung auch eine Kontrollfunktion.
