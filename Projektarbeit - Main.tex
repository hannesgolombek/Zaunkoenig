\documentclass[a4paper,oneside,11pt]{article}
\usepackage[ngerman]{babel}
\usepackage[T1]{fontenc}
\usepackage[right=2cm,left=3cm]{geometry}
\usepackage{graphicx}
\usepackage{pdfpages}
\usepackage{tabto}
\usepackage{setspace}
\usepackage{amsmath}
\usepackage{longtable}
\usepackage[T1]{fontenc}
\usepackage[headsepline]{scrlayer-scrpage}
\usepackage{placeins}
\usepackage[colorlinks=true,
linkcolor=black,
filecolor=black,      
urlcolor=black,
citecolor=black,
pdftitle={Auslegung eines alternativen Modellflügels für das Flugzeug „Zaunkönigin Glasfaserverbund-Holm-Bauweise},
bookmarks=true,]{hyperref}

\begin{document}
\numberwithin{equation}{section}
\numberwithin{figure}{section}
\onehalfspacing
\pagestyle{empty}	
\begin{center}
\begin{large}
	\textbf{Projektarbeit}\\

	\textbf{Auslegung eines alternativen Modellflügels für das Flugzeug „Zaunkönig“ in Glasfaserverbund-Holm-Bauweise}
	\textbf{Bericht zum 2. Review}\\
\end{large}

Hannes Golombek\\
Ole Scholz\\
Henri Kammler\\
Tristan Brack\\
Betreut von Malte Woidt\\
11.12.2020\\
\end{center}


 
\includepdf[pages=-]{PDFs/Aufgabenstellung Gruppe 2.pdf}

\pagestyle{scrheadings}
\clearscrheadfoot
\ihead{Übersicht/Abstract}
\ohead{ }


\noindent\large{\textbf{Abstract (O.S.)}}~\\

\noindent In this project work a model halve wing of the aircraft “Zaunkönig” will be constructed scaled down 1:4.7 as a glass fiber spar structure. The wing will be construed for strength and bending stiffness using analytical methods as much as possible, completed by numerical methods. The Programs ABAQUS and $ eLamX^{2} $ will be used in assistance. Afterward a short manuel to build the wing is given. For a simulated test with an eccentric force the mass, breaking load, shear center and as well lowering as torsion of the tip of the wing will be determined. Under special consideration of the weight normalized strength criterion the wing will be rated in relation to comparison data and the calculated values critically analyzed. 
\newpage
\clearscrheadfoot
\ohead{Seite \pagemark}
\ihead{\headmark}
\automark{section}
%\cfoot{Auslegung eines alternativen Modellflügels für das Flugzeug „Zaunkönig“ in Glasfaserverbund-Holm-Bauweise}

\newpage
\setcounter{page}{1}
\tableofcontents
\newpage

\section{Bezeichnungen}
\begin{tabbing}

\textbf{Lat. Großbuchstaben:}\\
$A$\quad\quad Festlager,!\\
$B$\quad\quad Loslager\\
$C$\quad\quad Position Wurzelrippe\\
$E$\quad\quad E-Modul\\
$F$\quad\quad Kraft\\
$G$\quad\quad Schubmodul\\
$I$\quad\quad Steifigkeit\\
$K$\quad\quad Rechen-Konstante\\
$L$\quad\quad \\
$Q$\quad\quad Querkrft\\
$R$\quad\quad Integrtionskonstante, Spannung\\
\\
\noindent\textbf{Lat. Kleinbuchstaben:}\\
$a$\quad\quad Länge Seitenverhältnis\\
$b$!\quad\quad Breite Seitenverhältnis\\
$f$\quad\quad Faser\\
$h$\quad\quad Höhe\\
$i$\quad\quad \\
$j$\quad\quad Sicherheit\\
$l$\quad\quad Länge\\
$m$\quad\quad Matrix\\
$n$\quad\quad Anzahl\\
$q$\quad\quad Schub\\
$\bar{q}$\quad\quad\\
$s$\quad\quad !,!\\
$t$\quad\quad Dicke\\
$w$\quad\quad Absenkung\\
$x$\quad\quad x-Koordinate\\
$y$\quad\quad y-Koordinate\\
$z$\quad\quad z-Koordinate\\
\\
\noindent\textbf{Griech. Großbuchtaben:}\\
\\
\noindent\textbf{Griech. Kleinbchstaben:}\\
$\epsilon$ \quad\quad Dehnung\\
$\phi$\quad\quad Mischungsverhältnis\\
$\kappa$\quad\quad Dickenverhältnis Sandwich\\
$\rho$\quad\quad Dichte\\
$\sigma$\quad\quad Zug-/Druckspannung\\
$\tau$\quad\quad Schubspannung\\
\\
\noindent\textbf{Zahlen:}\\
$0$\quad\quad von Holmstummelspitze bis A\
$1$\quad\quad von A bis B\\
$2$\quad\quad von B bis C\\
$3$\quad\quad von C bis Flügelspitze\\
$11$\quad\quad Faserhauptrichtung\\
$22$\quad\quad Fasernebenrichtung\\
\\
\noindent\textbf{Sonderzeichen:}\\
$\parallel$\quad\quad parallel\\
$\perp$\quad\quad orthogonal\\
$\#$\quad\quad unter 45°\\
$+$\quad\quad Zug\\
$-$\quad\quad Druck\\
$I$\quad\quad Bereich von A bis B\\
$II$\quad\quad Bereich von B bis C\\
$III$\quad\quad Bereich von C bis Flügelspitze\\
\\
\noindent\textbf{Wörter:}\\
$pruef$\quad\\
$max$\quad\\
$min$\quad\\
$Gurt$\quad\\
$Steg$\quad\\
$krit$\quad\\
\newpage
\section{Einleitung}

\subsection{Projektbeschreibung}
Der Zaunkönig ist ein in den frühen 1940er Jahren entstandenes Flugzeug, das unter der Leitung von Hermann Winter an der Technischen Hochschule Braunschweig konstruiert wurde. Da der Zaunkönig vornehmlich aus Holz gebaut wurde, soll jetzt ein neuer Flügel im Maßstab 1:4,7 aus Glasfaser-Kunststoffverbund (GFK) konstruiert werden. Der Flügel muss gewisse Anforderungen erfüllen, die im Folgenden definiert werden.
Bei der Tragfläche handelt es sich um einen Rechteckflügel, der im Original über Verstrebungen mit dem Rumpf verbunden ist. Diese Streben sollen in der neuen Konstruktion nicht vorhanden sein. Der Flügel soll im Rumpf verstiftet werden, wobei die Torsionsbelastung durch Querkraftbolzen aufgenommen wird. Insgesamt darf der Flügel das Gewicht von 0,750 kg nicht überschreiten.
Um die strukturmechanischen Anforderungen zu erfüllen wird der Flügel auf seine Steifigkeit und Festigkeit geprüft. Die Steifigkeit ist hinreichend, wenn der Flügel bei einer senkrechten Belastung von $ F_{pruef}=100N $ an der Endrippe eine Durchbiegung von $ w=22mm $ nicht überschreitet. Außerdem darf der Flügel bei einer Prüfkraft von $ F_{pruef}=500N $ nicht brechen. Die Haut muss so ausgelegt sein, dass kein Beulen auftritt. Zusätzlich müssen der Torsionswinkel und Schubmittelpunkt berechnet werden.
\subsection{Motivation (O.S.)}
Zunächst ist zu klären, warum es sinnvoll ist für diesen Flügel GFK zu verwenden. In der Luftfahrt wird immer nach Wegen gesucht das Gewicht zu minimieren, um die Wirtschaftlichkeit von Flugobjekten zu maximieren. Faser-Kunststoffverbunde (FKV) mit ihrer hohen spezifischen Festigkeit stellen hierbei einen idealen Kandidaten dar. Zusätzlich bieten FKV einfache Formgebungsmöglichkeiten für komplexe aerodynamische Profile und auch die Korrosionsbeständigkeit ist höher als bei konventionellen Werkstoffen. Als ein großer Nachteil ist der hohe Preis zu nennen, der in diesem Fall jedoch keine große Rolle spielt, da nur ein Modell entworfen wird und der Flügel nicht für hohe Stückzahlen optimiert wird. Glasfasern sind im Vergleich zu Kohlenstofffasern die günstigere Variante, auf sie wird in Kapitel \ref{Glasfaser} noch mal genauer eingegangen.
\subsection{Herangehensweise}
Text folgt
\newpage
\section{Grundlagen}
\subsection{Glasfaser}
\input{Projektarbeit - Glasfaser}
\subsection{Matrix}
Text folgt noch
\subsection{Netztheorie}
Text folgt noch
\subsection{Klassische Laminatheorie}
Text folgt noch
\subsection{Versagenskriterium nach Puck}
Text folgt noch
\newpage

\section{Auslegung des Holms nach Handbuchmethoden}
\subsection{Modellierung des Holms (T.B.)}
\subsection{Annahmen zur Modellierung (T.B.)}
Das Koordinatensystem des Flügels entspricht dem Flugzeugkoordinatensystem, sodass die \\Flügellängskoordinate durch $y$ definiert ist. Der Koordinatenursprung ist im Lager A positioniert. \\

\noindent Der Holm inklusive des Holmstummels wird für die Belastung durch eine Prüfkraft $F_{pruef}$ in negative z-Richtung als Biegebalken ausgelegt. Dafür ist er an zwei Stellen gelagert, dem Lager $A$ und Lager $B$. Die Lager entsprechen den Verstiftungen (siehe Bauteil U-Profil). Um eine Überbestimmung des Systems zu vermeiden, wird das Lager $B$ als Loslager angenommen. Die Querkraftbolzen werden nicht durch ein Lager, sondern durch eine zusätzlich angreifende Kraft $F_{Q}$ simuliert, da die biegeweiche Wurzelrippe eine nicht definierbare Absenkung erlaubt.\\

\noindent Als Randbedingungen der Modellierung sind die Halbspannweite $s$ und die Absenkung $w$ gegeben. Für die Absenkung $w$ soll eine Sicherheit $j=1,1$ gesetzt werden. Zwischen Lager $A$ und $B$ wird die Länge $l_{1}$ angenommen, zwischen Lager $B$ und der Wurzelrippe $C$ die Länge $l_{2}$. Die verbleibende Länge bis zur Flügelspitze, an der die Prüfkraft $F_{pruef}$ wirkt, wird $l_{3}$ bezeichnet. Die Halbspannweite $s$ wird beginnend in der Mitte der Verstiftungen bis zur Flügelspitze gemessen. Ausgehend von dem Holmstummelende bis zum Lager $A$ wird $l_{0}$ als Länge definiert. Diese Länge ist jedoch unerheblich für die Modellierung, sondern wird erst für die Massenbestimmung benötigt.\\

\noindent Anhand der Randbedingungen und der Einspannvorrichtung für den Versuchsaufbau ergeben sich folgende Längen (ebenfalls in Abb. ~\ref{fig:Holmmodellierung}~ dargestellt): 
\begin{equation}
	s = 0,848 m
\end{equation}
\begin{equation}
	l_{0} = 0,03 m
\end{equation}
\begin{equation}
	l_{1} = 0,076 m
\end{equation}
\begin{equation}
	l_{2} = 0,037 m 
\end{equation}
\begin{equation}
	l_{3} = s - \frac{l_{1}}{2} - l_{2} = 0,773 m
\end{equation}
\begin{equation}
	w_{j=1,1} = \frac{1}{j} * w = \frac{1}{1,1} * 0,022 m = 0,02 m
\end{equation}
\begin{figure}
	\includegraphics[width=1.0\textwidth]{Bilder/Balkenmodell.jpg}
	\caption{Modellierung des Holms}
	\label{fig:Holmmodellierung}
\end{figure}

\subsection{Analytische Lösung der Modellierung (T.B.)}
Um die Differentialgleichungen der Balkenbiegung lösen zu können, wird das System vorerst in drei Teilbereiche $I$, $II$ und $III$ aufgeteilt, die sich von Lager $A$ zu $B$, von Lager $B$ zur Wurzelrippe $C$ und von dort aus bis zur Flügelspitze erstrecken. \\

\noindent Dadurch ergeben sich folgende zwölf Differentialgleichungen:
\begin{equation}
	EI_{x}\cdot w_{I}^{''''}(y) = q_{I}(y)
\end{equation}
\begin{equation}
	EI_{x}\cdot w_{I}^{'''}(y) = q_{I}(y)\cdot y + R_{1} = -Q_{I}(y)
\end{equation}
\begin{equation}
	EI_{x}\cdot w_{I}^{''}(y) = \frac{q_{I}(y)}{2}\cdot y^{2} + R_{1}\cdot y + R_{2} = -M_{I}(y)
\end{equation}
\begin{equation}
	EI_{x}\cdot w_{I}^{'}(y) = \frac{q_{I}(y)}{6}\cdot y^{3} + \frac{R_{1}}{2}\cdot y^{2} + R_{2}\cdot y + R_{3} 
\end{equation}
\begin{equation}
	EI_{x}\cdot w_{I}(y) = \frac{q_{I}(y)}{24}\cdot y^{4} + \frac{R_{1}}{6}\cdot y^{3} + \frac{R_{3}}{2}\cdot y^{2} + R_{3}\cdot y + R_{4}
\end{equation}\\
\begin{equation}
	EI_{x}\cdot w_{II}^{''''}(y) = q_{II}(y)
\end{equation}
\begin{equation}
	EI_{x}\cdot w_{II}^{'''}(y) = q_{II}(y)\cdot y + R_{5} = -Q_{II}(y)
\end{equation}
\begin{equation}
	EI_{x}\cdot w_{II}^{''}(y) = \frac{q_{II}(y)}{2}\cdot y^{2} + R_{5}\cdot y + R_{6} = -M_{II}(y)
\end{equation}
\begin{equation}
	EI_{x}\cdot w_{II}^{'}(y) = \frac{q_{II}(y)}{6}\cdot y^{3} + \frac{R_{5}}{2}\cdot y^{2} + R_{6}\cdot y + R_{7} 
\end{equation}
\begin{equation}
	EI_{x}\cdot w_{II}(y) = \frac{q_{II}(y)}{24}\cdot y^{4} + \frac{R_{5}}{6}\cdot y^{3} + \frac{R_{6}}{2}\cdot y^{2} + R_{7}\cdot y + R_{8}
\end{equation}\\
\begin{equation}
EI_{x}\cdot w_{III}^{''''}(y) = q_{III}(y)
\end{equation}
\begin{equation}
EI_{x}\cdot w_{III}^{'''}(y) = q_{III}(y)\cdot y + R_{9} = -Q_{I}(y)
\end{equation}
\begin{equation}
EI_{x}\cdot w_{III}^{''}(y) = \frac{q_{III}(y)}{2}\cdot y^{2} + R_{9}\cdot y + R_{10} = -M_{I}(y)
\end{equation}
\begin{equation}
EI_{x}\cdot w_{IIII}^{'}(y) = \frac{q_{III}(y)}{6}\cdot y^{3} + \frac{R_{9}}{2}\cdot y^{2} + R_{10}\cdot y + R_{11} 
\end{equation}
\begin{equation}
EI_{x}\cdot w_{III}(y) = \frac{q_{III}(y)}{24}\cdot y^{4} + \frac{R_{9}}{6}\cdot y^{3} + \frac{R_{10}}{2}\cdot y^{2} + R_{11}\cdot y + R_{12}
\end{equation}\\

\noindent Die Randbedingungen der Modellierung ergeben sich folgend: 
\begin{equation}
	w_{I}(y=0)=0
\end{equation}
\begin{equation}
	M_{I}(y=0)=0
\end{equation}\\
\begin{equation}
	w_{I}(y=l_{1}) = 0
\end{equation}
\begin{equation}
	w_{II}(y=l_{1}) = 0
\end{equation}
\begin{equation}
	w_{I}^{'}(y=l_{1}) = w_{II}^{'}(y=l_{1})
\end{equation}
\begin{equation}
	M_{I}(y=l_{1}) = M_{II}(y=l_{1})
\end{equation}\\
\begin{equation}
	w_{II}(y=l_{1}+l_{2}) = w_{III}(y=l_{1}+l_{2})
\end{equation}
\begin{equation}
	w_{II}^{'}(y=l_{1}+l_{2}) = w_{III}^{'}(y=l_{1}+l_{2})
\end{equation}
\begin{equation}
	M_{II}(y=l_{1}+l_{2}) = M_{III}(y=l_{1}+l_{2})
\end{equation}
\begin{equation}
	Q_{II}(y=l_{1}+l_{2}) = Q_{III}(y=l_{1}+l_{2})+F_{Q}
\end{equation}\\
\begin{equation}
	M_{III}(y=l_{1}+l_{2}+l_{3})=0
\end{equation}
\begin{equation}
	Q_{III}(y=l_{1}+l_{2}+l_{3})=F_{pruef}
\end{equation}
Zusätzlich wird angenommen, dass $q_{I}(y)=q_{II}(y)=q_{III}(y)=0$ gilt, da keine Streckenlast angreift.\\

\noindent Als Lösung dieser Differentialgleichungen lässt sich die Querkraft $Q(y)$, das Moment $M(y)$ und die Biegelinie $w(y)$ ermitteln:\\

\begin{equation}
	Q(y,F_{pruef},F_{Q},EI_{x})=\left\{\begin{array}{ll}
		F_{pruef}\cdot \frac{l_{2}+l_{3}}{l_{1}}-F_{Q}\cdot \frac{l_{2}}{l_{1}}&,y\epsilon (0,l_{1})\\
		F_{pruef}+F_{Q}&,y\epsilon (l_{1}+l_{2})\\
		F_{pruef}&,y\epsilon (l_{1}+l_{2}+l_{3})
	\end{array}\right.
\end{equation}\\
\begin{equation}
	M(y,F_{pruef},F_{Q},EI_{x})=\left\{\begin{array}{ll}
		(-F_{pruef}\cdot \frac{l_{2}+l_{3}}{l_{1}}-F_{Q}\cdot \frac{l_{2}}{l_{1}})\cdot y&,y\epsilon (0,l_{1})\\
		F_{pruef}\cdot (y-l_{1}+l_{2}+l_{3})+F_{Q}\cdot (y-l_{1}+l_{2})&,y\epsilon (l_{1}+l_{2})\\
		F_{pruef}\cdot (y-l_{1}+l_{2}+l_{3})&,y\epsilon (l_{1}+l_{2}+l_{3})
	\end{array}\right.
\end{equation}\\
\begin{equation}
	w(y,F_{pruef},F_{Q},EI_{x})=\left\{\begin{array}{ll}
		\frac{1}{EI_{x}}\cdot\frac{1}{6}\cdot\biggl((F_{pruef}\cdot \frac{l_{2}+l_{3}}{l_{1}}-F_{Q}\cdot \frac{l_{2}}{l_{1}})\cdot y^{3}-\Bigl((l_{2}+l_{3})\cdot l_{1}\cdot F_{pruef}-l_{1}\cdot l_{2}\cdot F_{Q}\Bigl)\cdot y \biggr)&\\,y\epsilon (0,l_{1})\\
		\frac{1}{EI_{x}}\cdot\biggl(\frac{(-F_{pruef}-F_{Q})}{6}\cdot y^{3} + \frac{F_{pruef}\cdot (l_{1}+l_{2}+l_{3})+F_{Q}\cdot (l_{1}+L_{2})}{2}\cdot y^{2}\\ + \Bigl(F_{pruef}\cdot(-\frac{1}{2}\cdot l_{1}^{2}-\frac{2}{3}\cdot l_{1}\cdot l_{2}-\frac{2}{3}\cdot l_{1}\cdot l_{3})+ F_{Q}\cdot(-\frac{1}{2}l_{1}^{2}-\frac{2}{3}\cdot l_{1}\cdot l_{2})\Bigr)\cdot y \\+ F_{pruef}\cdot\frac{1}{6}\cdot(l_{1}^{3}+l_{1}^{2}\cdot l_{2}+l_{1}^{2}\cdot l_{3}) + F_{Q}\cdot\frac{1}{6}\cdot(l_{1}^{3}+l_{1}^{2}\cdot l_{2})\biggr)&\\,y\epsilon (l_{1}+l_{2})\\
		\frac{1}{EI_{x}}\cdot\biggl(-\frac{F_{pruef}}{6}\cdot y^{3} + \frac{ F_{pruef}\cdot (l_{1}+l_{2}+l_{3})}{2}\cdot y^{2} + \Bigl(F_{pruef} \\
		\cdot(-\frac{1}{2}\cdot l_{1}^{2} -\frac{2}{3}\cdot l_{1}\cdot l_{2}-\frac{2}{3}\cdot l_{1}\cdot l_{3}) + F_{Q}\cdot(\frac{1}{2}\cdot l_{2}^{2}+\frac{1}{2}\cdot l_{1}\cdot l_{2})\Bigr)\cdot y \\ +F_{pruef}\cdot\frac{1}{6}\cdot(l_{1}^{3}+l_{1}^{2}\cdot l_{2}+l_{1}^{2}\cdot l_{3}) + F_{Q}\cdot(-\frac{1}{6}\cdot l_{2}^{3}-\frac{1}{3}\cdot l_{1}^{2}\cdot l_{2}-\frac{1}{2}\cdot l_{2}^{2}\cdot l_{1})\biggr)&\\,y\epsilon (l_{1}+l_{2}+l_{3})
	\end{array}\right.
\end{equation}\\
Die Herleitung der Lösung wird dem Anhang beigefügt.\\

\noindent Um nun für die Biegesteifigkeit $EI_{x}$ ein Ergebnis zu erhalten, wird die Gleichung $w(y,F_{pruef},F_{Q}, EI_{x})$ nach $EI_{x}(y,F_{pruef},F_{Q},w)$ umgestellt. Die eingesetzten Werte ergeben sich aus der Auslegung auf Steifigkeit. Über die Wurzelrippe werden Kräfte des Holms in die Querkraftbolzen abgesetzt. Aufgrund der biegeweichen Wurzelrippe darf die Absenkung des Holms dort nicht mit null angenommen werden. Vereinfacht wird definiert, dass die eingeleitete Prüfkraft $F_{pruef}$ an den Querkraftbolzen um ihren Betrag abgesetzt wird, wie es tatsächlich an einem Flugzeugrumpf geschehen würde. 

\begin{equation}
	\begin{array}{l}
		EI_{x}(0.961m, 100N, -100N, 0.022m)= \\
		\frac{1}{w}\cdot\biggl(-\frac{F_{pruef}}{6}\cdot y^{3} + \frac{ F_{pruef}\cdot (l_{1}+l_{2}+l_{3})}{2}\cdot y^{2} + \Bigl(F_{pruef}\cdot(-\frac{1}{2}\cdot l_{1}^{2} -\frac{2}{3}\cdot l_{1}\cdot l_{2}-\frac{2}{3}\cdot l_{1}\cdot l_{3}) \\ +F_{Q}\cdot(\frac{1}{2}\cdot l_{2}^{2}+\frac{1}{2}\cdot l_{1}\cdot l_{2})\Bigr)\cdot y + F_{pruef}\cdot\frac{1}{6}\cdot(l_{1}^{3}+l_{1}^{2}\cdot l_{2}+l_{1}^{2}\cdot l_{3}) + F_{Q}\cdot(-\frac{1}{6}\cdot l_{2}^{3}-\frac{1}{3}\cdot l_{1}^{2}\cdot l_{2}-\frac{1}{2}\cdot l_{2}^{2}\cdot l_{1})\biggr)\\
		=962,552Nm^{2}
	\end{array}
\end{equation}

\subsection{Analyse der Modellierung (T.B.)}
In Abb. ~\ref{fig:Steifigkeitsauslegung} werden der Querkraftverläufe $Q(y,F_{pruef},F_{Q},EI_{x})$ als innere Schnittkraft, der Momentenverlauf $M(y,F_{pruef},F_{Q},EI_{x})$ als inneres Schnittmoment und die Biegelinie $w(y,F_{pruef},F_{Q},EI_{x})$ für den Nachweis der Steifigkeit graphisch dargestellt, über die gesamte Holmlänge und in einem vergrößerten Ausschnitt im Bereich der Lager. \\

\begin{figure}
	\includegraphics[width=1.0\textwidth]{Bilder/Grafiken Steifigkeit.png}
	\caption{Steifigkeitsauslegung}
	\label{fig:Steifigkeitsauslegung}
\end{figure}

\noindent Jedoch werden nicht bei dem Nachweis der Steifigkeit, sondern bei dem Nachweis der Festigkeit das maximale Schnittmoment und die maximale Schnittkraft erreicht. Bei diesem Nachweis beträgt die Prüfkraft $F_{pruef} = 500N$. Diese Kraft wird bei der Berechnung von $EI_{x}$ nicht beachtet, da bei dem Nachweis der Festigkeit die Absenkung $w$ kein Rolle spielt. In Abb. ~\ref{fig:Festigkeitsauslegung} werden die genannten Verläufe nun für den Festigkeitsnachweis dargestellt.
\begin{figure}
	\includegraphics[width=1.0\textwidth]{Bilder/Grafiken Festigkeit.png}
	\caption{Festigkeitsauslegung}
	\label{fig:Festigkeitsauslegung}
\end{figure}


\newpage
\subsection{Auslegung des Holms nach VDI 2013 (H.K.)}\label{VDI}
\label{VDI2013}
\subsection{Allgemeine Informationen zu der Richtlinie}
Auf Basis der in der Balkenberechnung bestimmten Parameter Biegesteifigkeit, maximales Biegemoment und der maximalen Querkraft, sollen die Gurte und der Steg dimensioniert werden. Die Vorauslegung erfolgt dabei anhand der VDI-Richtlinie 2013, diese enthält in einem Unterkapitel Informationen speziell zur Auslegung eines I-Trägers. Zur Festigkeitsauslegung werden Gurte und Steg in der Richtlinie getrennt voneinander betrachtet, unter der Annahme, dass der Gurt unter Vernachlässigung der Schubflussaufnahme das gesamte Biegemoment aufnehmen soll und der Steg neben einem über die Höhe konstanten Schubfluss auch durch die aufgeprägte Deformation der Gurte beansprucht wird. Das orthotrope Werkstoffverhalten des Laminats, sowie das Versagensverhalten bei verschiedenen Beanspruchungen, werden allein durch die "charakteristischen K-Werte", die für verschiedene Materialgruppen gültig sind, berücksichtigt. Die Berechnungen der Richtlinie erlauben auf diese Weise keine Aussagen über resultierende Versagensformen des ausgelegten Bauteils. Die vorangestellte Auslegung der Gurte ist in dieser Form nicht Teil der Richtlinie, erfolgt jedoch unter den gleichen Annahmen.     
Zusätzlich sei angemerkt, dass kapitelübergreifend die gesamte Auslegung nur an ausgewählten und speziell gekennzeichneten Stellen Sicherheitsfaktoren ungleich eins berücksichtigt. Grund dafür ist die Annahme, dass in den bereitgestellten Materialkennwerten ausreichende Sicherheiten verrechnet worden sind. Die grobe Vorauslegung hat den Anspruch die Grundlage für die aufbauende Berechnung mithilfe eines Laminatrechners zu legen.

\subsubsection{Dimensionierung der Gurte mit rechteckigem Querschnitt}
\label{GurtDim} 
Im ersten Auslegungsschritt der schubstarren Gurte wird die Einhaltung der Anforderungen an die Steifigkeit betrachtet.   
Die in der Balkenberechnung ermittelte Biegesteifigkeit $ EI_{x} = 962,552 Nm^{2} $, die erforderlich ist, damit bei einer Kraft $ F_{pruef}=100N $ die Flügelspitze eine Absenkung von $ w_{j=1,1}=20mm $ erfährt, muss, wegen oben genannter Annahmen, allein durch die Gurte aufgebracht werden. Im Sinne der kraftflussgerechten Gestaltung sollen die Glasfasern unidirektional in Längsrichtung des Gurtes angeordnet werden. Die Bezeichnungen der Längenangaben des Holms orientieren sich an Abbildung~\ref{fig: Rechteckholm}~.\\

\begin{figure}[h]
	\includegraphics[width=1.0\textwidth]{Bilder/RechteckHolm.jpg}
	\caption{Bezeichnungen des I-Holms}
	\label{fig: Rechteckholm}
\end{figure}

\noindent Die Gurtquerschnitte werden zur Bestimmung der notwendigen Lagenanzahl als rechteckig angenommen, erst in einem späteren Schritt soll die Form der Kontur dem vorgegebenen Hautprofil angepasst werden. Die Maße sind über die gesamte Länge des Holms als konstant anzusehen.\\
Zur Bestimmung des Flächenträgheitsmomentes $ I_{x} $ wird der E-Modul in Längsrichtung der Fasern gemäß der Mischungsregel nach \cite{item3} berechnet.\\
\begin{equation}
 E_{11}=  \varphi\cdot E_{f,11}+\left( 1-\varphi \right) \cdot E_{M}
\end{equation}
Mit den gegebenen Materialkennwerten $ E_{f,11}=74000MPa $, $ E_{m}=3300MPa $ und $ \varphi=0,4 $ bestimmt sich $ E_{11} = 31580 MPa $. Damit ergibt sich ein benötigtes Flächenträgheitsmoment von 
\begin{equation}
	I_{x,min} = \frac{962,552Nm^{2}}{31580\cdot 10^{6}Pa} =3,0479 \cdot 10^{-8} m^{4}
\end{equation}

\noindent Das Flächenträgheitsmoment der Gurte bestimmt sich aus den Flächenträgheitsmomenten der beiden Rechteckquerschnitte und ihren zugehörigen Steiner-Anteilen, die aus der Verschiebung der Gurte um jeweils $ \frac{h_{m}}{2} $ in z-Richtung resultieren. Da die Gurtdicke noch nicht bekannt ist, wird auf die Annahme der Gurte als "punktförmige Flächen" \cite{item15} verzichtet und der Eigenanteil mitbetrachtet. 
\begin{equation}
	\label{Ix}
	I_{x}=2\cdot\left(\frac{b\cdot h^{3}}{12}+b\cdot h\cdot\left(\frac{h_{m}}{2}\right)^{2}\right)
\end{equation}

\noindent Es wird nach einer Kombination aus Gurtbreite $ b $ und Gurthöhe $ h $ gesucht, die die Anforderungen an das Flächenträgheitsmoment erfüllt, aber dennoch zu einer möglichst geringen Gurtquerschnittsfläche und damit zu einer möglichst geringen Masse der Gurte führt. Um die Steiner-Anteile der Gurte zu maximieren, sollen die Gurte in einem möglichst großen Abstand zur neutralen Faser angeordnet werden. Gemäß der gegebenen technischen Zeichnung der Profilkontur, lässt sich das Profil von einem Rechteck der Höhe $ 37,5mm $ umrahmen. Dies entspricht jedoch nicht der Profildicke, da die Punkte mit dem größten Abstand zur Profilsehne auf der Ober- und Unterseite bei verschiedenen Flügeltiefen vorliegen. Zusätzlich muss oberhalb und unterhalb der Gurte ein Freiraum für die umliegenden Haut berücksichtigt werden. Deshalb wird die gesamte Gurthöhe auf $ h_{a}=36mm $ abgeschätzt. Die dadurch begrenzte Anzahl der Lagen in der Haut wird im Kapitel \ref{CAD} weiter erläutert.\\ 

\noindent Tabelle ~\ref{bh} enthält Werte der Gurtquerschnittsfläche bei verschiedenen Kombinationen von $ b $ und $ h $, die zum erforderlichen gesamten Flächenträgheitsmoment von $ I_{x,min} = 3,04722 \cdot 10^{-8} m^{4} $ führen.\\
\begin{table}[h]
	\caption{Verschiedene Kombinationsmöglichkeiten von $ b $ und $ h $}
	\label{bh}
	\begin{center}
		\begin{tabular}{l|c|r}
			$h$&$b$&$2\cdot b\cdot h$\\
			\hline
			$1mm$&$38,3mm$&$76,6mm^{2}$\\
			$1,25mm$&$31,1mm$&$77,7mm^{2}$\\
			$1,5mm$&$26,3mm$&$78,8mm^{2}$\\
			$2,25mm$&$18,3mm$&$82,3mm^{2}$\\
		\end{tabular}
	\end{center}
\end{table}

\noindent Den Daten ist zu entnehmen, dass breite Gurte geringer Dicke bei gleichem Flächenträgheitsmoment geringere Querschnittsflächen aufweisen. Aus diesem Grund sollen die Gurte möglichst breit gewählt werden. Die Breite der Gurte ist durch die vorgegebene Konstruktion der Platte zur Aufnahme der Tragfläche am Teststand begrenzt. Die vorgesehene Aussparung weist eine Breite von $ 30mm $ auf. Für die weitere Berechnung soll $ b=28mm $ gelten. Diese Annahme wird dadurch begründet, dass die Fertigung des Holms im Bereich des Modellbaus von Hand erfolgen würde, womit nur grobe Toleranzen einhaltbar sind.\\

\noindent Mithilfe eines Solvers bestimmt sich aus dem Flächenträgheitsmoment und der Gurtbreite die Gurthöhe $ h=1,866mm $.\\
\noindent Im nächsten Schritt wird die zu stapelnde Lagenanzahl ermittelt. Als vorwiegend unidirektionales Material steht das Glasgewebe Interglas 92145 mit einem Flächengewicht von $ 220\frac{g}{m^{2}} $ zur Verfügung. Nach \cite{item3} berechnet sich die Lagenanzahl $ n $ für eine Dicke des Verbundes $ t_{soll} $ zu:\\

\begin{equation}
	\label{gurtlagen}
	n=t_{soll}\cdot \frac{\varphi\cdot\rho_{f}}{\left(\frac{m_{f}}{L\cdot b}\right)}
\end{equation}

\noindent Mit $ \left(\frac{m_{f}}{L\cdot b}\right) = 220\frac{g}{m^{2}} $, $ t_{soll}=h $ und $ \rho_{f}=2550\frac{kg}{m^{3}} $ ergibt sich $ n=8,653 $. Es sind also 9 Lagen des Gewebes 92145 für jeden Gurt vorzusehen.Die sich aus 9 Lagen ergebende Gurthöhe kann durch Umstellen von Gleichung ~\ref{gurtlagen} zu $ \tilde{h}=1,941mm $ bestimmt werden. Für den zunächst angenommenen Fall von Gurten mit rechteckigen Querschnitten ist die Auslegung zur Einhaltung der Anforderungen an die Steifigkeit damit abgeschlossen.\\


\subsubsection{Nachrechnung der angepassten Gurte}
 Die Modellierung der Haut und der Holmgurte in einem CAD-Programm zeigt, dass die Gurte mit den berechneten Bemaßungen nicht innerhalb des Profils mit der als $ 0,75mm $ dick angenommenen Haut liegen. Die Anpassung der Konstruktion der Gurte erfolgt so, dass die Gurtoberseite an der Innenseite der Haut anliegt (vgl. Abschnitt \ref{GurtKonstrukt}).Die Gesamtbreite von $ 28mm $, sowie die Gurtdicke $ h $ bleiben dabei erhalten. Die Gesamthöhe $ h_{a} $ muss auf $ \tilde{h_{a}}=35,8mm $ leicht verringert werden. Abbildung ~\ref{fig: KrummerGurt} veranschaulicht die gekrümmte Form des oberen Holmgurtes.
 \begin{figure}[h]
 	\includegraphics[width=1.0\textwidth]{Bilder/KrummerGurt.jpg}
 	\caption{Angepasste gekrümmte Gurtkontur}
 	\label{fig: KrummerGurt}
 \end{figure}

\noindent Die angepasste Krümmung der Gurte führt zu einem veränderten Flächenträgheitsmoment $ \tilde{I_{x}} $ des Balkens, dass mithilfe des CAD-Programms exakt zu $ \tilde{I_{x}}=3,075406\cdot 10^{-8}m^{4} $ bestimmt werden kann. Da 
\begin{equation}
	\label{IVergleich}
	\tilde{I_{x}}=3,075406\cdot 10^{-8}m^{4} > I_{x,min}=3,04722\cdot 10^{-8}m^{4}
\end{equation}
gilt, genügen auch die veränderten Gurte der Steifigkeitsanforderung.\\
 
\noindent Abschließend wird gezeigt, dass die Festigkeit der Gurte einer Belastung der Flügelspitze durch $ F_{pruef}=500N $ standhält. Die aus der Biegung resultierenden und betragsmäßig gleichen Zug- und Druckspannungen werden dazu mit den vorhandenen UD-Festigkeitskennwerten des Handlaminats verglichen. Die Resultate der Balkenberechnungen zeigen, dass das maximale Biegemoment im Holm an Punkt C auftritt und $ M_{b}=500N\cdot 0,773m=386,5Nm $ beträgt.In den Randfasern der Gurte resultieren Spannungen, die sich gemäß \cite{item15} nach\\
\begin{equation}
	\sigma_{b}=\frac{M_{b}\cdot \tilde{h_{a}}}{\tilde{I_{x}}\cdot 2}
\end{equation} 
zu $ \sigma_{b}=224,96MPa $ berechnen. Da 
\begin{equation}
	\sigma_{b}< R^{(+)}_{||}=597,9 MPa < |R^{(-)}_{||}| =650,0 MPa
\end{equation} 
gilt, ist der Festigkeitsnachweis erbracht. Es kann davon ausgegangen werden, dass die Gurte bei einer Prüfkraft von $ F_{pruef}=500N $ nicht versagen. \\

\subsubsection{Bestimmung der Lagenanzahl des Steges}
Die Auslegung des Steges erfolgt auch anhand der VDI 2013. Dabei muss beachtet werden, dass der Steg sowohl durch Schubkräfte als auch durch Normalkräfte senkrecht und parallel zu den Gurten Belastungen erfährt. Die Dehnungen der Innenseiten der Gurte werden dem Steg aufgeprägt, da beide Bauteile stoffschlüssig miteinander verbunden sind. Anders als in der VDI 2013 wird jedoch nicht die Bruchdehnung der Gurte betrachtet, sondern die Dehnungen der Innenseiten bei einer Prüfkraft von 500N. So soll die Dimensionierung des Steges auf die Anforderungen an die Festigkeit angepasst werden, um Leichtbaupotentiale bestmöglich auszuschöpfen.\\

\noindent Die größte Längsdehnung der Gurte tritt an der Stelle C auf, da dort das größte Biegemoment wirkt. Sie lässt sich für die Innenseite der Gurte durch
\begin{equation}
	\epsilon_{Gurt}=\frac{\sigma_{innen}}{E_{11}}=\frac{\frac{F_{pruef}\cdot l_{3}\cdot h_{i}}{\tilde{I}_{x}\cdot 2}}{E_{11}}
\end{equation}  
 zu $ \epsilon_{Gurt}=6,351\cdot 10^{-3} $ berechnen. Auf der Zugseite ist die Dehnung positiv, auf der Druckseite negativ. Die dem Steg aufgeprägte Dehnung führt in Längsrichtung des Steges zu einem Normalkraftfluss, der sich nach VDI 2013 mit
 \begin{equation}
 	p_{\epsilon}=n\cdot \bar{q}\cdot K_{E\#}\cdot \epsilon_{Gurt}
 \end{equation} 
ermitteln lässt. $ K_{E\#} $ ist dabei ein verallgemeinerter Dimensionierungskennwert, der Tafel 3 der VDI 2013 zu $ K_{E\#}=1150\cdot 10^{3}m $ entnommen wird. Es ist zu beachten, dass in der VDI mit veralteten Einheiten, wie dem Kilopond, gerechnet wird. Flächengewichte $ \bar{q} $ sind durch Multiplikation der auf die Fläche bezogene Masse $ \frac{m_{f}}{L\cdot b} $ mit der Norm des Erdbeschleunigungsvektors $ \vec{g} $ zu ermitteln. Zur Berechnung von Hand wurden alle Kennwerte auf die SI-Einheiten zurückgeführt. $ n $ kennzeichnet auch hier die Lagenanzahl.\\

\noindent Zur kraftflussgerechten Gestaltung des Steges werden die Gewebelagen unter einem Winkel von $ 45^{\circ} $ zu den Holmgurten angeordnet. Deshalb muss die Belastung parallel zu den Faserrichtungen mithilfe einer Transformationsformel nach VDI 2013 berechnet werden.
\begin{equation}
	p_{\epsilon||}=p_{\epsilon}\cdot cos^{2}\left(45^{\circ} \right)=p_{\epsilon}\cdot 0,5 
\end{equation}

\noindent Die Normalkräfte in Längsrichtung an den Gurten bilden im Allgemeinen einen Winkel $ \neq180^{\circ} $ zueinander, da der Holm eine Absenkung erfährt. Daraus resultiert eine Normalkraft auf den Steg, die senkrecht zu den Gurten steht. Der resultierende Kraftfluss berechnet sich zu:
\begin{equation}
	p_{A}=\frac{2\cdot F_{pruef}\cdot l_{3}\cdot\epsilon_{Gurt}}{h_{m}^{2}}
\end{equation}
 Mit der oben genannten Transformationsformel ergibt sich die Belastung in Faserrichtung.
 \begin{equation}
 	p_{A||}=p_{A}\cdot cos^{2}\left(45^{\circ} \right)
 \end{equation} 

\noindent Darüber hinaus erfährt der Steg einen Schubkraftfluss durch die Querkraft. Wegen der vernachlässigbaren Längskraftaufnahme des Steges im Vergleich zu den Gurten, kann der Schubfluss über die Höhe des Steges als konstant angenommen werden. Die VDI-Richtlinie orientiert sich hier an den Berechnungsmethoden der Schubfeldtheorie, wie sie im entsprechenden Kapitel in \cite{item15} ausgeführt wird. Es muss berücksichtigt werden, dass die Modellierung des Holmes als Balken, der an zwei Punkten gelagert ist und durch die Querkraftbolzen eine weitere Kraft erfährt, zu einem anderen Querkraftverlauf führt als dem konstanten, der in der Richtlinie für den Kragbalken angenommen wurde. Den Berechnungen des Holms als Biegebalken kann für eine Kraft $ F_{prue}=500N $ eine maximale Querkraft von $ 5085,5N $ im Bereich $ 1 $ und eine betragsmäßig maximale Querkraft von $ 500N $ im Bereich $ 3 $ entnommen werden. Mit dem Ziel, im langen Bereich $ 3 $ Gewicht einzusparen, ist es vorteilhaft diesen Bereich geringer Querkraft getrennt von dem höher beanspruchten Bereich $ 1 $ auszulegen. Der resultierende Schubfluss berechnet sich mithilfe der folgenden Formel:\\
\begin{equation}
	p_{s||}=p_{s}=\frac{Q}{h_{i}}
\end{equation}
Der Kraftfluss, der durch den Steg aufgenommen werden muss, ergibt sich aus der Überlagerung der drei Kraftflüsse $ p_{s||}, p_{A||}, p_{\epsilon||} $. Die Tragfähigkeit einer Schicht des Verbundes unter Druckbeanspruchung wird durch $ K_{\sigma d} $ charakterisiert und kann ebenfalls Tafel 3 der VDI entnommen werden. An dieser Stelle geht die Richtlinie davon aus, dass die Druckfestigkeit des Laminats im Allgemeinen geringer ist als die Zugfestigkeit. Im vorliegenden Fall zeigen die gegebenen Materialkennwerte des Laminats, dass die Druckfestigkeiten $ R_{||}^{-} $ und $ R_{\perp}^{-} $ deutlich größer als die Zugfestigkeiten $ R_{||}^{+} $, bzw. $ R_{\perp}^{+} $ sind. Dies begründet die Vermutung, dass die vorgestellten Dimensionierungswerte nur für die Vorauslegung auf den vorliegenden Fall übertragbar sind. Da ein Teil der Schubbeanspruchung durch die Matrix geleitet wird, besteht die Gefahr eines Zwischenfaserbruches. VDI 2013 schlägt deshalb die Verwendung von $ K_{\sigma d}=30*10^{3}m $ vor. Zusätzlich muss der Anteil der Glasmengen in Kette und Schuß durch den Faktor $ k_{||} $ berücksichtigt werden. Das zur Verfügung stehende Gewebe Interglas 90070 hat annähernd gleiche Fadenanzahlen in Kette- und Schußrichtung, damit ist $ k_{||}=0,5 $. 
\begin{equation}
	n\cdot \bar{q}\cdot K_{\sigma d}\cdot k_{||}=p_{s||}+p_{A||}+p_{\epsilon||}
\end{equation}
Die Anzahl der notwendigen Gewebelagen n im Steg lässt sich nun durch Umstellen der Gleichungen und Einsetzen der bekannten Werte ermitteln.\\
\begin{equation}
	n=\frac{\frac{2\cdot F_{Pruef}\cdot l_{3}\cdot \epsilon_{Gurt}}{h_{m}^{2}\cdot 2}+\frac{Q}{h_{i}}}{\bar{q}\cdot \left(k_{||}\cdot K_{\sigma d}-K_{E\#}\cdot \epsilon_{Gurt}\cdot 0,5\right)}
\end{equation}
Damit ergibt sich die Lagenanzahl von $ n\left(500N\right)=1,99 $ für den Bereich $ 3 $ und $ n\left(5085,5N\right)=18,13 $ für die Bereiche $ 1 $. Um einen symmetrischen Lagenaufbau im Falle einer Sandwichkonstruktion zu ermöglichen, sind also $ 2 $ Lagen für den Bereich $ 3 $ und $ 20 $ Lagen für die Bereiche $ 1 $ und $ 2 $ vorzusehen.\\

\noindent Es ist zu betonen, dass diese Lagenanzahlen maßgeblich durch die Annahmen der Dimensionierungskennwerte $ K_{E\#} $ und $ K_{\sigma d} $ beeinflusst werden. Aus oben genannten Gründen können sie nicht als für diesen Fall exakt angenommen werden. Im Kapitel \ref{Puck} wird die hier ermittelte Lagenanzahl mit tatsächlichen Laminatkennwerten und einem Laminatrechner überprüft und angepasst.

\newpage
\subsection{Auslegung des Holms nach Klassischer Laminattheorie (T.B.)}\label{Puck}
\subsection{Grundlagen der CLT}
\subsection{Versagneskriterium nach Puck}
\subsection{eLamX (T.B.)}
ELamX ist eine Laminatberechnungsprogramm, das anhand der klassischen Laminattheorie mit unterschiedlichen Versagenskriterien berechnen kann, inwiefern ein gewählter Lagenaufbau den Festigkeitskriterien standhält. Zusätzlich sind weitere Funktionen, wie z.B. Beulberechnungen, Optimierungen etc. nutzbar, jedoch für diese Auslegung irrelevant.

\noindent Vorerst wurden die gegeben Materialeigenschaften der Aufgabenstellung als Fasermaterial, Matrixmaterial und  Materialeigenschaften definiert.\\

\begin{tabular}{ll|ll|ll}
	\multicolumn{2}{c}{Fasermaterial} &\multicolumn{2}{c}{Matrixmaterial}  &\multicolumn{2}{c}{Materialeigenschaften} \\
	\hline
	$\rho_{f}$ & $2,55 \frac{g}{cm^{3}}$  & $\rho_{m}$ & $1,18 \frac{g}{cm^{3}}$  & $R_{\parallel}^{+}$ & $597,9MPa$ \\
	\hline
	$E_{f,11}$ & $74000MPa$  & $E_{M}$ & $3300MPa$  & $R_{\parallel}^{-}$ & $650,0MPa$\\
	\hline
	$E_{f,22}$ & $74000MPa$  & $G_{M}$ & $1222MPa$  & $R_{\perp}^{+}$ & $37,7MPa$\\
	\hline
	$G_{f,12}$ & $30800MPa$ & $\nu_{M}$ & $0,35$  & $R_{\perp}^{+-}$ & $130,0MPa$\\
	\hline
	$\nu_{f,21}$ & $0,2$  & &   & $R_{\parallel\perp}$ & $37,5MPa$\\
\end{tabular}\\

\noindent Mit einem Faservolumenanteil $\varphi=0,4$ ergeben sich folgende weitere Materialeigenschaften:\\

\begin{tabular}{ll}
	$\rho$ & $1,728 \frac{g}{cm^{3}}$ \\
	\hline
	$E_{\parallel}$ & $31580 MPa$\\
	\hline
	$E_{\perp}$ & $5341,2MPa$\\
	\hline
	$\nu_{\parallel\perp}$ & $0,29$\\
	\hline
	$G_{\parallel\perp}$ & $1984,5 MPa$\\
\end{tabular}\\

\noindent Anschließend werden die nach Kapitel ? (VDI 2013) berechneten Laminate bzw. Lagenzusammensetzungen aus aus mehreren Material-Lagen zusammengesetzt. Eine Gewebelage wird dabei durch zwei einzelne Materiallagen mit einem Winkel von $90°$ zueinander simuliert, sodass sich die doppelte Anzahl des Materials gegenüber der Lagenanzahl ergibt.Für die Holmgurte ergibt sich ein Lagenaufbau nach \~{ref:} und für den Steg nach \~{ref:}.\\ 

\noindent Anschließend werden 
\newpage
\subsection{Beulabschätzung des Holms (T.B.)}
\subsection{Beulsicherheit der Gurte (T.B.)}
Nachdem die Holmgurte auf Festigkeit und Steifigkeit ausgelegt worden sind, muss überprüft werden, ob der Effekt des Beulens auftritt.\\

\noindent Dazu werden folgende Annahmen getroffen:

\begin{enumerate}
	\item Die Berechnung erfolgt nach [1]
	\item Es wird angenommen, dass die orthotropen Gewebe hinreichend mit den Gleichungen für isotropes Material berechnet werden können. Diese Annahme wird mit erfolgten Zulassungen für Segelflugzeuge anhand dieser Formeln begründet.
	\item Die Gurte werden als ebene, unendlich lange Streifen betrachtet. Die tatsächliche Krümmung dieser beeinflusst die Beulsicherheit positiv.
	\item Da die Mitte in $x$-Richtung der Holmgurte mit dem Holmsteg verklebt ist, kann diese Klebelinie als freie Lagerung gesehen werden. Somit halbiert sich die angenommene Holmgurtbreite.
	\item Die äußeren Kanten in $x$-Richtung sind frei und nicht gelagert.
	\item Die äußeren Kanten in $y$-Richtung werden an den jeweiligen Rippen gestützt.
	\item Der Druckgurt wird nur durch Druckspannungen beansprucht. Die Schubspannungen werden durch das hohe Verhältnis von Länge zu  Höhe vernachlässigt.
	\item Als größtmögliche Länge bei höchster Biegespannung wird $l_{3}$ bestimmt.
\end{enumerate}
Das Seitenverhältnis beträgt 
\begin{equation}
	\frac{b}{a}=\frac{\frac{28 mm}{2}}{773 mm}=0,018 \approx 0
\end{equation}
\noindent Nach [Hertel, Abbildung 84] ergibt sich:
\begin{equation}
	k_{d}=0,4
\end{equation}
und somit die kritische Spannung:
\begin{equation}
	\sigma_{krit,d}=k_{d}\cdot E_{\parallel}\cdot\biggl(\frac{d}{b}\biggr)^{2}\\
	= 242,82 MPa
\end{equation}
Im Vergleich zu der tatsächlich maximal auftretenden Randfaserspannung der Gurte ergibt sich die Sicherheit gegen Beulen zu 
\begin{equation}
	j_{Gurt}=\frac{\sigma_{krit,d}}{\sigma_{d}}=\frac{242,81 MPa}{224,96 MPa}=1,08
\end{equation}


\subsection{Beulsicherheit des Steges (T.B.)}
Ebenfalls muss der Holmsteg nach der Auslegung hinsichtlich der Sicherheit gegen Beulen überprüft werden. Folgende Annahmen werden dafür getroffen:\\
\begin{enumerate}
	\item Die Berechnung erfolgt, wie bei der Berechnung der Holmgurte, nach [1].
	\item Es wird angenommen, dass die orthotropen Gewebe hinreichend mit den Gleichnugen für isotropes Material berechnet werden können, Diese Entscheidung wir ebenfalls mit erfolgten Zulassungen für Segelflugzeuge begründet.
	\item Der Steg wird als ebener, unendlich langer Streifen betrachtet.
	\item Die Verklebung des Steges wird als gestützte, gelenkige Lagerung an allen vier Kanten angenommen.
	\item Der Steg wird durch Biegung und Schubspannung beansprucht.
\end{enumerate}
Für den Steg müssen die drei Bereiche der Holmauslegung auf die Beulsicherheit geprüft werden. \\
\noindent Im Folgenden wir die Beulsicherheit des Bereichs $I$ berechnet:\\
\noindent Das Seitenverhältnis ergibt sich zu 
\begin{equation}
	\frac{b}{a}=\frac{35,8 mm-2\cdot 1,941 mm}{76 mm}=0,042
\end{equation}
Dadurch lässt sich mit 
\begin{equation}
	\frac{\sigma_{max}}{\sigma_{min}}=-1
\end{equation}
nach Hertel, Abbildung 85 der Beulfaktor ermitteln zu
\begin{equation}
	k_{b} = 21,8
\end{equation}
Da das Dickenverhältnis von Steglagen zu Schaumkern sehr klein ausgelegt werden soll, wird
\begin{equation}
	\kappa = 1
\end{equation}
definiert. Dadurch ergibt sich die kritische Biegespannung zu
\begin{equation}
	\sigma_{krit,B}=\kappa\cdot\ k_{b}\cdot E_{\parallel}\cdot\Bigl(\frac{1,882 mm}{35,8 mm-2\cdot 1,941 mm}\Bigr)^{2}=15596,73MPPa
\end{equation}
Das Verhältnis der Biegespannung zur kritischen Biegespannung ist
\begin{equation}
	j_{1}=\frac{200,56 Mpa}{Ergebnis}
\end{equation}. Für den Schub wird nach Hertel der Beulfaktor zu
\begin{equation}
	k_{s}=5,5
\end{equation}
Damit wir die kritische Schubspannung zu 
\begin{equation}
	\tau_{krit}=\kappa\cdot k\cdot E_{\#}\cdot\Bigl(\frac{1,88 2mm}{35,8 mm - 2\cdot 1,941  mm}\Bigr)^{2}=164,02 MPa
\end{equation}
Die tatsächlich auftretende Schubspannung beträgt 
\begin{equation}
	\tau=\frac{3}{2}\cdot \frac{5085,5 N}{1,882 mm\cdot(35,8 mm-2\cdot 1,941 mm)}=126,99 Mpa
\end{equation}
sodass das Verhältnis der Schubspannung zur kritischen 
\begin{equation}
	j_{2}=\frac{126,99 MPa}{164,02 MPa}
\end{equation}
ergibt. Die Gesamtsicherheit beträgt nach [Quelle?]
\begin{equation}
	j=\sqrt{\frac{1}{j_{1}^{2}+j_{2}^{2}}}=1,148
\end{equation}
Somit kann rückgeschlossen werden, dass dieser Bereich des Holmsteges schon ohne Schaumkern sicher gegen Beulen ist.\\

\noindent Nun wird der Bereich $II$ betrachtet:
Da keine innere Querkraft herrscht, kann die Sicherheit durch Biegung außer Acht gelassen werden. Die Sicherheit gegen Beulen ist demnach nur von dem Schub abhängig.
Das Seitenverhältnis beträgt 
\begin{equation}
	\frac{a}{b}=\frac{35,8mm - 2\cdot 1,941mm}{37mm}=0,863
\end{equation}
Damit ergibt sich der Beulfaktor zu 
\begin{equation}
	k_{s}=6,8
\end{equation}
und weiterhin wird mit 
\begin{equation}
	\kappa = 1 
\end{equation}
gerechnet. Damit lässt sich 
\begin{equation}
	 \tau_{krit} = k_{s}\cdot \kappa\cdot G_{\#}\cdot \Bigl(\frac{1,882mm}{35,8mm-2\cdot 1,941mm}\Bigr)^{2}= 202,79 MPa
\end{equation}
berechnen. Mit dem gleichen maximalen Schub
\begin{equation}
	\tau=126,99 MPa
\end{equation}
 wie in Bereich I kann somit die Sicherheit zu 
 \begin{equation}
 	j=\frac{202,79MPa}{126,99MPa}=1,59
 \end{equation}
bestimmt werden. Auch dieser Stegbereich $II$ ist gegen Beulen sicher.\\

\noindent Abschließend wird der verbliebene Bereich $III$ überprüft.\\
\noindent Erneut wird das Seitenverhältnis ermittelt zu 
\begin{equation}
	\frac{b}{a}=\frac{35,88mm - 2\cdot 1,941mm}{773mm}=0,041\approx 0
\end{equation}
Dadurch ist
\begin{equation}
	k_{d}=21,8
\end{equation}
nach \cite{item1} für 
\begin{equation}
	\frac{\sigma_{max}}{\sigma_{min}}=-1
\end{equation} 
und
\begin{equation}
	k_{s} = 4,8
\end{equation}
Für die Belastung auf Druck durch Biegung wirkt maximal die Schubspannung
\begin{equation}
	\sigma_{b} = \frac{500N\cdot 0,773m}{3,075406\cdot 10^{-8}m^{4}}\cdot\frac{0,0358m - 2\cdot 1,941\cdot 10^{-3}m}{2}=200,56 MPa
\end{equation}
für den Schub wirkt die maximale Schubspannung von
\begin{equation}
	\tau_{max}=\frac{3}{2}\frac{500N}{0,313mm\cdot(35,8mm-2\cdot 1,941mm)}=75,07MPa
\end{equation}
Die Sicherheit gegen Beulen berechnet sich nun zu 
\begin{equation}
	j=\sqrt{\frac{1}{(\frac{\sigma_{v}}{\sigma_{krit}})^{2}+(\frac{\tau}{\tau_{krit}})^{2}}}
\end{equation}
mit 
\begin{equation}
	\sigma_{krit}=k\cdot k\cdot E_{\parallel}\cdot \Bigl(\frac{0,313mm+x}{35,8mm -2\cdot 1,941mm}\Bigr)^{2}
\end{equation}
und
\begin{equation}
	\tau_{krit} = \kappa\cdot k_{s}\cdot G_{\#}\cdot\Bigl(\frac{0,313mm + x}{35,8mm -2\cdot 1,941mm}\Bigr)^{2}
\end{equation}
Dieses mal kann $\kappa=3$ genutzt werden, sofern eine ausreichende Dicke $x$ auftritt, dessen Lösung analytisch ermittelt wird. Dabei wird eine Mindestdicke von 
\begin{equation}
	x=0,569mm
\end{equation}
ermittelt, allerdings soll ein Schaum der Dicke $x=2mm$ verbaut werden, um das exakte Anpassen der Bauteilmaße zu vereinfachen. Damit ergibt sich eine Beulsicherheit im Bereich $III$ von
\begin{equation}
	j=6,88
\end{equation}
Damit die äußeren Steglagen eine konstante Stegdicke bilden, soll in den Bereichen $I$ und $II$ eine Schaumdicke von $x=0,431mm$ verbaut werden. Der geschliffene Schaum erhöht zudem in diesen Bereichen die Beulsicherheit, obwohl kein zusätzlicher Schaum benötigt wäre. Der Übergang der beiden Schaumdicken soll zudem nicht direkt an der Wurzelrippe beginnen, sondern erst $23mm$ zur Flügelspitze versetzt erfolgen, um direkt an dem Kraftangriffspunk $C$ die Sicherheit ebenfalls zu erhöhen.

\newpage
\subsection{Auslegung der Klebeverbindung (T.B.)}
Der Holm, bzw. der Tragflügel generell, wird aus verschiedenen Bauteilen gefertigt, die miteinander verbunden werden müssen. Unterschieden werden kann beim Fügen zwischen kraftschlüssiger Verbindung (z.B. Schrauben), formschlüssiger Verbindung (z.B. Schnapphaken) und stoffschlüssiger Verbindung (z.B. Löten oder Kleben). FVK werden meistens miteinander verklebt, dabei muss kann man auf drei Varianten zurückgreifen:
\begin{itemize}
	\item Nass-Nass-Verklebung: Zwei Laminate werden im nicht ausgehärteten Zustand, also während ihrer Fertigung, gefügt.
	\item Trocken-Nass-Verklebung: Obwohl ein Laminat schon gehärtet ist, kann es dennoch mit einem zweiten, nicht gehärteten Laminat gefügt werden. Bei dem gehärteten Laminat muss eine geforderte Rauheit garantiert werden.
	\item Trocken-Trocken-Verklebung: Um hierbei eine Verbindung zweier gehärteten Laminate zu schaffen, wird ein Fügestoff genutzt. Dieser wird durch das verwendete Harzsystem mit einer Beimischung von Aerosil oder Baumwollflocken zur Andickung gebildet. Beide Bauteile müssen eine Mindestrauheit in der Klebefläche aufweisen können.
\end{itemize}

\noindent In der folgenden Dimensionierungsmethode für Klebeflächen wird lediglich auf Schubspannungen in deren Flächen-Ebene eingegangen. Schälen oder senkrechte Spannungen dazu sind somit ausgeschlossen. Außerdem soll nur mit einfachen Klebeverbindungen von aufeinander liegenden Bauteilen gerechnet werden, da ein Anschrägen bzw. Schäften an Bauteilen solch kleiner Abmessungen teilweise ausgeschlossen werden muss.\\

\noindent Als letzte analytische Auslegung des Holms sollen zwei bedeutende Klebeflächen berechnet werden.
\subsubsection{Klebeverbindung Steg - Gurt}
Die Klebverbindung wird ähnlich der VDI 2013 ausgelegt, sodass nur die Abtriebskraft des Holms und die übertragene Querkraft den Schubfluss für die Belastung definieren:
\begin{equation}
	p=\sqrt{p_{A}^{2}+p_{s}^{2}}
\end{equation}
Die Länge ergibt sich aus 
\begin{equation}
	l=\frac{p}{\tau_{zul}}
\end{equation}
wobei die zulässige Klebeschubspannung
\begin{equation}
	\tau_{zul}=10 Pa
\end{equation}
nach Angaben der Aufgabenstellung der Schubspannung für Mumpe entsprechen soll. Die Verklebung muss wegen kleiner Bauteilmaße \glqq Trocken-Trocken\grqq\: erfolgen.\\

\noindent Bereich $I$ und $II$ werden, ähnlich der Beulberechnung, zusammen ausgelegt mit den kritischsten aller Werte, sodass 
\begin{equation}
	l=\frac{\sqrt{(4282,26\frac{N}{m})^{2}+(159330,16\frac{N}{m})^{2}}}{10^{7}\frac{N}{m^{2}}}=15,9mm
\end{equation}
als Klebebreite benötigt werden. Für den Bereich $III$ ergibt sich
\begin{equation}
	l=\frac{\sqrt{(4282,26\frac{N}{m})^{2}+(15665,14\frac{N}{m})^{2}}}{10^{7}\frac{N}{m^{2}}}=1,62mm
\end{equation}
Beide Klebebreiten passen auf die verbleibende innere Holmgurtflächen und sollen durch Mumpe ohne zusätzliche Gewebelagen realisiert werden. Die Querschnittsfläche der Mumpe sollte somit einem gleichseitigem Dreieck entsprechen.

\subsubsection{Klebeverbindung Holm - Rippen}
Die vergrößerten Klebeflächen der Rippen gegenüber dem Holm sollen durch Holzklötze ermöglicht werden, die auf beide Seiten des Stegs zwischen die Holmgurte geklebt werden. Die Breite dieser Klötze errechnet sich aus 
\begin{equation}
	A=b\cdot h=\frac{F}{\tau_{zul}}
\end{equation}
mit der maximal abgesetzten Kraft von $F=500N$.
Somit haben die Holzklötze eine Breite von 
\begin{equation}
	b=1,56mm
\end{equation}
bei einer Höhe von $35,8mm-2\cdot 1,941mm$.
\newpage
\subsection{Bolzenauslegung (H.G.)}


\subsubsection{Bolzenberechnung (H.G.)} \label{Bolzenberechnung}
Zunächst muss eine Auslegung für die Flächenpressung erfolgen. Dabei ist für die Buche $\sigma_{p,zul}=60\frac{N}{mm^2}$.
Die projizierte Fläche ist $A=a*d$ , d ist hierbei der Durchmesser des Bolzens und a die Länge des Lochs im Steg.\\
\noindent
Die Flächenpressung ist nun: 
\begin{equation}
	p=\frac{F}{A}=\frac{5085,5\mathrm{N}}{a*8mm}
\end{equation}
Mit der zulässigen Flächenpressung für Buchenholz lässt sich die Gleichung nun nach a auflösen (siehe Glg. \ref{Fpressung}).
\begin{equation}
\label{Fpressung}
	a=\frac{F}{\sigma_{p,zul}*d}=\frac{5085,5\mathrm{N}}{60\frac{N}{mm^{2}}*8mm}=10,59mm
\end{equation}
Somit muss die Holzverstärkung des Stegs an der Stelle der Lager mindestens eine Breite von 10,59$\mathrm{mm}$ aufweisen. Im Weiteren wird eine Breite von 11$\mathrm{mm}$ angenommen.\\
\noindent
Nun werden die Bolzen, die den Holm im für den Versuchsaufbau vorgegebenen U-Profil fixieren, ausgelegt. Die Bolzen sind auf Biegung belastet, wodurch Gleichung \ref{Biegespannung1} angenommen wird.
\begin{equation}
\label{Biegespannung1}
	\sigma_{b}=\frac{M_{b}}{W} 
\end{equation}
 
 \begin{equation}
 \label{WKreis}
 	W_{Kreis}=\frac{\pi*d^{3}}{32}
 \end{equation}
  
 \begin{equation}
 \label{Moment}
 	M_{b}=F*l
 \end{equation}
 Mit Gleichung \ref{WKreis} und Gleichung \ref{Moment} ergibt sich die Biegespannung zu Gleichung \ref{Biegespannung}.
 \begin{equation}
 \label{Biegespannung}
 	\sigma_{b}=F*\frac{32*l}{\pi*d^{3}}
 \end{equation}
Die zu ertragende Kraft ist $5085,5 \mathrm{N}$. Diese lässt sich jetzt noch halbieren, da mit der oben genannten Annahme nur die Hälfte des Bolzens betrachtet wird. Damit ergibt sich die gesuchte Kraft (siehe Glg. \ref{KraftB}).
 \begin{equation}
 \label{KraftB}
 	F_{bel}=\frac{F}{2} =2542,75 \mathrm{N}
 \end{equation}
 Nun kann mit $d=8mm$ und $l=8,559mm$ ein passendes Material gesucht werden. Für Stahl gilt $\sigma_{b,F}\approx1,2*R_{e}$, somit wird $R_{e}$ wie in Gleichung \ref{RE} berechnet. 
 \begin{equation}
 \label{RE}
 	F*\frac{32*l}{1,2*\pi*d^{3}}=360,8\mathrm{MPa}\leq
R_{e} \end{equation}  
\noindent
Mit den getroffenen Annahmen ist S620Q mit einer Streckgrenze von $R_{e}=620\frac{N}{mm^{2}}$ ein geeigneter Stahl, mit dem eine ausreichende Sicherheit gegeben ist.\\
\noindent
Für die Querkraftbolzen wird ebenfalls S620Q verwendet. Die aufzunehmenden Kräfte lassen sich aus dem Kräftegleichgewicht ermitteln. $Q_{1}$ ist die Kraft des Bolzens rechts vom Holm und $Q_{2}$ die Kraft des Bolzens links vom Holm. Damit ergeben sich bei einer Prüfkraft von 500N folgende Kräfte:
$$Q_{1}=409,85N$$
$$Q_{2}=90,15N$$
Für den Benötigten Durchmesser lässt sich nun Gleichung \ref{Biegespannung} nach d umstellen.
\begin{equation}
d=\sqrt[3]{\frac{32*F*l}{1,2*\pi*R_{e}}}
\end{equation}
Mit einem Hebelarm l von 10mm ergibt sich:
\begin{equation}
d=3,82mm
\end{equation}
Da die Querkraftbolzen auf keinen Fall versagen dürfen und in der Massenbilanz nicht relevant sind, werden sie mit einem Durchmesser von 8mm bemessen.\cite{item6}\\
 
 
 
 
  
 

\newpage
\section{Auslegung der Flügelschale nach Handbuchmethoden}
\subsection{Schubfluss}

\subsubsection{Theorie (O.S.)}
Im Leichtbau ist es wichtig geeignete Konstruktionen zu entwerfen, um Eigenschaften der Werkstoffe ideal nutzen zu können. Das Gesamtbauteil ist dann meist so kompliziert, dass sich die für das Bauteil zu lösenden Differenzialgleichungen, wie zum Beispiel die der Elastostatik, keine geschlossenen Lösungen finden lassen. Das Ziel ist es nun durch Annahmen und Vereinfachungen ein sinnvolles und lösbares Ingenieursmodell zu finden. Alternativ könnten auch Lösungen mittels numerischer Methoden ermittelt werde, hierzu mehr in Kapitel \ref{FEM}.

\paragraph{Zylindrische, dünnwandige Profile}~\\
Häufig werden zylindrische, dünnwandige Profil genutzt, da sie bei vergleichsweiser niedriger Masse noch immer gut Werte bei zum Beispiel der Biegesteifigkeit liefern. Sie kennzeichnen sich dadurch aus, dass deutlich höhere Ausmaße in x-Richtung haben, als in jede andere (zylindrisch). Eine Laufvariable $s$ verläuft in der y-z-Ebene durch die Mitte der Profildicke $t(s)$. Die Dicke ist nicht zwangsläufig über $s$ konstant, muss aber deutlich kleiner als alle anderen Abmessungen sein (dünnwandig) \cite{item15}. Man kann vereinfacht annehmen
\begin{equation}
	\iint dA=\int t(s)ds.
\end{equation} 
Der Querschnitt muss in x-Richtung konstant sein und auch bei Belastung seine Gestalt beibehalten. Der Schubfluss im Profil ist als
\begin{equation}\label{tau}
	q=\tau t(s)
\end{equation}
und der Normalkraftfluss als
\begin{equation}
	n_x=\sigma_x t(s)
\end{equation}
definiert. Für diese Kraftflüsse gilt das hydrodynamische Analogon. Das heißt mit einer Änderung der Dicke muss der Kraftfluss antiproportional ab- bzw. zunehmen, damit das Kräftegleichgewicht erfüllt bleibt. Für Knotenpunkte muss auch gelten, dass der Betrag der Kraftflüsse in den Knoten hinein denen aus ihm heraus gleichen. Des Weiteren lässt sich noch zwischen offenen und geschlossenen Profilen unterscheiden, wobei es sich bei zweiterem um Ein- oder Mehrzeller handeln kann.\cite{item15}

\paragraph{Koordinatensysteme}~\\
Bei den Berechnungen kann viel Arbeit gespart werden, indem das Koordinatensystem mit Ursprung und Achsenausrichtung klug gewählt wird. Das allgemeine Koordinatensystem, das unabhängig vom betrachteten Profil ist, bietet hierbei die wenigstem um nicht zu sagen keine Vorteile. Verschiebt man seinen Ursprung in den Schwerpunkt erhält man das Schwerpunkt-Koordinatensystem. Es wird mit einem Querstrich über den Koordinaten gekennzeichnet ($\bar{x},\bar{y},\bar{z}$). Es ist sinnvoll aus dieser Perspektive den Schubfluss zu betrachten.
\begin{figure}[h]
	\centering
	\includegraphics[width=1\textwidth]{Bilder/schubfluss-infinit}
	\caption{Infinitesimales Profilelement aus \cite{item15}}
	\label{schubfluss-infinit}
\end{figure}
Aus dem Kräftegleichgewicht in $x$- und $s$-Richtung an einem infinitesimalen Volumen des dünnwandigen Profils ergibt sich der Zusammenhang zwischen den Kraftflüssen zu
\begin{equation}
	\frac{\partial n_x}{\partial x} + \frac{\partial q}{\partial s} = 0
\end{equation}
\begin{equation}\label{n_s}
\frac{\partial n_s}{\partial s} + \frac{\partial q}{\partial x} = 0
\end{equation}
Da jedoch eine Krümmung in $s$-Richtung vorliegen kann, zeigt die Resultierende des Normalkraftflusses $n_s$ aus der Ebene heraus. Da bei dünnen Scheibenelementen ein ebener Spannungszustand herrscht, darf dies nicht sein. Daraus folgt, dass beide Terme in Gleichung \ref{n_s} gleich null sein müssen. Der Schubfluss ist also in $x$-Richtung konstant und lässt sich somit zu
\begin{equation}
	q(s)=-\int\frac{\partial n_x(x,s)}{\partial x}ds+q_0.
\end{equation}
integrieren. Wird nun der Normalkraftfluss in Abhängigkeit von der Querkraft $Q$ gesetzt, ergibt sich die $Q$-$SI$nen-Formel nach \cite{item15}:
\begin{equation}\label{qs}
	q(s)=-(Q_{\bar{z}}\frac{S_{\bar{y}}(s)I_{\bar{z}}-S_{\bar{z}}(s)I_{\bar{yz}}}{I_{\bar{y}}I_{\bar{z}}-I_{\bar{yz}}^2}+Q_{\bar{y}}\frac{S_{\bar{z}}(s)I_{\bar{y}}-S_{\bar{y}}(s)I_{\bar{yz}}}{I_{\bar{z}}I_{\bar{y}}-I_{\bar{yz}}^2})+q_0
\end{equation}
Mit den Flächenträgheitsmomenten $I$ (siehe Gleichungen \ref{FT1}-\ref{FT3}). Spätestens hier zeigt sich, dass für dieses Modell das Superpositionsprinzip anwendbar ist und man die Kräfte $Q_y$ und $Q_z$ getrennt betrachten kann.
Hier lässt sich direkt erkennen, wie diese Formel durch die Wahl eines besseren Koordinatensystems vereinfacht werden kann. Für das Hauptachsen-Koordinatensystem bleibt der Schwerpunkt weiterhin der Ursprung, jedoch werden die Achsen so um die x-Achse gedreht, dass die Deviationsmomente $I_{yz}$ verschwinden. Es wird mit einem Dach über den Koordinaten gekennzeichnet ($\hat{x},\hat{y},\hat{z}$). Die Integrationskonstante
\begin{equation}
	q_{0} = q_{0b}+q_{0T}
\end{equation}
ist nur bei geschlossenen Profilen ungleich null. Sie sich aus einem Teil, der aus der Biegung entsteht $q_{0b}$ und einem aus der Torsion $q_{0T}$ zusammen. Es gibt einen Punkt in der y-z-Ebene für jedes Profil, wo eine angreifende Kraft keine Torsion verursacht. Dieser Punkt ist von hoher Bedeutung und wir Schubmittelpunkt genannt.
\subsubsection{Idealisierung(H.G.)}
Für die Berechnung des Schubmittelpunkts wird das Flügelprofil als vereinfachter Mehrzeller angenommen.
Dabei wird der Ursprung des Koordinatensystems am unteren rechten Rand gesetzt. Das Modell wird in 10 Teilstrecken $s_{i}$ aufgeteilt. Die Dicke $t$ wird über die Schale konstant angenommen. Der Schaum wird in allen Abschnitten der Schale und im Holm vernachlässigt, da seine Hauptaufgabe der Beulsteifigkeit gilt und er bei der Schubaufnahme im Vergleich zum Laminat nur eine unbedeutende Rolle spielt. Der Steg ist in 2 Abschnitte unterteilt. Für die maximale Belastung und somit für die Auslegung relevant ist nur der Teil mit dem dünneren Gewebe, sodass hier für die Dicke des Stegs $t_{1}=0,314\mathrm{mm}$ angenommen wird. Für die Gurte gilt, dass alle Fasern parallel in x-Richtung ausgerichtet sind. Sie werden in dieser Rechnung ignoriert, da sie folglich im Gegensatz zum $\pm45^\circ$-Gewebe vernachlässigbare Schubkräfte aufnehmen können (D = $t$). Außerdem würde ein über die Dicke variabler Schubmodul die Rechnung unnötig verkomplizieren, da die uns bekannten Methoden zur Schubflussberechnung nur für homogene Werkstoffe gültig sind \cite{item15}.

Diese Annahmen bezüglich des Schaums und der Gurte sind unproblematisch, da sie so getroffen wurden, dass der Flügel sogar noch höheren Belastungen als errechnet standhalten könnte. Als einziges ist zu betrachten, dass durch das Wegfallen des Schaums in der Schale die GFK-Schichten des Sandwichs aufeinander fallen und ihre Position somit um wenige Millimeter ungenau ist. Da die genau errechnete Dicke auf eine Lagenschicht aufgerundet werden muss, ist dies vermutlich kompensierbar, sollte jedoch im Hinterkopf behalten werden.

Abbildung \ref{Fluegel1} zeigt, dass der Profil durch 4 geraden Strecken und einen viertel Kreis modelliert wurde. Die Längen der einzelnen Teilabschnitte lassen sich im Anhang der Abbildung \ref{fig:LaengenS} entnehmen.
\begin{figure}[h]
 \centering
 \includegraphics[width=1\textwidth]{Bilder/Model1}
 \caption{vereinfachtes Model}
 \label{Fluegel1}
\end{figure}
\subsubsection{Schwerpunktkoordinaten(H.G.)}\label{SP-Koord}
Zunächst wird von einer Dicke $t=0,2\mathrm{mm}$ ausgegangen und die statischen Momente in den einzelnen Teilstücken berechnet.
\begin{equation}\label{SM1}
	S_{z}=\int_{A}^{}y \mathrm{d}A =t\int_{s}^{}y \mathrm{d}s
\end{equation}
\begin{equation}\label{SM2}
	S_{y}=\int_{A}^{}z \mathrm{d}A =t\int_{s}^{}z \mathrm{d}s 
\end{equation}
Damit ergeben sich für die statischen Momente in $\mathrm{mm}^3$:

\begin{center}
\begin{tabular}[h]{l|c|c}
	
Bauteilabschnitt&$S_{y}/\mathrm{mm}^3$&$S_{z}/\mathrm{mm}^3$\\
\hline
1& -281,25&160,54\\
2&-120,382&124,42\\
3&-220,78&605,24\\
4&-97,13&163,27\\
5&-571,91&2555,63\\
6&-58,13&965,14\\
7&0&1789,88\\
8&0&163,80\\
9&0&124,60\\
10&0&140,63\\
\end{tabular}
\end{center}

\noindent Nun kann aus den statischen Momenten der Schwerpunkt bestimmt werden, indem das Statische Moment über alle Flächen zusammenaddiert wird:
\begin{equation}
	y_{0}=\frac{\int_{s}{}t(s)y\mathrm{d}s}{\int_{s}{}t(s)\mathrm{d}s}=\frac{S_{z}}{A}
\end{equation}
\begin{equation}
	z_{0}=\frac{\int_{s}{}t(s)z\mathrm{d}s}{\int_{s}{}t(s)\mathrm{d}s}=\frac{S_{y}}{A}
\end{equation}
Mit einer Dicke von $t=0,2\mathrm{mm}$ ergeben sich Lagen der Schwerpunktkoordinaten zu:
$$
	y_{0}=78,53\mathrm{mm}
$$
$$
	z_{0}=-15,39\mathrm{mm}
$$

Damit können nun mit
\begin{equation}
	S_{\bar{y}}=S_y-z_0A
\end{equation} 
\begin{equation}
S_{\bar{z}}=S_z-y_0A
\end{equation} 
die Verläufe der statischen Momente in Bezug auf den Schwerpunkt ermittelt werden, wobei die Endwerte der vorherigen Verläufe nach dem hydrodynamischen Analogon den Anfangswert $s_0$ des folgenden Bereichs bestimmen. Dabei ergeben sich folgende Endwerte nach \cite{TMk} in $\mathrm{mm}^3$:
\begin{center}
\begin{tabular}[h]{l|c|c}
Bauteilabschnitt&$S_{\bar{y}}/\mathrm{mm}^3$&$S_{\bar{z}}/\mathrm{mm}^3$\\
\hline
1&-99,91&-764,60\\
2&-159,18&-860,06\\
3&-39,53&-319,43\\
4&-252,75&-1236,10\\
5&-493,03&-372,27\\
6&-458,59&120,60\\
7&-201,65&599,69\\
8&-158,55&543,61\\
9&-155,45&448,34\\
10&0&0\\
\end{tabular}
\end{center}
Für ein offenes Profil muss gelten, dass an den freien Rändern der Schubfluss $ 0 $ ist, was automatisch dadurch erfüllt ist, dass beide statischen Momente ungefähr null sind. Das Profil wird in diesem Fall an den Stellen 1 und 3 geschnitten.


Nun werden die Flächenträgheitsmomente 
\begin{equation}\label{FT1}
	I_{y} = \int_{A}^{}z^2dA
\end{equation}
\begin{equation}
I_{z} = \int_{A}^{}y^2dA
\end{equation}
\begin{equation}\label{FT3}
I_{yz} = \int_{A}^{}zydA
\end{equation}
 zuerst aus ihrem eigenen Schwerpunkt aus errechnet, sodass die auf den Gesamtschwerpunkt bezogenen Flächenträgheitsmomente $I_{\bar{y}}$, $I_{\bar{z}}$ und $I_{\bar{y}\bar{z}}$ mit Hilfe des Steinerschen Satz
 \begin{equation}
 	I_{\bar{y}} = I_{y} + z^2A
 \end{equation}
\begin{equation}
I_{\bar{z}} = I_{z} + y^2A
\end{equation}
\begin{equation}
I_{\bar{y}\bar{z}} = I_{yz} + zyA
\end{equation}
 ermittelt werden können.
Es ergibt sich:
\begin{center}

\begin{tabular}[h]{l|c|c|c||c|c|c}
Bauteilabschnitt&$I_{y}/\mathrm{mm}^4$&$I_{z}/\mathrm{mm}^4$&$I_{zy}/\mathrm{mm}^4$&$I_{\bar{y}}/\mathrm{mm}^4$&$I_{\bar{z}}/\mathrm{mm}^4$&$I_{\bar{y}\bar{z}}/\mathrm{mm}^4$\\
\hline
1&2857,61&2857,61&140,63&5649,01&22688,56&-7299,54\\
2&0,83&44,91&6,06&1255,75&3299,07&2026,88\\
3&1379,88&0,10&0&1512,59&8665,70&1072,37\\
4&0,83&44,91&6,06&1043,48&1189,34&1098,42\\
5&373,09&20459,31&2762,56&3053,01&55094,88&-6871,78\\
6&187,25&265,93&-223,13&284,51&20659,15&2599,67\\
7&0,06&9689,11&0&3955,08&23439,99&7374,62\\
8&0,01&45,731&0&663,45&1168,88&-863,21\\
9&0,01&45,731&0&663,45&3287,88&-1466,61\\
10&0,03&878,91&0&1777,09&27679,54&-6901,18\\
\hline
$\sum{}$&-&-&-&19957,41&187173,00&-9230,37
\end{tabular}
\end{center}
Es kann sofort erkannt werden, dass $I_{\bar{y}\bar{z}} \neq 0$ ist und es sich somit nicht um ein Hauptachensystem handelt. Das Deviationsmoment  $I_{\bar{y}\bar{z}}$ ist jedoch im Vergleich zu den anderen Flächenträgheitsmomenten sehr niedrig. 
Aus dem Zusammenhang
\begin{equation}
	tan(\varphi)=\frac{2I_{\bar{y}\bar{z}}}{I_{\bar{z}}-I_{\bar{y}}}
\end{equation}
nach \cite{item15} lässt sich erkennen, dass der Winkel ($\varphi =-3,15^\circ$) zwischen dem Hauptachsen- und Schwerpunkt-Koordinatensystem nur sehr gering ist. Im Folgenden finden alle Betrachtungen trotzdem weiterhin mit den Koordinatenachsen in dieser Ausrichtung statt.

\subsubsection{Schubmittelpunkt(H.G.)}
\begin{figure}[h]
	\centering
	\includegraphics[width=1\textwidth]{Bilder/Flügel-Pol-offen}
	\caption{offenes Profil mit Pol}
	\label{Fluegel2}
\end{figure}
Nun ist es von Interesse, und im Rahmen der Aufgabenstellung auch gefordert, den Schubmittelpunkt, an dem eine angreifende Kraft reine Biegung ohne Torsion bewirkt, zu bestimmen.
Zunächst wird der Schubmittelpunkt des wie in Abschnitt \ref{SP-Koord} aufgeschnittenen Profils betrachtet (siehe Abb. \ref{Fluegel2}). Wir betrachten die Kraft $Q$, die im Schubmittelpunkt angreift und äquivalent zu $q(s)$ nach
\begin{equation}
	Q=\int_{s}^{}q(s)ds
\end{equation}
ist. Gleichzeitig muss aus der Momentenäquivalenz gelten
\begin{equation}
	Qr=\int_{s}q(s)r(s)ds
\end{equation}
wobei $r$ der jeweilige Hebelarm zu einem beliebigen Pol ist. Unter Anwendung des Superpositionsprinzips lässt sich die Querkraft $Q$ in ihre Komponenten der Koordinatenrichtungen zerlegen und jeweils eine gleich null setzten. Zusammen mit Gleichung (\ref{qs}) erhält man dadurch die folgenden Gleichungen zur Bestimmung des Schubmittelpunkts beim offenen Profil nach \cite{item15}:
\begin{equation}
	y_{M}=\frac{-I_{\bar{z}}\int S_{\bar{y}}(s) r_{t}\mathrm{d}s+I_{\bar{yz}}\int S_{\bar{z}}(s) r_{t}\mathrm{d}s}{I_{\bar{y}}I_{\bar{z}}-I_{\bar{yz}}^2}
\end{equation}
\begin{equation}
	z_{M}=\frac{-I_{\bar{yz}}\int S_{\bar{y}}(s) r_{t}\mathrm{d}s+I_{\bar{y}}\int S_{\bar{z}}(s) r_{t}\mathrm{d}s}{I_{\bar{y}}I_{\bar{z}}-I_{\bar{yz}}^2}
\end{equation}
Daraus folgt:
$$
	y_{M}=174,74\mathrm{mm}
$$
$$
	z_{M}=-37,62\mathrm{mm}
$$
Im nächsten Schritt wird das Profil geschlossen, sodass ein Zweizeller entsteht. An den vorher noch geschnittenen Kanten kann nun Schubfluss herrschen. Dies wird durch die Konstanten $q_{0b,1}$ und $q_{0b,2}$ in der jeweiligen Zelle erreicht.
Da die Verwindung $\vartheta$ für beide Zellen gleich und im Falle der weiterhin reinen Biegung null sein müssen, erhält man für jede Koordinatenrichtung zwei Gleichungen mit zwei unbekannten.
\begin{equation}
	\vartheta_{1}=\vartheta_{2}=0
\end{equation}
Nach \cite{item15}:
\begin{equation}
	\vartheta_{i} = \frac{1}{2A_{0i}G}\oint\frac{q(s)}{t(s)}ds
\end{equation}
\begin{equation}
	\vartheta_{i} = \frac{1}{2A_{0i}G}(\oint\frac{q_{offen}(s)}{t(s)}ds+q_{0b,i}\oint\frac{1}{t(s)}ds-q_{0b,i\pm1}\int\frac{1}{t(s)})
\end{equation}
Die Verwindung ist als spezifischer Drillwinkel mit
\begin{equation}
	\vartheta = \frac{d\varphi}{dx}
\end{equation}
definiert, wobei der $\varphi$ den Drillwinkel angibt.
Mit den Werten für die umschlossenen Flächen
$$
	A_{01}=3087,57\mathrm{mm}^2
$$
$$
	A_{02}=1616,41\mathrm{mm}^2
$$
ergeben sich die Konstanten:
$$
	q_{0b,1\bar{z}}=-1,72\cdot10^{-2}\mathrm{mm}Q_{\bar{z}}
$$$$
	q_{0b,2\bar{z}}=-4,47\cdot10^{-4}\mathrm{mm}Q_{\bar{z}}
$$$$
	q_{0b,1\bar{y}}=-2,54\cdot10^{-3}\mathrm{mm}Q_{\bar{y}}
$$$$
	q_{0b,2\bar{y}}=-4,53\cdot10^{-4}\mathrm{mm}Q_{\bar{y}}
$$
Mit den Schubfluss des geschlossenen Profils in die Momentenäquivalenz eingesetzt, ergibt sich der Schubmittelpunkt ($y_{Mg}, z_{Mg}$) zu:
\begin{equation}
	Q_{\bar{z}}(y_{Mg}-y_{M})=\sum_{i=0}^{2}q_{0b,i\bar{z}}2A_{0,i}
\end{equation}
\begin{equation}
Q_{\bar{y}}(z_{Mg}-z_{M})=\sum_{i=0}^{2}q_{0b,i\bar{y}}2A_{0,i}
\end{equation}

$$
	y_{M_{g}}=70,09\mathrm{mm}
$$
$$
	z_{M_{g}}=-20,46\mathrm{mm}
$$
\begin{figure}[h]
	\centering
	\includegraphics[width=1\textwidth]{Bilder/SMP}
	\caption{Schubmittelpunkt des geschlossenen und offenen Profils}
\end{figure}
\subsubsection{Torsion (O.S.)}
\begin{figure}[h]
	\centering
	\includegraphics[width=1\textwidth]{Bilder/Torsion}
	\caption{Angreifende Kraft und positive Drehrichtung}
\end{figure}
In den bisherigen Berechnungen wurde immer davon ausgegangen, dass die Kraft im Schubmittelpunkt angreift. Durch den Versuchsaufbau ist jedoch vorgegeben, dass eine Prüflast in z-Richtung an den $l/4$-Linie aufgebracht wird. Mit dem errechneten Schubmittelpunkt lässt sich erkennen, dass ein Torsionsmoment 
\begin{equation}
	M_{xT}=(y_{l/4}-y_{0})\cdot F
\end{equation}
entsteht, das noch zusätzlichen Schubfluss und eine Verwindung $\vartheta$ bewirkt. Genauer gesagt wird hier die St. Vernatsche Torsion behandelt, bei der zwar Verwölbung auftritt, diese aber nicht behindert wird \cite{item15}. Wieder können die konstanten Schubanteile $q_{0T,i}$ aus der Momentenäquivalenz und der Verwindung bestimmt werden. Wobei zu beachten ist, dass die Verwindung hier nicht mehr wie bei der reinen Biegung null ist, aber über die Verträglichkeitsbedingung weiterhin gelten muss
\begin{equation}
	\vartheta_{1}=\vartheta_{2}=\vartheta.
\end{equation}
Da die bisherigen Anteile des Schubflusses definitionsgemäß weder Verwindung verursachen, noch bezüglich des Schubmittelpunkts ein Moment kompensieren, fallen sie aus den Gleichungen raus. Es bleiben nur noch die Konstanten $q_{0T,i}$ übrig. Damit ergeben sich die vereinfachten Formeln für die Verwindung und die Momentenäqivalenz:
\begin{equation}\label{verdrillung}
	\vartheta_{i} = \frac{1}{2A_{0i}G}(q_{0T,i}\oint\frac{1}{t(s)}ds-q_{0T,i\pm1}\int\frac{1}{t(s)}ds)
\end{equation}
\begin{equation}
		M_{xT}=2\sum q_{0T,i}\cdot A_{0i}
\end{equation}
Für eine maximale Prüflast von $F=500\mathrm{N}$ die der Flügel aushalten muss, erhält man
$$
	q_{0T,1}=-1,45\mathrm{N/mm}
$$
$$
	q_{0T,2}=-1,51\mathrm{N/mm}.
$$
Setzt man nun diese Werte in Gleichung (\ref{verdrillung}) erhält man für beide Zellen einer Verwindung von
$$
	\vartheta =-0,00195 ^\circ.
$$
Integriert man diesen konstanten Wert über die gesamte Länge des Flügels, erhält man den Drillwinkel am Ende mit $\varphi = -1,5^\circ$.

\subsubsection{Schubspannung (O.S.)}
Das Ziel dieser Berechnung war nicht nur den Schubmittelpunkt und den Drillinkel zu bestimmen, sondern hauptsächlich die Schubspannung in der Haut zu erhalten, um diese auslegen zu können. Da die Hautdicke auf dem gesamten Umfang konstant bleiben soll, ist an dieser Stelle nur die maximale Schubspannung für alle Bereiche der Haut $\tau_{max}$, die sich einfach aus Gleichung (\ref{tau}) ermitteln lässt, interessant. Die Dicke $t$ ist jedoch nicht einfach aus den gesamten Rechnungen rausziehbar, weil es auch Anteile, wie zum Beispiel den Holm, gibt, die von $t$ unabhängig sind. Deswegen wird iterativ vorgegangen, wobei zuerst mit einer zufällig gewählten Dicke die Rechenschritte durchgeführt werden, in dem in diesem Kapitel gezeigten Fall mit $t=0,2\mathrm{mm}$. Am Ende der Berechnung wird aus den maximalen Schubfluss die minimale Dicke
\begin{equation}
	t_{min} = \frac{q_{max}}{R_{\tau}}.
\end{equation} 
bestimmt. Die Festigkeit bei reiner Schubbelastung $R_{\tau} = 166,67\mathrm{N/mm^2}$ wurde mittels ELamX bestimmt. Diese Dicke kann jedoch nicht als einfach so als Ergebnis genommen werden, da sich der Schubfluss mit variabler Dicke mit verändert. Die Rechnung muss mit dem neuen Wert erneut durchgeführt werden. Somit nähert man sich Schritt für Schritt dem Idealwert, bei dem die vorliegende Dicke der minimalen Dicke entspricht.

Nach einigen Iterationen bildet sich der Wert $t = 0,03\mathrm{mm}$ als Grenzwert heraus. Da dies ungefähr 0,4 Lagen bei einer Dicke $0,0783\mathrm{mm}$ des $\pm45^\circ$-Gewebes entspricht, muss für die Fertigung des Flügels auf eine ganzzahlige Lagenanzahl aufgerundet werden. Hier wurde eine Dicke von 
\begin{equation}
	t = 0,1566 \mathrm{mm}
\end{equation}
gewählt, was $2$ Lagen entspricht. Dies macht einen symmetrischen Aufbau der Haut möglich und gewährleistet die beste Beulsteifigkeit. Außerdem ist eine ausreichende Sicherheit in Folge der getroffenen Vereinfachungen gewährleistet.

Mit dieser Dicke verändern sich die Werte des endgültigen Flügels im Vergleich zu den bisher in diesem Kapitel mit $t=0,2\mathrm{mm}$ durchgeführten Berechnungen. Der Schubmittelpunkt verschiebt sich minimal zu
$$
	y_{M_{g}}=69,29\mathrm{mm}
$$
$$
	z_{M_{g}}=-22,85\mathrm{mm}.
$$
Auch der Drillwinkel vergrößert sich leicht durch die gesenkte dünnere Haut Steifigkeit zu
$$
	\varphi =-1,87 ^\circ.
$$
Aus den Verläufen der Schubflüsse, wie sie in Abbildung \ref{fig:S1}-\ref{fig:S10} zu sehen, erkennt man, dass die betragsmäßig maximale Schubspannung am Endpunkt des Bereichs 8 auftritt.
$$
	|\tau_{max}|=\tau(s_8=14)=42,35\mathrm{N/mm^2}\leq R_{\tau}
$$
Im Steg (Bereich 3) tritt zwar ein deutlich höherer Schubfluss auf, jedoch ergibt sich durch die ebenfalls deutlich höhere Dicke eine geringere Spannung.








\newpage
\subsection{Auslegung der Flügelschale nach Klassischer Laminattheorie (T.B.)}
\subsubsection{eLamX (T.B.)}

Auch hier wird die Rechnung nach Handbuch-Methoden mittels eLamX überprüft. \\

\noindent Der Lagenaufbau ähnelt dem des Steges, lediglich die Anzahl der Lagen wird auf Zwei reduziert. Auch hierbei ergibt sich eine Sicherheit $>1$, wie dem Kapitel \ref{Abbildungen} zu entnehmen ist.
\newpage
\subsection{Beulabschätzung der Flügelschale (T.B.)}
Nachdem die Flügelhaut auf Festigkeit ausgelegt worden ist, muss überprüft werden, ob der Effekt des Beulens auftritt.\\

\noindent Dazu werden folgende Annahmen getroffen:

\begin{enumerate}
	\item Die Flügelschale wird als nahezu ebener, unendlicher Streifen betrachtet. Die tatsächliche Krümmung und die Knicke erhöhen die Stabilität deutlich, sodass durch diese konservative Rechnung die Sicherheit gewährleistet bleibt.
	\item Es tritt die ermittelte Schubspannung in der Schale und eine aufgeprägte Druckspannung von dem Holm auf. Vernachlässigt wird, wie in der Auslegung nach  VDI 2013, eine Druckspannung durch die angreifende Prüfkraft selbst.
	\item Die Flügelschale wird als an beiden Rippen fest gelagert betrachtet. Eine Verformung der Rippen wird außer Acht gelassen.
	\item Die Flügelnase und Klappenkante werden als stützende Lager angenommen, da diese einen engen Krümmungsradius aufweisen, welcher stabile und steife Strukturen hervorbringt.
	\item Der Holmsteg wird ebenfalls an seinen beiden Enden, verbunden mit der Flügelschale über die Holmgurte, als stützendes Lager angenommen.
	\item Dadurch ergibt sich die maximale Breite eines Streifens zu $124,5 mm$ - von der Klappenkante zum oberen Steg-Lager.
\end{enumerate}

\noindent Die Berechnung erfolgt nach den gleichen Formeln der Berechnung des Kapitels \ref{Beulsicherheit Steg}. \\

\noindent Das Seitenverhältnis beträgt für die maximale Streifenbreite
\begin{equation}
	\frac{b}{a}=\frac{124,5 mm}{773 mm}=0,16
\end{equation}
und somit wird der Beulfaktor zu 
\begin{equation}
	k_{s}=5
\end{equation} 
ermittelt. Auch hier beträgt die Steifigkeitserhöhung $\kappa=3$ und der Schubmodul $\hat{G}_{xy}=8577,8 MPa$. Dadurch lässt sich die zulässige Schubspannung ermitteln:
\begin{equation}
	\tau_{krit}=3\cdot 5\cdot 8577,8 MPa\cdot\biggl(\frac{2\cdot 0,078mm + 3mm}{124,5 mm}\biggr)^{2} =82,68 MPa
\end{equation}
Die Schaumdicke wird durch konstruktive Vorgaben auf $3 mm$ gesetzt. Dia maximal auftretende Schubspannung beträgt
\begin{equation}
	\tau_{max}=42,346 MPa
\end{equation}
(im Bereich der unteren Steg-Lagerung), sodass die Sicherheit gegen Schubbeanspruchung 
\begin{equation}
	j_{1}=1,95
\end{equation}
beträgt. Der Beulfaktor bei Druckbeanspruchung beträgt
\begin{equation}
	k_{d} = 4,9
\end{equation}
Mit dem E-Modul $E_{Fl\ddot{u}gelschale} = 6639,8 MPa$ kann die kritische Druckspannung zu
\begin{equation}
	\sigma_{krit} = 3\cdot 4,9\cdot 6639,8 Mpa\cdot\Bigl(\frac{2\cdot 0,078mm+3mm}{124,5mm}\Bigr)^{2}=62,72 MPa
\end{equation}
bestimmt werden. Es wird angenommen, dass die Flügelschale eine Druckspannung durch die aufgeprägte Dehnung des Gurtes erfährt, die wiederum aus der maximalen Biegespannung im Holm erfolgt:
\begin{equation}
	\sigma_{max}=E_{Fl\ddot{u}gelschale}\cdot\frac{\sigma_{b}}{E_{Gurt}}=6639,8 MPa\cdot\frac{224,96 MPa}{31580 MPa}=47,29 MPa
\end{equation}
Dadurch, dass die Sicherheit gegen Druckbeanspruchung
\begin{equation}
	j_{2}=1,32
\end{equation}
beträgt, resultiert eine Gesamtsicherheit gegenüber einer Beulerscheinung
\begin{equation}
	j=\sqrt{\frac{1}{\frac{1}{1,952}+\frac{1}{1,32}}}=1,09
\end{equation}\\

 \noindent Dieses stellt jedoch eine theoretische Sicherheit dar, da die maximale Streifenbreite und die größte Schubspannung nicht im selben Bauteil-Bereich liegen und die maximale Randfaserspannung durch Biegung nicht an den tatsächlich sich reduzierenden Höhenverlauf angepasst worden ist. Die Sicherheit ist in der Realität somit höher, ebenfalls wird sie durch den tatsächlich gekrümmten Verlauf erhöht.
\newpage
\section{Numerische Berechnung}\label{FEM}
\subsection{Vorteile der Finiten Element Methode (O.S.)}
Nachdem der Flügel nun analytisch ausgelegt wurde, stellt sich die Frage, ob eine numerische Herangehensweise hier überhaupt noch sinnvoll ist. In den vorherigen Kapiteln wurde viele Vereinfachungen angenommen, um die Berechnungen mit verhältnismäßigen Aufwand zu bewältigen. Je komplexer ein Bauteil ist, desto unwirtschaftlicher wird es, dieses in seiner Fülle analytisch zu berechnen oder gar unmöglich, wenn keine geschlossenen Lösungen bekannt sind. Im Leichtbau werden diese Details benutzt, um Beiteile an de Stellen zu verstärken, wo besonders große Lasten auftreten (z.B. Rippen). Wenn diese nun für eine einfachere Berechnung wegfallen, muss das Restliche Bauteil robuster ausgelegt werden, was zu einer vermeidbaren Gewichtszunahme führt. Auch wenn es sich bei numerischen Lösungen nur um Annäherungen an den wahren Zustand handelt, kann durch einen hohen Vernetzungsgrad ein präziseres Ergebnis erzielt werden als das vereinfachte analytische. Somit übernimmt die nummerische Berechnung auch eine Kontrollfunktion.
\subsection{Funktionsweise der Finiten Element Methode (O.S.)}
\subsubsection{Schwache Lösung der Elastostatik}
Für die Berechnung der Elastostatik sind die Gleichgewichtsbedingung (\ref{GGB}),  Verzerrungs-Verschie\\bungsbedingung (\ref{VVB}) und das Stoffgesetz (\ref{SG}) auch bei der Finiten Elemente Methode (FEM) ausschlaggebend
\begin{equation}\label{GGB}
	\underline{0} = \underline{X} + \underline{\underline{D}}^T \sigma 
\end{equation}
\begin{equation}\label{VVB}
	\underline{\epsilon} = \underline{\underline{D}} \underline{u}
\end{equation}
\begin{equation}\label{SG}
	\underline{\sigma} = \underline{\underline{E}} \underline{\epsilon}
\end{equation}
\cite{item14}. Wobei \underline{$X$} der Vektor der Volumenkräfte, \underline{\underline{$E$}} die Steifigkeitsmatrix, \underline{$\epsilon$} der Verzerrungsvektor, \underline{$u$} der Verschiebungsvektor und \underline{\underline{$D$}} die Operatormatrix ist.\\
Um einer aufwendigen Bestimmung der analytischen Lösung zu entgehen, bedient sich die FEM an dem Prinzip der \textit{schwachen Lösung}. Hierbei hat man eine Differenzialgleichung, die in dem betrachteten Gebiet gleich null ist. Für die Elastostatik kann man hierbei die Gleichgewichtsbedingung verwenden. Diese kann man mit $\delta\underline{u}$ multiplizieren und über das Gebiet integrieren, sodass man
\begin{equation}
	\int_{V}^{}\delta\underline{u}^T\underline{\underline{D}}^T\underline{\sigma}\,dV + \int_{V}^{}\delta\underline{u}^T\underline{X}\,dV = 0
\end{equation}
erhält. Umgeformt ergibt sich das zu
\begin{equation}\label{SL}
	\int_{V}^{}\delta\underline{u}^T\underline{\underline{D}}^T\underline{\underline{E}}\underline{\underline{D}}\underline{u}\,dV = \int_{O_{p}}^{}\delta\underline{u}^T\underline{p}\,dO_p + \int_{V}^{}\delta\underline{u}^T\underline{X}\,dV
\end{equation}
Wobei die Terme auf der rechten Seite den Lasten entsprechen, die auf das Volumen $V$ und die mit $p$ belastete Oberfläche $O_{p}$ wirken. Die schwierig zu lösende Differenzialgleichung hat sich nun schon zu einem Integrationsproblem vereinfacht. Daraus kann die Verschiebung bestimmt werden, weswegen man dies auch die Weggrößenmethode nennt. Die Verzerrungen und Spannungen erhält man aus Nachrechnungen, die mit den Gleichungen (\ref{VVB}) und (\ref{SG}) berechnet werden. Diese Gleichung ist noch ganz allgemein fürs Kontinuum gültig. Im nächsten Schritt wird das Gebiet in eine finite Menge von Elementen zerteilt.
\subsubsection{Diskretisierung}
Die Werte der Verschiebung $\underline{u}$ werden nur an Aufpunkten, den sogenannten Knoten, bestimmt. Mittels einer von Laufvariable $\underline{x}$ abhängigen Formfunktion $\underline{\underline{N}}$ wird der Verlauf von einem Knoten zu seinen Nachbarn definiert, um wieder einen kontinuierlichen Verlauf zu erhalten. Hierbei muss die Formfunktion an dem Knoten, von dem sie ausgeht, den Wert $1$  und bei jedem anderen den Wert $0$ annehmen. Allgemein ergibt sich somit
\begin{equation}
	\underline{u}(\underline{x})=\underline{\underline{N}}(\underline{x})\underline{u}^{(e)}
\end{equation}
wobei $\underline{u}^{(e)}$ der Vektor der Verschiebungen eines Elementes $e$ ist.
Wenn man nun diese Gleichung in die Gleichung der schwachen Lösung (\ref{SL}) einsetzt, lässt sich $\delta(\underline{u}^{(e)})^T$ aus den Integralen raus ziehen und kürzen, da es von $\underline{x}$ unabhängig ist.
\begin{equation}
	\int_{V}^{}\underline{\underline{N}}^T\underline{\underline{D}}^T\underline{\underline{E}}\underline{\underline{D}}\underline{\underline{N}}\,dV\underline{u}^{(e)} = \int_{O_{p}}^{}\underline{\underline{N}}^T\underline{p}\,dO_p + \int_{V}^{}\underline{\underline{N}}^T\underline{X}\,dV
\end{equation}
Wobei sich das linke Integral zu der Elementsteifigkeitsmatrix $\underline{\underline{K}}^{(e)}$ ergibt.
\begin{equation}
	\underline{\underline{K}}^{(e)} = \int_{V}^{}\underline{\underline{N}}^T\underline{\underline{D}}^T\underline{\underline{E}}\underline{\underline{D}}\underline{\underline{N}}\,dV
\end{equation}
Die einzelnen Elementsteifigkeitsmatrixen lassen sich zu einer Gesamtsteifigkeitsmatrix zusammenfassen, mit der dann die Lösung berechent wird.
\newpage
\subsection{ABAQUS Analyse (H.G.)}

\subsubsection{Modell}
Die numerische Analyse findet mit Hilfe des Programms ABAQUS des Anbieters Dassault Systemes statt. Das Programm ist für Studenten kostenlos erhältlich, jedoch ist diese Version auf eine Anzahl von 1000 Knoten beim Vernetzen des Bauteils beschränkt.\\
Das zuvor erstellte CAD-Modell wird zunächst in ABAQUS importiert. Da es sich nun noch um ein Bauteil aus Vollmaterialien handelt und für eine möglichst genaue Analyse ein Schalenmodell verwendet werden soll, muss das Modell zunächst umgewandelt werden. Dafür werden die Volumenkörper des Modells in zweidimensionale Flächen umgewandelt, denen im Folgenden dann eine Dicke zugeordnet werden kann (siehe Abb.\ref{Schalenmodell}).

\begin{figure}[h]
 \centering
 \includegraphics[scale=0.4]{Bilder/Flügel_ABAQUS}
 \caption{Schalenmodell}
 \label{Schalenmodell}
\end{figure}
\noindent
Daraufhin werden die Kennwerte der Materialien in das Programm integriert. Hierbei werden die Werte für GFK, den Schaum und die Rippen eingetragen (siehe Abb.\ref{Material}). Diese wurden in Kapitel \ref{elamx} mit Hilfe von $elamX^{2}$ festgelegt. Für GFK wird das Material als Laminat erstellt, wodurch der E-Modul parallel und senkrecht zur Faser, sowie die Schubmoduln definiert werden können.\\
\noindent
Nun wird das Schalenmodell in die verschiedenen Segmente unterteilt, denen dann Materialien und deren entsprechende Dicken und Anzahl von Schichten zugeordnet werden. Hierbei wird das Bauteil in den dicken und den dünnen Teil des Stegs, die Gurte, die Haut, sowie die Rippen unterteilt. Für die Bauteile aus GFK werden erst zwei zum Flügelkoordinatensystem um 45 und -45 Grad gedrehte GFK-Schichten definiert (siehe Abb. \ref{Steg_Material}), dann der Schaum und daraufhin erneut zwei versetzte GFK-Schichten. Eine Schicht hat hierbei ein Viertel der zuvor berechneten Dicke ohne den Schaum. Für die Rippen wird eine Dicke von 3mm angenommen.\\
\noindent
Als nächstes werden die Randbedingungen und angreifenden Kräfte festgelegt. Der Holm ist in der Einspannung fest gelagert. Daher wird an der Stelle der Lager A und B eine feste Einspannung angenommen. Die aufgenommenen Kräfte durch die Querkraftbolzen werden als Kräfte angebracht. Die Prüfkraft wird am Ende des Flügels an der L/4-Linie eingeleitet.\\
Daraufhin wird das Bauteil vernetzt. Hierbei wird eine Anzahl an Knoten pro Längeneinheit eingegeben und das Programm verbindet diese zu einem Netz aus Flächen. Durch die Begrenzung auf 1000 Knoten ist das Netz jedoch recht grob, wodurch die Berechnung ebenfalls eine gewisse Ungenauigkeit aufweisen kann. \\
\newpage
\subsubsection{Analyse}
Zunächst wird eine Prüfkraft von 100N angelegt, um die maximale Absenkung zu ermitteln. Hierbei ergibt sich für die Absenkung $z_{max}=17,34mm$, welche erwartungsgemäß an der Spitze des Flügels auftritt. Dies ist weniger als die geforderte Absenkung von $z_{max}=20mm$, auf die der Flügel in den vorherigen Kapiteln ausgelegt wurde, und somit kann dieses Aufgabenkriterium als erfüllt angesehen werden (siehe Abb.\ref{Absenkung}). Die Abweichung von diesem Wert lässt sich dadurch begründen, dass die vorherigen Berechnungen davon ausgegangen sind, dass nur der Holm die Biegung trägt. Auch wenn dies eine gute Näherung ist und für die Auslegung auch sinnvoll, trägt in Realität auch die Schale einen kleinen Anteil zur Biegesteifigkeit bei.
\begin{figure}[h]
 \centering
 \includegraphics[scale=0.4]{Bilder/Absenkung_100N}
 \caption{Absenkung bei einer Belastung von 100N}
 \label{Absenkung}
\end{figure}
\noindent
Um herauszufinden, ob der Flügel eine hinreichende Festigkeit aufweist, wird die Prüfkraft auf 500N erhöht. Nun müssen die maximalen Spannungen im Bauteil mit den Kennwerten verglichen werden. Das Modell zeigt nun maximale Spannungen an der Stelle des Lagers B an. Da sich dort allerdings noch eine nicht mit modellierte Verstärkung durch Holzblöcke befindet, sind diese Werte deutlich höher als in der Realität. Daher werden nur die Werte betrachtet, die sich ab der Wurzelrippe ergeben.\\
Die maximale Vergleichsspannung (Mises) ergibt sich im Steg des Holms. Diese wir hier nicht weiter behandelt, da bei ihr die Information über den Spannungszustand verloren geht und dieser für einen isotropen Stoff ausschlaggebend ist. Die Normalspannung in $x_{1}$-Richtung ist im vorderen Teil des Stegs am höchsten und beträgt beim Maximum $85,63\frac{N}{mm^2}$.\\
Die Normalspannung in $x_{2}$-Richtung ist im Gurt der Unterseite des Flügels am höchsten (im Versuchsaufbau Oberseite) und beträgt maximal $14,25\frac{N}{mm^2}$. Die Spannung in $x_{3}$-Richtung ist überall 0, da bei den Schalenelementen von einem ebenen Spannungszustand ausgegangen wird.\\
Die Schubspannung ist in der Schale auf der Unterseite am höchsten. Hier ergibt sich ein Maximalwert von $16,72\frac{N}{mm^2}$.\\
\noindent
Das angenommene Koordinatensystem bezieht sich auf die oberste Schicht des Laminats. Daher sind $x_{1}$ und $x_{2}$ immer in Richtung der Faser oder orthogonal dazu.\\
Da die Spannungen nicht in allen Schichten gleich sein müssen, haben sie allerdings eine geringe Aussagekraft. Eine bessere Aussage lässt sich durch die maximalen Dehnungen in den einzelnen Koordinatenrichtungen treffen. Durch die Annahme, dass die Dehnung in allen Schichten gleich ist, kann diese in elamX eingegeben werden und so eine Sicherheit gegen Bruch ermittelt werden
(siehe Abb. \ref{sicher-steg} - \ref{sicher-gurt}).
\begin{center}
\begin{tabular}[h]{l|c|c|c|c}
Richtung&Gurt oben&Gurt unten&Steg&Haut\\
\hline
$\varepsilon_{x}$&-0,00142&0,00206&0,00276&-0,00107\\
$\varepsilon_{y}$&-0,0015&0,00226&-0,00275&-0,000664\\
$\gamma_{xy}$&-0,00754&0,00843&0,00523&-0,00754\\
minimale Sicherheit&2,332&1,31&2,32&2,785
\end{tabular}
\end{center}

\noindent
Die geringste Sicherheit ergibt sich im unteren Gurt mit 1,31. Dies ist hinreichend, um die Festigkeit des Flügels zu bestätigen.\\
\noindent
Die Werte sind durch das Zusammenspiel aller Bauteile sehr viel kleiner als in den vorherigen Berechnungen. Dadurch bestätigt das FEM- Modell die vorangegangenen Berechnungen.
\newpage
\subsubsection{Beulanalyse}
Da der Holmstummel bis zur Wurzelrippe C nicht ausreichend modelliert werden konnte und die Holzklötze in der Holmwurzel jegliches Beulen stark behindern, wird für die Beulauslegung eine feste Einspannung am Lager C angenommen (siehe Abb.\ref{BEinspannung}).
\begin{figure}[h]
 \centering
 \includegraphics[scale=0.4]{Bilder/Beuleinspannung}
 \caption{Einspannung am Lager C}
 \label{BEinspannung}
\end{figure}\\
\noindent
Nun können die Eigenwerte für die Belastung von 100N, 500N und 1000N abgelesen werden.
\begin{center}
\begin{tabular}[h]{l|c}
Prüfkraft&Beulfaktor\\
\hline
100N&10,253\\
500N&2,0506\\
1000N&1,0253\\
\end{tabular}
\end{center}
\noindent
Diese Werte sind alle größer als 1 und ändern sich antiproportional zur Kraft, damit ist der Flügel für alle drei Belastungsfälle ausreichend gegen das Beulen dimensioniert.
Das Programm ermöglicht außerdem die Ausgabe möglicher Beulformen. Ein Beispiel ist in Abbildung \ref{Beulform} dargestellt. Die Haut würde nah am Holmstummel beulen. 
\begin{figure}[h]
 \centering
 \includegraphics[scale=0.4]{Bilder/Beulen100N}
 \caption{Beispiel Beulform}
 \label{Beulform}
\end{figure}
\newpage
\newpage

 

  

\newpage
\section{Konstruktion und Fertigung des Flügels}
\subsection{CAD-Konstruktion des Flügels (H.K.)}\label{CAD}
Auf Basis der verfeinerten Dimensionierung des Holmes mithilfe von ELAMX und der Beulabschätzung, soll nun ein CAD-Modell des Flügels erstellt werden. Als Grundlage dient eine unvollständige technische Zeichnung der Profilkontur, aus der exakt entnommen werden kann, dass das Profil ohne die Hochauftriebselemente oder Querruder $ 172mm $ tief ist und eine Profildicke von $ 37,5mm $ aufweist. Aus den bekannten Längenangangaben kann der Maßstab der gedruckten Zeichnung zu $ 1:1,039 $ berechnet werden. Mithilfe eines Rechtecks, das die Kontur gerade umschließt, können weitere Punkte auf der Kontur des Profils ermittelt werden. Im CAD-Programm werden Tangentenbögen von Punkt zu Punkt gelegt, um die Kontur hinreichend glatt anzunähern.\\


\noindent In den Bereichen oberhalb und unterhalb des Holms soll die Haut nicht in Sandwich-Bauweise ausgeführt sein. Für die Auslegung des Holms wurde davon ausgegangen, dass eine Dicke des Verbundmaterials der Haut von $ 0,75mm $ ausreichend ist. Zunächst wird davon ausgegangen, dass für die Haut das Gewebe Interglas 90070 verwendet wird, das ein Flächengewicht von $ 80\frac{g}{m^{2}} $ aufweist. Nach Gleichung ~\ref{gurtlagen} entsprechen $ 9 $ Lagen dieses Gewebes der angenommenen Hautdicke. Dies erscheint ausreichend. Sollten weniger Lagen für die Haut benötigt werden, kann der entstehende Freiraum zwischen den Gurten und der Haut aufgefüllt werden. Um die Hautdicke von $ 0,75mm $ im Bereich der Gurte zu berücksichtigen, wird ein Offset von dieser Breite nach innen gerichtet.\\
\noindent Der zu Beginn des Abschnitts ~\ref{GurtDim} dimensionierte Holm mit rechteckigen Gurtquerschnitten, $ b=28mm $ und $ h_{a}=36mm $ wird nun so auf die Kontur des Profils gelegt, dass die Überdeckung der Gurte mit der umgebenden Haut möglichst gering ausfällt. Dann wird die Höhe $ h_{a} $ an den örtlichen inneren Abstand der oberen und unteren Haut auf $ \tilde{h_{a}}=35,8mm $ angepasst. Der rechteckige Querschnitt der Gurte wird mithilfe eines Offsets von $ \tilde{h}=1,941mm $ der Kontur der Haut angepasst. Diese Anpassungsmaßnahmen senken das Flächenträgheitsmoment leicht. Das resultierende Flächenträgheitsmoment $ \tilde{I_{x}} $ lässt sich aufgrund der komplexen Querschnittsgeometrie der Gurte mit dem CAD-Programm bestimmen. Der Vergleich mit dem erforderlichen Flächenträgheitsmoment zeigt, dass die angepasste Geometrie der Gurte die Steifigkeitsbedingung erfüllt.

\newpage
\subsection{Massenabschätzung (H.K.)}
Auf Basis der Dimensionierung der einzelnen Komponenten kann die Masse der Tragfläche abgeschätzt werden. Im Folgenden werden die Volumina bestimmt, um mithilfe der Dichte des jeweiligen Werkstoffes auf die Masse zu schließen. Mithilfe des CAD-Programms können die Volumina zeitsparend und exakt bestimmt werden. Dies ist besonders für komplizierte Geometrien, zum Beispiel für den Schaumkern der Haut hilfreich.
\subsubsection{Masse der Gurte}
Für den oberen und unteren Gurt ergeben sich folgende Werte: 
\begin{equation}
	V_{Gurt,o}=A_{Gurt,o}\cdot \left(l_{0}+l_{1}+l_{2}+l_{3}\right)\\
	=4,983\cdot 10^{-5}m^{3}
\end{equation}
\begin{equation}
	V_{Gurt,u}=A_{Gurt,u}\cdot \left(l_{0}+l_{1}+l_{2}+l_{3}\right)\\
	=4,98\cdot 10^{-5}m^{3}.
\end{equation}
 Im nächsten Schritt wird die Dichte der Faserverbundbauteile bestimmt. Sie errechnet sich gemäß der Mischungsregel nach \cite{item3} zu
\begin{equation}
	\rho_{Verbund}=\varphi\cdot\rho_{f}+\left(1-\varphi\right)\cdot\rho_{m}
	=1728\frac{kg}{m^{3}}.
\end{equation}
Gemäß dieser Formel sind die Dichten der Verbundbauteile unabhängig von den verwendeten Geweben und ihren Flächengewichten, solange das gleiche Fasermaterial und der gleiche Faservolumengehalt vorliegen. Damit ergeben sich die Massen der Gurte zu
\begin{equation}
	m_{Gurt,o}=V_{Gurt,o}\cdot\rho_{Verbund}=0,0861kg
\end{equation}
\begin{equation}
		m_{Gurt,u}=V_{Gurt,u}\cdot\rho_{Verbund}=0,086kg .
\end{equation}

\subsubsection{Masse des Stegs}
Unterteilt in den Anteil des Schaums und den des Verbundmaterials, ergeben sich für den Steg folgende Volumina:
\begin{equation}
	V_{Steg,Verb}=A_{Steg,Verb}\cdot\tilde{h_{i}}=1,813\cdot10^{-5}m^{3}
\end{equation}
\begin{equation}
	V_{Steg,Schaum}=A_{Schaum}\cdot\tilde{h_{i}}=4,953\cdot10^{-5}m^{3}.
\end{equation}
Für die jeweiligen Volumina folgt mit der Dichte $ \rho_{Verbund} $ und der Dichte des Schaums $ \rho_{Schaum}=35\frac{kg}{m^{3}} $:
\begin{equation}
	m_{Steg,Verb}=V_{Steg,Verb}\cdot\rho_{Verbund}=0,031kg
\end{equation}
\begin{equation}
		m_{Steg,Schaum}=V_{Steg,Schaum}\cdot\rho_{Schaum}=0,002kg
\end{equation}
Als Werkstoff für den Schaum steht Styrodur zur Verfügung. Laut Trendbericht aus dem Magazin \textit{Kunststoffe} in der Ausgabe 10/2008 \cite{item7}, beträgt die Dichte von extrudiertem Polystyrol-Hartschaumstoff (XPS) $ 25\frac{kg}{m^{3}} $ bis $ 45\frac{kg}{m^{3}} $. Für die Massenabschätzung wurde der Mittelwert angenommen.

\subsubsection{Masse der Haut}
Das Volumen des Schaumkerns der Haut beträgt 
\begin{equation}
	V_{Haut,Schaum}=A_{Haut,Schaum}\cdot l_{3}=6,143\cdot10^{-4}m^{3}.
\end{equation}
Mit der Dichte $ \rho_{Schaum} $ beträgt die Masse des Kerns
\begin{equation}
	m_{Haut,Schaum}=V_{Haut,Schaum}\cdot\rho_{Schaum}=0,022kg.
\end{equation}
Zur Bestimmung der Masse des Faserverbundanteils in der Haut wird die Länge der abgewickelten Hautschichten mit dem CAD-Programm bestimmt. Für die innere und die äußere Schicht ergeben sich:
\begin{equation}
	l_{Haut,innen}=346,46mm
\end{equation}
\begin{equation}
	l_{Haut,aussen}=367,83mm
\end{equation}
Die Dicke einer Schicht des Interglas 90070 Gewebes mit einer flächenbezogenen Masse von $ 80\frac{g}{m^{2}} $
berechnet sich nach \cite{item3} mit der Formel:
\begin{equation}
	t=n\cdot\frac{m}{Lb}\cdot\frac{1}{\varphi\cdot\rho_{f}}.
\end{equation}
Für eine Schicht, $ \varphi=0,4 $ und $ \rho_{f}=2550\frac{kg}{m^{3}} $ ist $ t_{Haut}=0,078mm $.
Die Breite entspricht Länge $ l_{3} $, damit ergibt sich das Volumen der inneren und äußeren Hautschicht zu:
\begin{equation}
	V_{Haut,innen}=l_{Haut,innen}\cdot l_{3}\cdot t_{Haut}=2,089\cdot 10^{-5}m^{3}
\end{equation}
\begin{equation}
	V_{Haut,aussen}=l_{Haut,aussen}\cdot l_{3}\cdot t_{Haut}=2,218\cdot 10^{-5}m^{3}
\end{equation}
und für die Massen folgt:
\begin{equation}
	m_{Haut,innen}=V_{Haut,innen}\cdot \rho_{Verbund}=0,036kg
\end{equation}
\begin{equation}
	m_{Haut,aussen}=V_{Haut,aussen}\cdot \rho_{Verbund}=0,038kg.
\end{equation}
Für die Gesamtdicke des Verbundanteils der Haut wurde in der Auslegung $ 0,75mm $ vorgesehen. Aus der geringeren tatsächlichen Dicke ergeben sich ober- und unterhalb der Gurte Freiräume mit einer Dicke von $t_{frei}= 0,75mm-2\cdot 0,078mm=0,594mm $. Dieser Freiraum wird mit Harz ausgefüllt und muss zusätzlich berechnet werden. Die Länge des Bereichs, in dem kein Schaumkern die innere und äußere Lage trennt und in der folglich dieser Freiraum auftritt, wird im CAD-Programm für die Oberseite zu $ l_{frei,o}= 31,18mm $ und für die Unterseite zu $ l_{frei,u}=32,1mm $ bestimmt. Die Krümmung in diesem Bereich kann wegen des großen Krümmungsradius und der kleinen Länge vernachlässigt werden. Es ergeben sich die Querschnittsflächen der Freiräume:
\begin{equation}
\begin{array}{l}
		A_{frei,o}=l_{frei,o}\cdot t_{frei}=18,52mm^{2} \\
		A_{frei,u}=l_{frei,u}\cdot t_{frei}=19,07mm^{2}.
\end{array}
\end{equation}
Über die Extrusionslänge der Tragfläche $ l_{3} $ folgt für das Volumen der Freiräume:
\begin{equation}
	\begin{array}{l}
		V_{frei,o}=A_{frei,o}\cdot l_{3}= 1,432\cdot 10^{-5}m^{3} \\
		V_{frei,u}=A_{frei,u}\cdot l_{3}= 1,474\cdot 10^{-5}m^{3} 
	\end{array}
\end{equation}
und mit der Dichte $ \rho_{m}=1180\frac{kg}{m^{3}} $:
\begin{equation}
	\begin{array}{l}
		m_{frei,o}=V_{frei,o}\cdot \rho_{m}=0,017kg\\
		m_{frei,u}=V_{frei,u}\cdot \rho_{m}=0,018kg.
	\end{array}
\end{equation}

\subsubsection{Masse der Holzklötze und Rippen}
Da die Holzklötze ebenfalls eine komplizierte Geometrie aufweisen, wird auch ihre Masse der Einfachheit halber mit dem CAD-Programm berechnet. Als Material wurde in der Bolzenauslegung Buchenholz gewählt,das nach \cite{item19} eine Dichte von $ \rho_{Buche}=610\frac{kg}{m^{3}} $ aufweißt. Die Masse eines jeden Holzklotzes wird damit zu $ m_{Klotz}=0,016kg $ bestimmt.
Auf die gleiche Weise wird zur Bestimmung der Massen der Rippen verfahren. Die Wurzelrippe und die Endrippe bestehen jeweils aus zwei Teilen, die Masse einer Rippe wird mithilfe des Programms für Buchenholz zu $ m_{Rippe}=0,004kg $ bestimmt.Darüber hinaus müssen die Massen der Hülsen ermittelt werden. Als Werkstoff ist Messing vorgesehen, dessen Dichte gemäß \cite{item20} zu $ \rho_{Messing}=8400\frac{kg}{m^{3}} $ angenommen wird. Die beiden Hülsen der Hauptbolzen wiegen damit jeweils $ m_{Haupthuelse}=0,003kg $ und die der Querkraftbolzen jeweils $ m_{Querhuelse}=0,001kg $. Die beiden zur Verbindung mit der Belastungseinheit des Teststandes erforderlichen Gewindehülsen der Endrippe werden aus Stahl gefertigt. Ihre Masse beträgt jeweils $ m_{Endhuelse}=0,001kg $.

\subsubsection{Abschätzung der Verklebungen und der Gesamtmasse}
Einen weiteren Anteil an der Gesamtmasse liefern die Klebeverbindungen. Zur Verklebung der Rippen mit dem Holm sind Holzklötze vorgesehen, die an die Vorder- und Hinterseite des Stegs geklebt werden und dadurch eine ausreichend große Klebefläche zur Verfügung stellen. Die Breite der Klötze wurde im Abschnitt 8.2 zu $ 1,56mm $ berechnet. Zur einfachen Fertigung können Holzleisten mit einem quadratischen Querschnitt von $ 2mm\cdot2mm $ gewählt werden. Bei einer Länge von $ \tilde{h_{i}} $ beträgt die Masse jeder Leiste weniger als $ 0,1g $. Da nur vier Leisten vorgesehen sind, kann die Masse vernachlässigt werden. Die Verklebungen von Steg und Gurt, sowie die Klebestellen zwischen den Halbschalen, sollen mit Mumpe und ohne den Einsatz zusätzlicher Gewebelagen oder Leisten erfolgen. Es wird angenommen, dass die Massen der Klebestellen, die allein an der Kontaktfläche zweier Bauteile liegen, vernachlässigbar sind. Lediglich die Klebebreite zwischen den Gurten und dem Steg in den Bereichen 1 und 2 ist breiter als der Steg selbst, sodass von außen zusätzlich Mumpe aufgetragen werden muss. Die aufzutragende Breite pro Gurtseite beträgt $ 15,9mm-2,314mm=13,59mm $. Es wird ein gleichschenkliger Dreiecksquerschnitt der Klebefuge angenommen, die Länge der Fuge beträgt $ l_{0}+l_{1}+l_{2}=0,143m $ in Holmlängsrichtung. Mit $ \rho_{M}=1180\frac{kg}{m^{3}} $ ergibt sich die Masse der Verklebung zu $ m_{Kleber}=0,015kg $       \\

\noindent Abschließend wird die Gesamtmasse aus der Summe der einzelnen Massen berechnet. Sie ergibt sich zu:
\begin{equation}
\begin{array}{l}
	m_{ges}= m_{Gurt,o}+m_{Gurt,u}+m_{Steg,Verb}+m_{Steg,Schaum} \\ +m_{Haut,Schaum}+m_{Haut,innen}+m_{Haut,aussen} \\ +2\cdot m_{Rippe}+2\cdot m_{Klotz} \\
	+2\cdot m_{Haupthuelse}+2\cdot m_{Querhuelse}+2\cdot m_{Endhuelse}\\
	+m_{frei,o}+m_{frei,u}+m_{Kleber}\\
	=0,366kg
\end{array}  
\end{equation}
Die Gewichtslimitierung von 0,75kg wird eingehalten.


\newpage
\subsection{Bauanleitung für die Fertigung (T.B.)}
Nach der erfolgten Auslegung, Dimensionierung und Konstruktion des Tragflügels erfolgt, die Entwicklungsphase abschließend, der Bau eines Prototyps. Versuchsbauten werden, neben der Erprobung der Fertigung und Funktion, hauptsächlich für Strukturtests zum Nachweis der erfolgten Rechnungen genutzt. Um den Modellflügel des Zaunkönigs fertigen zu können, wird im Folgenden eine grob-strukturierte Anleitung gestellt. Dabei werden jedoch nicht alle notwendigen Nebenschritte genannt, sondern von erfahrenen Handwerker als bekannt vorausgesetzt. Ebenfalls wird nicht auf eine vorherige Fertigung kleinerer Teile eingegangen.\\

\noindent Der Bau des Flügels erfolgt in drei Abschnitten. Zuerst soll der Holm gebaut werden, danach die Flügelschale und abschließend erfolgt die Verklebung der Bauteile.
Es werden beide Negativformen der Profilform zur Verfügung gestellt, verbaut werden die Gewebe Interglas 90070 (bidirektionial) und 92145 (annähernd unidirektional), das Epoxidharz L385 inkl. Härter H386, Mikroballons zum Andicken des Harze (ugs. Mumpe), Messing und Buchenholz. Zur Positionierung von Bauteilen und zum eigenen Formenbau sollen Strangprofile genutzt werden. 

\begin{enumerate}
	\item \textbf{Holm:}
	\begin{enumerate}
		\item Jeweils zwei Aluminiumprofile werden als Formwände für die Holmgurte in beiden Profilformen positioniert. Dadurch wird die Breite garantiert, während die Profilform die Wölbung schafft. Die Flügelform wird durch gefräste Profile verlängert, um den Holmstummel ebenfalls in gewölbter Geometrie bauen zu können. 
		\item Anschließend werden je 9 Lagen Interglas 92145 in diesen Formen laminiert. (Belegung nach Kapitel  4.4) 
		\item Nach dem Aushärten werden sowohl die Gurte als auch die Aluminiumprofile wieder aus der Profilform entnommen. 
		\item Die Gurte werden an den Enden auf die passende Länge geschliffen.
		\item Der Schaumstoff wird zugeschnitten und die Abstufung geschliffen.
		\item Der Schaumstoff wird nun mit angedicktem Harz (Harz vermischt mit Microballons) \glqq abgespachtelt\grqq, um ein Vollsaugen der Poren mit nicht-gehärtetem Harz zu vermeiden. Der folgende Schritt muss so zeitnah erfolgen, dass die Härtung noch nicht erfolgt ist. Dadurch wird eine Nass-Nass-Verklebung des Schaumstoffs mit Geweben garantiert.
		\item Es werden Lagen Interglas 90700 auf einer ebenen Fläche laminiert. Dabei muss auf die Abstufung von 12 auf 2 Gewebelagen und auf die 45°-Ausrichtung geachtet werden. Danach wird der Schaum positioniert und anschließend wieder 12 bis 2 Lagen in umgekehrter Reihenfolge der Abstufung belegt (Belegung nach Kapitel 4.4).
		\item Nach dem Härten des Steg-Sandwichs wird dieser auf das Endmaß geschliffen.
		\item Mittels kleiner Holz-Klebewinkel wird der Steg auf einem Holmgurt positioniert und mit Harz verklebt.
		\item Nun werden die Holzklötze für die Verstiftungen , die Wurzlrippe und die Endrippe eingeklebt. An den verbleibenden Kanten der Verklebung des Steges mit dem Gurt wird mit Mumpe die nötige Klebefläche geschaffen.
		\item Sobald die Mumpe gehärtet ist, wird der andere Gurt auf den Steg, die Rippen und die Holzklötze geklebt und die Klebekanten werden ebenfalls mit Mumpe ausgefüllt.
		\item Abschließend werden in den Steg an den entsprechenden Stellen Löcher gebohrt und die Messinghülsen eingesetzt. Dabei kann die exakte Ausrichtung der Buchsen durch eine gleichzeitige Verbindung des Teststands mit Bolzen/Stiften erfolgen.
	\end{enumerate}
	\item \textbf{Flügelschale:}
	\begin{enumerate}
		\item Anhand der Profilformen wird der Schaumstoff zugeschnitten und die Schrägen werden geschliffen.
		\item Der Schaumstoff wird auch hierbei mit angedicktem Harz \glqq abgespachtelt\grqq, um ein Vollsaugen der Poren mit nicht-gehärtetem Harz zu vermeiden. Der folgende Schritt muss wiederum so zeitnah erfolgen, dass die Härtung noch nicht erfolgt ist. Dadurch wird eine Nass-Nass-Verklebung des Schaumstoffs mit Geweben garantiert.
		\item In beide Profilschalen wird die äußere Lage Interglas 90070 laminiert, anschließend wird der Schaumstoff positioniert und die innere Lage folgt (Belegung nach Kapitel 4.4).
	\end{enumerate}
	\item \textbf{Verklebung:}
	\begin{enumerate}
		\item Nun wird der Holm inkl. der Rippen in die untere Schale mit Mumpe geklebt. Entsprechende Hohlräume sollten mit Mumpe aufgefüllt werden, um eine durchgehend stoffschlüssige Verbindung zu schaffen.
		\item Darauf wird die obere Flügelschale verklebt, dabei wird diese sowohl mit dem Holm, als auch mit der unteren Schale verklebt wird. Da es sich hierbei um eine Blind-Verklebung handelt, ist es zu Empfehlen, etwas überschüssig Mumpe aufzubringen.
		\item Die äußeren Klebekanten beider Flügelschalen werden in Form geschliffen.
		\item Der Flügel ist nun fertiggestellt, abschließend könnte dieser gespachtelt und lackiert werden. Darauf wird jedoch aus Gewichtsgründen verzichtet. Um dabei nicht die äußere Gewebelage zu beschädigen, wäre es dafür angebracht, eine Schutz-Gewebelage zusätzlich in den Lagenaufbau einzuplanen.
	\end{enumerate}
\end{enumerate}

\noindent Als grundlegendes Fertigungsverfahren steht das Handlaminier-Verfahren zur Verfügung. Neben diesem steht außerdem das Vakuumpressen und die Vakuuminjektion zur Auswahl. Ersteres Herstellungsverfahren beruht darauf, dass Gewebe- bzw. Gelegelagen per Hand ausgerichtet und anschließend manuell mit Pinseln und Schaum-Rollen mit Harz getränkt und entlüftet werden. Dieses bringt einen niedrigen Faservolumengehalt, jedoch auch eine individuellere Gestaltung eines Bauteils in offenen Formen mit sich. \\
Beim Vakuumpressen handelt sich um eine Weiterentwicklung dieses Verfahrens. Nach dem eigentlichen Laminieren wird das Bauteil zusätzlich mit einer Folie abgedichtet, sodass die eingeschlossene Luft entzogen werden kann und die Folie somit einen zusätzlichen Druck auf das Bauteil ausübt. Dadurch kann überschüssiges Harz von einem aufgebrachtem Fließ aufgenommen werden, sodass der Faservolumenanteil steigt. Es muss jedoch in Erfahrung gebracht werden, ob bei Sandwich-Laminaten der Stützstoff dem Flächendruck standhalten kann.\\
Der Faservolumenanteil kann weiter gesteigert werden, in dem das Harz nicht während der Belegung eingebracht, sondern nach dem Vakuumieren mittels Unterdruck in das Gewebe gesaugt wird. Hierbei ist jedoch Erfahrung nötig, ob das Harz die geforderte Zähigkeit aufweist, um alle Fasern tränken zu können. Der hohe Faservolumengehalt (ca. 50\%) ist jedoch auch mit einem höherem Aufwand verbunden.\\

\noindent Es erscheint sinnvoll, den Tragflügel im Handlaminat herzustellen. Für die Gewichtseinsparung und Steifigkeitserhöhung (Mischugsregel) durch einen höheren Faservolumengehalt wären die beiden anderen Verfahren vorteilhafter, dieses würde aber nicht den zusätzlichen Aufwand und die nicht zwingend erforderlichen Gewichtsersparnisse rechtfertigen. Außerdem können so kurzfristig kleinere Anpassung der Formgestaltung während der Fertigung einfach durchgeführt werden, was sich für einen Prototypen als unabdingbar gestaltet. Außerdem werden durch die Wahl der Fertigung ein geringerer Erfahrungsschatz und weniger Halb- und Werkzeuge für die Fertigung benötigt.
\newpage
\section{Zusammenfassung}
\subsection{Was ist geschehen H.K.}
\subsection{Gewichtsnormalisiertes Festigkeitskriterium O.S.}
Um eine Vergleichbarkeit zwischen verschiedenen Flügeln zu schaffen wird die gewichtsnormalisierten Festigkeit
\begin{equation}
	P=\frac{m_{\mathrm{Belastung,max}}}{m_{\mathrm{Fluegel}}}
\end{equation}
definiert. Es wird also die Gewichtskraft als Masse $m_{\mathrm{Belastung,max}}$, die der Flügel im Testaufbau maximal aushält, ins Verhältnis mit der Flügelmasse $m_{\mathrm{Fluegel}}$ gesetzt. Auch wenn das Ziel war einen möglichst leichten Flügel, der den in der Aufgabenstellung formulierten Anforderungen entspricht, zu konstruieren, wird somit berücksichtigt, dass mehr Material zwar eine höhere Masse mit sich bringt, aber er wahrscheinlich auch größere Lasten aushalten kann. Üblicher Weise würde man die Bruchlast im Teststand ermitteln, aber auf Grund der COVID-19 Pandemie ist es uns nicht möglich den Flügel zu bauen, geschweige denn zu testen.
	
Eine Möglichkeit wäre es $500$ Newton als Bruchlast anzunehmen, da wir im Rahmen der Projektarbeit nachgewiesen haben, dass unsere Konstruktion dies aushält. Da jedoch immer eher vorsichtige Annahmen getroffen und die errechneten Werte dann meist noch aufgerundet wurden, wäre die gewichtsnormalisierten Festigkeit weit unter dem voraussichtlich im Teststand ermittelten Wert. Somit wäre der Aussagewert sehr gering.
Eine bessere Möglichkeit bietet hier das FE-Modell. Dies ist zwar auch nicht perfekt, da zum einen die Einspannung auf Grund der Holzblöcke nicht hundertprozentig realistisch modelliert werden kann. Die kritischste Stelle liegt nach unseren Überlegungen und durch ABAQUS bestätigt am Ansatz des Holmstummels, also dem Flügelende, dass am Bug anliegt, bzw. an der Stelle, wo der GFK im Steg dicker wird. Diese Stellen liegen weit genug von der Einspannung entfernt, sodass die Werte als plausibel angenommen werden können. Zum anderen ist durch die Studentenversion von ABAQUS der Idealisierungsgrad auf $1000$ Knoten und somit auch die Genauigkeit der Ergebnisse beschränkt. Trotzdem ist dies mit den zur Verfügung stehenden Mitteln der beste Ansatz um die Bruchlast zu ermitteln, da hier als einziges der Flügel als Ganzes modelliert wird.

Da nicht bekannt ist, in welchem Bereich das Material zuerst versagt, wird in Gurt, Steg und Schale jeweils der Ort der höchsten Belastung einzeln betrachtet. Da nur ein einheitlich Wert für die Spannung der Elemente gegeben wird, diese in Realität aber nicht über alle Schichten konstant ist, wird die Dehnung betrachtet, da sie für alle identisch ist. Wird nun die Dehnung in die beiden Achsenrichtungen und der Schubwinkel in ElamX eingegeben, erhält man sofort die Sicherheit gegen den Bruch. Die Ergebnisse sind in Abbildung \ref{Sicherheit1} bis \ref{Sicherheit4} zu erkennen, wobei es zwei verschieden Werte für den Holm gibt, da er auf der einen Seiten auf Druck und auf der anderen auf Zug belastet wird. Der [] hat also die geringste Sicherheit von []. Da die Belastung im Flügel linear mit der anliegenden Kraft steigen, lässt sich der Faktor direkt auf die aufgebrachte Last von $500\mathrm{N}$ übertragen. Somit ergibt sich eine Bruchlast von []N oder mit einer Erdbeschleunigung von $9,81\frac{\mathrm{m}}{\mathrm{s^2}}$
$$m_{\mathrm{Belastung,max}} = []\mathrm{kg}. $$ Eine erneute FEM-Berechnung mit der gesteigerten Kraft liefert neue Dehnungen, die im [] wie erwartet eine Sicherheit von ungefähr $1$ ergeben. Da der kleinste Beulfaktor höher als diese Sicherheit war, kommt es auch nicht bei der erhöhten Last zum Beulen.
Für das Gewichtsnormalisiertes Festigkeitskriterium ergibt sich nun also mit $m_{\mathrm{Fluegel}}$ aus der Massenabschätzung ungefähr ein Wert von
$$P = [].$$
\subsection{Diskussion der Ergebnisse}
\subsection{Optimierungsmöglichkeiten}
optimierung der Gurtlagendicke -> nicht merh per hand analytisch bestimmbar, verschiedene Dicken an verschiedenen Stellen

Berechnung der gegebenen Wurzelrippen -> Absenkung und Gewichtsoptimierung, optimierte Annahmen für die Balkenberechnung

wenigerkonservativ rechnen, Gesamtbauteil statt Aufteilung der Aufgaben, Schaum mit einbeziehen

Holzklotz optimieren, andere Bauteile auf Sicherheit von 1 optimieren

Optimierung des e-Mailverkehrs

andere Materialien 

Optimierung FEM, höherer Idealisierungsgrad
\newpage
\section{Quellenverzeichnis}
\begin{thebibliography}{99}          
	\bibitem{item1}
	H. Hertel:
	\textit {Leichtbau: Bauelemente, Bemessungen und Konstruktionen von Flugzeugen u.a.}.
	Springer Verlag Berlin/Göttingen/Heidelberg, 1960.
	
	\bibitem{item2}
	Mises, R. V.:
	\textit {Fluglehre: Vorträge über Theorie und Berechnung der Flugzeuge in elementarer Darstellung}.
	Springer Verlag, 1936.
	
	\bibitem{item3}
	Helmut Schürmann:
	\textit {Konstruieren mit Faser-Kunststoff-Verbunden}.
	Springer Verlag, 2005.
	
	\bibitem{item4}
	Elmar Witten:
	\textit {Handbuch Faserverbundkunststoffe/Composites: Grundlagen, Verarbeitung, Anwendungen}.
	Springer Vieweg, 4.Auflage, 2014.
	
	\bibitem{item5}
	VDI-Gesellschaft Materials Engineering:
	\textit{VDI 2013 (Blatt 1)-Dimensionierung von Bauteilen aus GFK (Glasfaserverstärkte Kunststoffe)}.
	VDI-Gesellschaft Materials Engineering, 1970.
	
	\bibitem{item6}
	Roland Gomeringer, Roland Kilgus, Volker Menges, Stefan Oesterle, Thomas Rapp, Claudius Scholer, Andreas Stenzel, Andreas Stephan, Falko Wieneke:
	\textit{Tabellenbuch Metall}
	Verlag Europa Lehrmittel, 2014.
	
	\bibitem{item7}
	Wolfgang Glenz:
	\textit{Kunststoffe}
	Carl Hanser Verlag, 10/2008.
	
	\bibitem{item8}
	Peter Wriggers, Udo Nackenhorst, Sascha Beuermann, Holger Spiess, Stefan Löhnert:
	\textit{Technische Mechanik kompakt}
	Vieweg+Teubner Verlag, 2016.

	\bibitem{item9}
	Markus Linke, Eckart Nast:
	\textit{Festigkeitslehre für den Leichtbau, Ein Lehrbuch zur Technischen Mechanik}
	Springer Verlag, 2015.


	
	
\end{thebibliography}
\newpage
\section{Abbildungsverzeichnis}
\begingroup
\renewcommand{\section}[2]{}%
\makeatletter
\renewcommand*\l@figure{\@dottedtocline{1}{1.5em}{3em}}
\makeatother
\listoffigures
\endgroup
\newpage
%\section{Tabellenverzeichnis}
%\begingroup
%\renewcommand{\section}[2]{}%
%\listoftables
%\endgroup
%\newpage
\section{Anhang}
\subsection{Abbildungen}
\begin{figure}[h]
	\includegraphics[width=1.0\textwidth]{Bilder/Lagenaufbau Holmgurte.png}
	\caption{Lagenaufbau Holmgurte}
	\label{fig:Lagenaufbau Holmgurte}
\end{figure}
\begin{figure}
	\includegraphics[width=1.0\textwidth]{Bilder/Lagenaufbau Steg dünn.png}
	\caption{Lagenaufbau Steg Bereich $III$}
	\label{fig:Lagenaufbau Steg dünn}
\end{figure}
\begin{figure}
	\includegraphics[width=1.0\textwidth]{Bilder/Lagenaufbau Steg dick.png}
	\caption{Lagenaufbau Steg Bereich $I$\&$II$}
	\label{fig:Lagenaufbau Steg dick}
\end{figure}
\begin{figure}
	\includegraphics[width=1.0\textwidth]{Bilder/Berechnung Holmgurte.png}
	\caption{Berechnung Holmgurte}
	\label{fig:Berechnung Holmgurte}
\end{figure}
\begin{figure}
	\includegraphics[width=1.0\textwidth]{Bilder/Berechnung Steg dünn.png}
	\caption{Berechnung Steg Bereich $III$}
	\label{fig:Berechnung Steg dünn}
\end{figure}
\begin{figure}
	\includegraphics[width=1.0\textwidth]{Bilder/Berechnung Steg dick.png}
	\caption{Berechnung Steg Bereich $I$\&$II$}
	\label{fig:Berechung Steg dick}
\end{figure}
\begin{figure}
	\includegraphics[width=1.0\textwidth]{Bilder/Konstanten Holmgurte.png}
	\caption{Ingenieurskonstanten Holmgurte}
	\label{fig:Ingenieurskonstanten Holmgurte}
\end{figure}
\begin{figure}
	\includegraphics[width=1.0\textwidth]{Bilder/Konstanten Steg dünn.png}
	\caption{BerechnIngenieurskonstantenung Steg Bereich $III$}
	\label{fig:Ingenieurskonstanten Steg dünn}
\end{figure}
\begin{figure}
	\includegraphics[width=1.0\textwidth]{Bilder/Konstanten Steg dick.png}
	\caption{Ingenieurskonstanten Steg Bereich $I$\&$II$}
	\label{fig:Ingenieurskonstanten Steg dick}
\end{figure}



\end{document}
