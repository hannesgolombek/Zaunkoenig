\subsection{Was ist geschehen H.K.}
\subsection{Gewichtsnormalisiertes Festigkeitskriterium O.S.}
Um eine Vergleichbarkeit zwischen verschiedenen Flügeln zu schaffen wird die gewichtsnormalisierten Festigkeit
\begin{equation}
	P=\frac{m_{\mathrm{Belastung,max}}}{m_{\mathrm{Fluegel}}}
\end{equation}
definiert. Es wird also die Gewichtskraft als Masse $m_{\mathrm{Belastung,max}}$, die der Flügel im Testaufbau maximal aushält, ins Verhältnis mit der Flügelmasse $m_{\mathrm{Fluegel}}$ gesetzt. Auch wenn das Ziel war einen möglichst leichten Flügel, der den in der Aufgabenstellung formulierten Anforderungen entspricht, zu konstruieren, wird somit berücksichtigt, dass mehr Material zwar eine höhere Masse mit sich bringt, aber er wahrscheinlich auch größere Lasten aushalten kann. Üblicher Weise würde man die Bruchlast im Teststand ermitteln, aber auf Grund der COVID-19 Pandemie ist es uns nicht möglich den Flügel zu bauen, geschweige denn zu testen.
	
Eine Möglichkeit wäre es $500$ Newton als Bruchlast anzunehmen, da wir im Rahmen der Projektarbeit nachgewiesen haben, dass unsere Konstruktion dies aushält. Da jedoch immer eher vorsichtige Annahmen getroffen und die errechneten Werte dann meist noch aufgerundet wurden, wäre die gewichtsnormalisierten Festigkeit weit unter dem voraussichtlich im Teststand ermittelten Wert. Somit wäre der Aussagewert sehr gering.
Eine bessere Möglichkeit bietet hier das FE-Modell. Dies ist zwar auch nicht perfekt, da zum einen die Einspannung auf Grund der Holzblöcke nicht hundertprozentig realistisch modelliert werden kann. Die kritischste Stelle liegt nach unseren Überlegungen und durch ABAQUS bestätigt am Ansatz des Holmstummels, also dem Flügelende, dass am Bug anliegt, bzw. an der Stelle, wo der GFK im Steg dicker wird. Diese Stellen liegen weit genug von der Einspannung entfernt, sodass die Werte als plausibel angenommen werden können. Zum anderen ist durch die Studentenversion von ABAQUS der Idealisierungsgrad auf $1000$ Knoten und somit auch die Genauigkeit der Ergebnisse beschränkt. Trotzdem ist dies mit den zur Verfügung stehenden Mitteln der beste Ansatz um die Bruchlast zu ermitteln, da hier als einziges der Flügel als Ganzes modelliert wird.

Da nicht bekannt ist, in welchem Bereich das Material zuerst versagt, wird in Gurt, Steg und Schale jeweils der Ort der höchsten Belastung einzeln betrachtet. Da nur ein einheitlich Wert für die Spannung der Elemente gegeben wird, diese in Realität aber nicht über alle Schichten konstant ist, wird die Dehnung betrachtet, da sie für alle identisch ist. Wird nun die Dehnung in die beiden Achsenrichtungen und der Schubwinkel in ElamX eingegeben, erhält man sofort die Sicherheit gegen den Bruch. Die Ergebnisse sind in Abbildung \ref{Sicherheit1} bis \ref{Sicherheit4} zu erkennen, wobei es zwei verschieden Werte für den Holm gibt, da er auf der einen Seiten auf Druck und auf der anderen auf Zug belastet wird. Der [] hat also die geringste Sicherheit von []. Da die Belastung im Flügel linear mit der anliegenden Kraft steigen, lässt sich der Faktor direkt auf die aufgebrachte Last von $500\mathrm{N}$ übertragen. Somit ergibt sich eine Bruchlast von []N oder mit einer Erdbeschleunigung von $9,81\frac{\mathrm{m}}{\mathrm{s^2}}$
$$m_{\mathrm{Belastung,max}} = []\mathrm{kg}. $$ Eine erneute FEM-Berechnung mit der gesteigerten Kraft liefert neue Dehnungen, die im [] wie erwartet eine Sicherheit von ungefähr $1$ ergeben. Da der kleinste Beulfaktor höher als diese Sicherheit war, kommt es auch nicht bei der erhöhten Last zum Beulen.
Für das Gewichtsnormalisiertes Festigkeitskriterium ergibt sich nun also mit $m_{\mathrm{Fluegel}}$ aus der Massenabschätzung ungefähr ein Wert von
$$P = [].$$
\subsection{Diskussion der Ergebnisse}
\subsection{Optimierungsmöglichkeiten}
optimierung der Gurtlagendicke -> nicht merh per hand analytisch bestimmbar, verschiedene Dicken an verschiedenen Stellen

Berechnung der gegebenen Wurzelrippen -> Absenkung und Gewichtsoptimierung, optimierte Annahmen für die Balkenberechnung

wenigerkonservativ rechnen, Gesamtbauteil statt Aufteilung der Aufgaben, Schaum mit einbeziehen

Holzklotz optimieren, andere Bauteile auf Sicherheit von 1 optimieren

Optimierung des e-Mailverkehrs

andere Materialien 

Optimierung FEM, höherer Idealisierungsgrad