\subsection{Die Berechnungen im Überblick (H.K.)}
Im Folgenden werden die Ergebnisse der Auslegungsschritte und Berechnungen zusammengefasst und die wichtigsten Aspekte herausgestellt.\\

\noindent Die gesamte Auslegung folgte übergeordneten \textit{Zielsetzungen}, die das mechanische Verhalten bei quasistatischer Belastung an der Endrippe vorgeben. Bei einer Prüfkraft von $ F_{pruef}=100N $ darf sich die Flügelspitze um höchstens $ w(100N)=22mm $ absenken. Um diese Bedingung sicher zu erfüllen wurde ein Sicherheitsfaktor von $ j=1,1 $ festgelegt, der die zulässige Absenkung auf $ w_{j=1,1}(100N)=20mm $ verringert. Bei einer Prüfkraft von $ F_{pruef}=500N $ darf die Tragfläche nicht versagen und Beulerscheinungen dürfen nicht auftreten. Zusätzlich sollte ein Hauptaugenmerk auf das möglichst geringe Gewicht der Konstruktion gerichtet sein.\\

\noindent Der Auslegung des Holms ging die \textit{Modellierung als Biegebalken} voraus, der mit der schubstarren Balkentheorie nach Bernoulli berechnet wurde. Zur Modellierung wurden die Hauptbolzen als Lager interpretiert, die Querkraftbolzen und die Prüfkraft als äußere Lasten. Vorgegebene Längenangaben des Teststandes und der Tragfläche bestimmen die Maße der einzelnen Bereiche. Für jeden Bereich wurden die Differentialgleichungen des Biegebalkens aufgestellt und gelöst. Die zentralen Ergebnisse sind die Kenntnisse der Schnittkraft- und -momentenverläufe, sowie die erforderliche Biegesteifigkeit von $ EI_{x}=962,552Nm^{2} $ zur Einhaltung der maximalen Absenkung $ w_{j=1,1}(100N) $. Zur Einhaltung der Anforderungen an die Festigkeit sind die Querkräfte $ Q_{1}(500N)=5085,5N $ in Bereich $ I $, $ Q_{2}(500N)=0N $ in Bereich $ II $ und $ Q_{3}(500N)=-500N $ in Bereich $ III $ relevant.\\

\noindent Die \textit{Holmgurte} nehmen das Biegemoment auf und sind damit Hauptbestandteil der Auslegung zur Einhaltung des Steifigkeitskriteriums. Um das erforderliche Flächenträgheitsmoment der Rechteckquerschnitte bei einer möglichst kleinen und damit gewichtsoptimierten Querschnittsfläche zu erreichen, wurden die Gurte in z-Richtung möglichst weit auseinander gelegt und möglichst breit in x-Richtung gestaltet. Die resultierende Gurtdicke wurde mit $ n=9 $ Lagen des annähernd unidirektionalen Gewebes Interglas 92145 erreicht und beträgt damit $ \tilde{h}=1,941mm $. Die Anpassung der rechteckigen Gurtquerschnitte an die gekrümmte Profilkontur der Tragfläche verringerte das Flächenträgheitsmoment leicht, dennoch ist das dadurch entstandene $ \tilde{I_{x}}>I_{min} $ und maximale Randfaserspannung an der Stelle weniger als halb so groß, wie die zulässige Spannung des unidirektionalen Handlaminats.\\

\noindent Der \textit{Holmsteg} wird durch drei Kraftflüsse beansprucht. Durch die stoffschlüssige Verbindung von Gurten und Steg wird die Längsdehnung $ \epsilon_{Gurt} $ dem Steg aufgeprägt und führt zu einem Normalkraftfluss $ n_{\epsilon} $ parallel zu den Gurten. Aufgrund der Absenkung des Holms entstehen Druckkräfte, die einen zu den Gurten senkrechten Kraftfluss $ n_{A} $ erzeugen. Zusätzlich wirkt ein Schubfluss $ q_{s} $. Zur Vorauslegung wurde der Glasfaserverbund durch allgemeine Dimensionierungskennwerte charakterisiert, bezüglich der Steifigkeitseigenschaften mit $ K_{E\#} $ und mit $ K_{\sigma d} $ für die Festigkeitseigenschaften. Mithilfe der VDI 2013 konnten auf diese Weise die Lagenanzahlen zu $ n=20 $ für die Bereiche $ I $ und $ II $ und $ n=2 $ für den Bereich $ III $ bestimmt werden.\\

\noindent Der Tatsache, dass die Auslegung des Holmstegs nach dieser Richtlinie nur überschlägig erfolgt ist, wurde in der detaillierteren Dimensionierung mit der \textit{CLT} Rechnung getragen. Diese berücksichtigt die gegebenen Materialkennwerte des Laminats. Die aus der Balkenanalyse und Schubflussberechnung hervorgehenden Beanspruchungen, sowie die Lagenanzahlen der Vorauslegung wurden als Grundlage für das Festigkeitskriterium nach Puck verwendet. Mithilfe des Laminatrechners ließen sich die Lagenanzahlen und Beanspruchungen variieren und Sicherheiten gegen Faser- und Zwischenfaserbruch bestimmen. Die Analyse hat gezeigt, dass die Lagenanzahlen erhöht werden mussten, auf $ n=24 $ in den Bereichen $I$ und $II$, sowie $ n=4 $ im Bereich $ III $.\\

\noindent Um die \textit{Sicherheit gegen Beulen} für den Holm zu prüfen, wurden anschließend der auf Druck beanspruchte Gurt und der Steg mithilfe der Berechnungsmethoden nach Hertel analysiert. Für den Druckgurt konnte unter Verwendung des E-Moduls und der Gurtdicke die Sicherheit gegen Beulen zu $ j_{Gurt}=1,08 $ bestimmt werden. Der Steg wurde bereichsweise mit den jeweils vorliegenden Biege- und Schubbeanspruchungen untersucht. Die Bereiche $ I $ und $ II $ sind mit Sicherheiten von $ j_{I}=1,148 $, bzw. $ j_{II}=1,59 $ schon ohne die Verwendung eines Schaumkerns sicher gegen Beulen. Für den Bereich $ III $ ist eine Sandwichkonstruktion zur Gewährleistung der Beulsicherheit notwendig. Ein $ 2mm $ starker Schaumkern wurde gewählt, um die Fertigung zu vereinfachen. Damit ergibt sich eine Beulsicherheit von $ j_{III}=6,88 $ im Bereich $ III $ und eine zusätzliche Sicherheit für die Bereiche $ I $ und $ II $.\\

\noindent Die \textit{Klebeverbindungen} zwischen dem Steg und den Gurten werden durch das Aufbringen von Mumpe hergestellt. Mit der gegebenen Schubfestigkeit $ \tau_{Mumpe}=10MPa $ wird eine Klebebreite von $ l=15,9mm $ in den Bereichen $ I $ und $ II $ benötigt, für den schwächer beanspruchten Bereich $ III $ reichen $ l=1,62mm $. Bei einer Prüfkraft von $ F_{pruef}=500N $ wird gemäß der grundlegenden Annahme $ F_{Q}=F_{pruef} $ eine ebenso große Kraft über die Wurzelrippe abgesetzt. Eine schmale Holzleiste mit einer Breite von $ 1,56mm $ kann auf jeder Seite des Stegs die Klebefläche zwischen Wurzelrippe und Steg bereitstellen.\\

\noindent Als Verbindung der Tragfläche zum Teststand sind zwei \textit{Hauptbolzen} und zwei Querkraftbolzen vorgesehen, deren Durchmesser bereits durch die Bohrungen im Teststand festgelegt ist. Beide Hauptbolzen werden bei einer Prüfkraft von $ F_{pruef}=500N $durch eine $ 5085,5N $ beansprucht. Unter Berücksichtigung der Bolzenlänge wurde eine Mindeststreckgrenze des Bolzenwerkstoffs von $ R_{e}=360,8MPa $ ermittelt. Als Werkstoff für alle Bolzen wurde daraufhin S620Q gewählt. Diese Wahl gewährleistet auch, dass die Querkraftbolzen mit einem Durchmesser von $ d=8mm $ sicher ausgelegt sind.\\

\noindent Die im Allgemeinen außerhalb des Schubmittelpunkts der Tragfläche angreifende Belastung, führt zu einem Torsionsmoment und einer Biegebelastung des Flügels. Letztere wird durch den Holm aufgenommen, der unter dieser Annahme bereits dimensioniert wurde, ersteres muss durch die Flügelschale aufgenommen werden und darf hier nicht zu Beulerscheinungen führen. Zur Bestimmung des Schubflusses in der Schale, wurde die Profilkontur durch zehn Abschnitte angenähert und der Schaum der Sandwichkonstruktion vernachlässigt. Für die erste Berechnung wurde die Hautdicke zu $ t=0,2mm=konst. $ angenommen. Bezüglich der vordersten Kante und der speziell definierten Koordinatenrichtungen des angenäherten Profils liegt der Schwerpunkt bei $ y_{0}=78,53mm $ und $ z_{0}=-15,39mm $. Über die Flächenträgheits-, Deviations- und statischen Momente wurde die Lage des Schubmittelpunkts im Falle des offenen Profils lokalisiert, der bei $ y_{M}=207,60mm $ und $ z_{M}=-41,35mm $ und damit, wie erwartet, außerhalb des Profils liegt. Die Kontur wurde geschlossen und über die Momentenäquivalenz die Lage des Schubmittelpunktes zu $ y_{Mg}=76,78mm $, $ z_{Mg}=-21,56mm $ berechnet. Für den enstandenen Zweizeller wurde unter Annahme gleicher Verwindung beider Zellen der Drillwinkel bei einer Belastung von $ F_{pruef}=500N $ im l/4-Punkt zu $ \varphi=-1,87^{\circ} $ ermittelt. Darüber hinaus konnte mit den Kenntnissen auftretender Schubflüsse die maximale Schubspannung ermittelt werden. Zur Dimensionierung der Hautstärke wurde die Berechnung der Schubflüsse für mehrere Hautstärken $ t $ vorgenommen. Die so bestimmten maximalen Schubflüsse wurden mit den zugehörigen Lagenaufbauten in den Laminatrechner $ eLamX^{2} $ eingegeben, um die Sicherheiten gegen Faser- und Zwischenfaserbruch zu ermitteln. $ n=2 $ Lagen ermöglichen eine Sandwichkonstruktion und sichern die Auslegung gegenüber den Vereinfachungen ab. Daraus ergab sich ein Drillwinkel für die neue Dicke von $ \varphi(500N)=-2,32^{\circ} $.\\

\noindent Die Auslegung nach Handbuchmethoden wurde durch die \textit{Beulabschätzung der Flügelschale} abgeschlossen. Dazu wurde der gefährdetste Profilabschnitt betrachtet, dessen Krümmung vernachlässigt wurde, um die Sicherheit zu erhöhen und um die Rechnung zu vereinfachen. Weiterhin wurde mit der größten auftretenden Schubspannung gerechnet, obwohl diese nicht im betrachteten Bereich vorliegt. Bei Verwendung eines 3mm dicken Schaumkerns ergibt sich eine Sicherheit gegen Beulen von $ j=1,09 $.

\subsection{Die Konstruktion und FEM-Analyse im Überblick (H.K.)}
Als \textit{Grundlage der FEM-Analyse} und zur Veranschaulichung der vorangegangenen Auslegung nach Handbuchmethoden, wurde ein Modell der Tragfläche im CAD-Programm erstellt. Es berücksichtigt alle bisherigen geometrischen Größen. Desweiteren wurden zusätzliche Bauteile und Details erstellt. Für die Sandwichkonstruktion der Flügelschale wurde eine Lösung gefunden, bei der der Schaumkern die beulgefährdeten Bereiche schützt, im Bereich der Gurte jedoch ausläuft, um die Höhe $ \tilde{h_{a}} $ des Holms nicht weiter einzuschränken. Es wurden Wurzel- und Endrippe konstruiert, in denen Bohrungen zur Aufnahme der Querkraftbolzen und zur Verschraubung der Endplatte vorgesehen sind. Messinghülsen wurden für die Bohrungen der Bolzen erstellt, um die betreffenden Bereiche zu schützen und die Montage zu erleichtern. Auch die Holzklötze, die die Auflagefläche der Hauptbolzen am Steg vergrößern, wurden konstruiert. Alle Bauteile wurden zum Schluss im CAD-Programm zusammengebaut. Die jeweiligen Dichten wurden den Komponenten zugeordnet und ermöglichten eine Massenabschätzung des Tragflügels zu $ m_{ges}=0,366kg $. \\

\noindent Der im CAD-Programm zusammengebaute Tragflügel wurde zur \textit{FEM-Analyse in Abaqus} importiert. Die Analyse dünnwandiger Bauteile wird durch die Verwendung von Schalenmodellen erleichtert, sodass der Volumenkörper im ersten Schritt in Flächen umgewandelt wurde. Die Materialkennwerte für Styrodur und GFK wurden in das Programm integriert und das Laminat zusammengestellt. Anschließend wurde den einzelnen Flächen die Dicke des zugehörigen Bauteils, bei GFK-Teilen die jeweilige Lagenanzahl zugeordnet. Lager A und B wurden als feste Einspannung modelliert, die Querkraftbolzen und Prüfkraft als äußere Lasten, letztere mit Angriffspunkt im l/4-Punkt. Die Analyse hat zeigt, dass sich die Flügelspitze bei einer Prüfkraft von $ F_{pruef}=100N $ um $ w(100N)=17,34mm $ absenkt. Dieses Ergebnis bestätigt die Einhaltung der Anforderung an die Steifigkeit die eine Absenkung von $ w_{zul}=20mm $ erlaubt. Um die Einhaltung der Festigkeitsanforderungen zu prüfen, wurde ein Prüfkraft von 500N eingestellt. Spannungsspitzen ergaben sich im Bereich der Hauptbolzen, die jedoch durch die nicht modellierten Holzklötze gemildert würden. Genauere Aussagen zur Ertragbarkeit der Belastungen, lieferte die Analyse der größten aufgetretenen Dehnungen im Faserkoordinatensystem, die als für alle Lagen gleich groß angenommen wurden. Die Dehnungen konnten in $elamX^{2}$ übertragen werden, um auf diese Weise die Sicherheiten gegen Faser- und Zwischenfaserbruch an einer kritischen Stelle zu ermitteln. Bei einer Prüfkraft von $ F_{pruef}=500N $ ergaben sich die kleinsten Sicherheiten gegen Zfb im Steg zu $ j_{min}=2,32 $, in der Haut zu $ j_{min}=2,785 $, am oberen Gurt zu $ j_{min}=2,332 $ und am unteren Gurt zu $ j_{min} =1,312$. Es wird vermutet, dass diese, gegenüber der beabsichtigten Auslegung, großen Sicherheiten auf die konservativen Annahmen der einzelnen Auslegungsschritte zurückzuführen sind. Die gegenseitige Stützung der Komponenten untereinander, die in der Auslegung nach Handbuchmethoden nicht berücksichtigt werden konnten, wird so deutlich.\\

\noindent Mithilfe des FEM-Programms wurde auch eine \textit{Beulanalyse} durchgeführt. Da der Holmstummel nicht genau genug modelliert werden konnte, wurde vereinfachend eine Einspannung an den Querkraftbolzen angenommen. Nahe dieser Einspannung tritt das Beulen am ehesten auf. Der Beulfaktor entspricht dem Sicherheitsfaktor bzgl. einer Beulform. Selbst bei $ F_{pruef}=1000N $ beträgt er noch $ 1,0253 $, damit wird die Konstruktion als sicher gegen Beulen im beabsichtigten Bereich der Prüfkraft angenommen.\\

\noindent Abschließend wurde das gewichtsnormalisierte Festigkeitskriterium eingeführt, dass die Güte der Konstruktion hinsichtlich Belastbarkeit in nur einem Wert zusammenfasst. Zur Ermittlung dieses Wertes wird maximal ertragbare Belastung in Kilogramm durch die Flügelmasse geteilt. Das Kriterium eignet sich besonders zum Vergleich verschiedener konstruktiver Lösungen. Die maximale Belastbarkeit des in dieser Arbeit vorgestellten Flügels wurde mithilfe von ABAQUS und $ eLamX^{2} $ ermittelt. Die Dehnungen bei $ F_{pruef}=500N $ wurden an kritischen Stellen der einzelnen Komponenten bestimmt und in $ eLamX^{2} $ übertragen. Die geringste Sicherheit gegen Zwischenfaserbruch wurde im unteren Gurt lokalisiert. Sie beträgt $ j_{min}=1,31 $. Die Anpassung der Prüfkraft auf $ F_{pruef}=650N $ senkte diese Sicherheit auf ca. 1, sodass $ 650N\hat{=}66,26kg $ als maximale Belastbarkeit angenommen wurde. Das gewichtsnormalisierte Festigkeitskriterium liefert damit einen Wert von $ P=181,04 $. Diese Ergebnisse wurden als Diskussionsgrundlage zum Vergleich mit zwei anderen Tragflächen verwendet. Es stellte sich heraus, dass die in dieser Arbeit behandelte Tragfläche im Vergleich zu einer anderen Lösung mit den gleichen Anforderungen um ca. $ 17\% $ leichter ist und dennoch $ 30\%$ größere Lasten erträgt. Auch wurde ein Vergleich zu einer im Teststand erprobten Tragfläche gezogen. Diese erreichte ein $ P=66,38 $ und wog $ 90\% $ mehr als der Gegenstand dieser Arbeit. Trotz der guten Ergebnisse im Vergleich zu anderen Arbeiten, wurden für diese Tragfläche mehrere Optimierungsmöglichkeiten identifiziert, die einerseits Leichtbaupotentiale in der analytischen Auslegung, andererseits in der numerischen Strategie eröffnen.   
