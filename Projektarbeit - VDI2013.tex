\begin{document}
Auf Basis der in der Balkenberechnung bestimmten Parameter Biegesteifigkeit, maximales Biegemoment und der maximalen Querkraft, sollen die Gurte und der Steg dimensioniert werden. Die Vorauslegung soll dabei anhand der VDI- Richtlinie 2013 erfolgen, diese enthält in einem Unterkapitel Informationen speziell zur Auslegung eines I-Trägers. Dabei ist zu beachten, dass bei einigen Berechnungen Vereinfachungen angenommen werden, die an den betreffenden Stellen spezifiziert werden. Zusätzlich sei angemerkt, dass die gesamte erste Auslegung ohne Sicherheitsfaktoren erfolgt. Grund dafür ist die Annahme, dass in den bereitgestellten Materialkennwerten ausreichende Sicherheiten verrechnet worden sind.

\subsection{Dimensionierung der Gurte zur Einhaltung der Anforderungen an die Steifigkeit}
Bei der Auslegung der Gurte auf Steifigkeit wird angenommen, dass der Steg des I-Trägers keine Längskräfte aufnimmt und der Biegung nicht entgegenwirken kann. Die in der Balkenberechnung ermittelte Biegesteifigkeit $ EI_{x} $, die erforderlich ist, damit bei einer Kraft $ F_{pruef}=100N $ die Flügelspitze eine Absenkung von 22mm erfährt, muss allein durch die Gurte aufgebracht werden. Im Sinne der kraftflussgerechten Gestaltung sollen die Glasfasern unidirektional in Längsrichtung des Gurtes angeordnet werden.
\end{document}