\begin{thebibliography}{99}          
	\bibitem{item1}
	H. Hertel:
	\textit {Leichtbau: Bauelemente, Bemessungen und Konstruktionen von Flugzeugen u.a.}.
	Springer Verlag Berlin/Göttingen/Heidelberg, 1960.
	
	\bibitem{item2}
	Mises, R. V.:
	\textit {Fluglehre: Vorträge über Theorie und Berechnung der Flugzeuge in elementarer Darstellung}.
	Springer Verlag, 1936.
	
	\bibitem{item3}
	Helmut Schürmann:
	\textit {Konstruieren mit Faser-Kunststoff-Verbunden}.
	Springer Verlag, 2005.
	
	\bibitem{item4}
	Elmar Witten:
	\textit {Handbuch Faserverbundkunststoffe/Composites: Grundlagen, Verarbeitung, Anwendungen}.
	Springer Vieweg, 4.Auflage, 2014.
	
	\bibitem{item5}
	VDI-Gesellschaft Materials Engineering:
	\textit{VDI 2013 (Blatt 1)-Dimensionierung von Bauteilen aus GFK (Glasfaserverstärkte Kunststoffe)}.
	VDI-Gesellschaft Materials Engineering, 1970.
	
	\bibitem{item6}
	Roland Gomeringer, Roland Kilgus, Volker Menges, Stefan Oesterle, Thomas Rapp, Claudius Scholer, Andreas Stenzel, Andreas Stephan, Falko Wieneke:
	\textit{Tabellenbuch Metall}
	Verlag Europa Lehrmittel, 2014.
	
	\bibitem{item7}
	Wolfgang Glenz:
	\textit{Kunststoffe}
	Carl Hanser Verlag, 10/2008.
	
	
	\bibitem{TMk}
	Peter Wriggers, Udo Nackenhorst, Sascha Beuermann, Holger Spiess, Stefan Löhnert :
	\textit{Technische Mechanik kompakt}
	Vieweg+Teubner Verlag , 2016
	
	
	\bibitem{EdL}
	Prof. Dr.-Ing. Peter Horst :
	\textit{Elemente des Leichtbaus Vorlesungsskript}
	Institut für Flugzeugbau und Leichtbau, TU Braunschweig , 2020.

	\bibitem{item9}
	Markus Linke, Eckart Nast:
	\textit{Festigkeitslehre für den Leichtbau, Ein Lehrbuch zur Technischen Mechanik}
	Springer Verlag, 2015.


	
	
\end{thebibliography}