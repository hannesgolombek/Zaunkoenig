\subsection{Projektbeschreibung}
Der Zaunkönig ist ein in den frühen 1940er Jahren entstandenes Flugzeug, das unter der Leitung von Hermann Winter an der TU-Braunschweig konstruiert wurde. (Quelle: Wikipedia (bessere Quelle suchen)) Da der Zaunkönig damals vornehmlich aus Holz gebaut wurde soll jetzt ein neuer Flügel im Maßstab 1:4,7 aus Glasfaser Kunststoffverbund (GFK) konstruiert werden. Der Flügel muss gewisse Anforderungen erfüllen, die im nachfolgenden definiert werden.
Bei dem Flügel handelt es sich um einen Rechteckflügel und ist an den Punkten l=1/4 l und l=1/2 l über Verstrebungen mit dem Rumpf verbunden. Diese Streben sollen in der neuen Konstruktion nicht vorhanden sein. Der Flügel soll im Rumpf verstiftet werden, wobei die Torsionsbelastung durch Querkraftbolzen aufgenommen wird. Insgesamt darf der Flügel das Gewicht von 0,750 kg nicht überschreiten.
Um die strukturmechanischen Anforderungen zu erfüllen wird der Flügel auf seine Steifigkeit und Festigkeit geprüft. Die Steifigkeit ist hinreichen, wenn der Flügel bei einer senkrechten Belastung von 100 N am L/4- Punkt, eine Durchbiegung von z=22mm nicht überschreitet. Außerdem darf der Flügel bei einer Prüfkraft von 500 N nicht brechen. Die Haut muss so ausgelegt sein, dass kein Beulen auftritt. Zusätzlich müssen der Torsionswinkel und Torsionsmittelpunkt berechnet werden.
\subsection{Motivation}
Zunächst ist zu klären, warum es überhaupt sinnvoll ist für diesen Flügel GFK zu verwenden. In der Luftfahrt wird immer nach Wegen gesucht das Gewicht zu minimieren, um die Wirtschaftlichkeit von Flugobjekten zu maximieren. Faser-Kunststoffverbunde (FKV) mit ihrer hohen spezifischen Festigkeit stellen hierbei einen idealen Kandidaten dar. Zusätzlich bieten FKV einfache Formgebung für komplexe aerodynamische Profile und auch die Korrosionsbeständigkeit ist höher, als bei konventionellen Werkstoffen. Als ein großer Nachteil ist hier jedoch der hohe Preis zu nennen, der jedoch in unserem Fall keine große Rolle spielt, da wir nur ein Modell entwerfen und der Flügel nicht für hohe Stückzahlen konstruiert wird. Glasfasern sind im Vergleich zu Kohlenstofffasern die günstigere Variante, aber auf Glasfasern wird in Kapitel 2.1 noch mal genauer eingegangen.
\subsection{Herangehensweise}
Im ersten Kapitel werden zunächst verschieden Lösungsvarianten und allgemeine Informationen zum Thema GFK Materialien vorgestellt. Daraufhin wird eine Lösungsvariante festgelegt. Um die Anforderungen zu erfüllen werden in Kapitel 3 Handbuchmethoden verwendet um eine Grobauslegung des Flügels durchzuführen. In Kapitel 4 werden diese dann mit Hilfe der Finiten Elemente Methode verifiziert. In Kapitel 5 werden die Ergebnisse mit den Konstruktionen vorheriger studentischer Arbeiten verglichen.