
\subsection{Projektbeschreibung (H.G.)}
Der LF1 \textit{Zaunkönig} ist ein in den frühen 1940er Jahren entstandenes Flugzeug, das unter der Leitung von Hermann Winter an der Technischen Hochschule Braunschweig konstruiert wurde. Da der Zaunkönig vornehmlich aus Holz gebaut wurde, soll jetzt ein neuer Flügel im Maßstab 1:4,7 aus Glasfaser-Kunststoffverbund (GFK) konstruiert werden. Der Flügel muss gewisse Anforderungen erfüllen, die im Folgenden definiert werden.
Bei der Tragfläche handelt es sich um einen Rechteckflügel, der im Original über Verstrebungen mit dem Rumpf verbunden ist. Diese Streben sollen in der neuen Konstruktion nicht vorhanden sein. Der Flügel soll im Rumpf verstiftet werden, wobei die Torsionsbelastung durch Querkraftbolzen aufgenommen wird. Insgesamt darf der Flügel das Gewicht von 0,750 kg nicht überschreiten.
Um die strukturmechanischen Anforderungen zu erfüllen wird der Flügel auf seine Steifigkeit und Festigkeit geprüft. Die Steifigkeit ist hinreichend, wenn der Flügel bei einer senkrechten Belastung von $ F_{pruef}=100N $ an der Endrippe eine Durchbiegung von $ w=22mm $ nicht überschreitet. Außerdem darf der Flügel bei einer Prüfkraft von $ F_{pruef}=500N $ nicht brechen. Der Flügel muss so auf Stabilität ausgelegt sein, dass kein Beulen auftritt. Zusätzlich sollen der Torsionswinkel und Schubmittelpunkt berechnet werden.
\subsection{Motivation (O.S.)}
Zunächst ist zu klären, warum es sinnvoll ist für diesen Flügel GFK zu verwenden. In der Luftfahrt wird immer nach Wegen gesucht das Gewicht zu minimieren, um die Wirtschaftlichkeit von Flugobjekten zu maximieren. Faser-Kunststoffverbunde (FKV) stellen hierbei mit ihrer hohen spezifischen Festigkeit einen idealen Kandidaten dar. Zusätzlich bieten FKV einfache Formgebungsmöglichkeiten für komplexe aerodynamische Profile und auch die Korrosionsbeständigkeit ist höher als bei konventionellen Werkstoffen. Als ein großer Nachteil ist der hohe Preis zu nennen, der in diesem Fall jedoch keine große Rolle spielt, da nur ein Modell entworfen wird und der Flügel nicht für hohe Stückzahlen optimiert wird. Glasfasern sind im Vergleich zu Kohlenstofffasern die günstigere Variante, auf sie wird in Kapitel \ref{Glasfaser} noch mal genauer eingegangen.
\subsection{Lösungsstrategie (H.G.)}
Um eine möglichst optimale Lösung für die Aufgabenstellung zu finden wird zunächst das Vorgehen festgelegt. Die Dimensionierung des Holms erfolgt im ersten Schritt analytisch nach Handbuchmethoden. Um die maximalen Beanspruchungen zu errechnen, wird der Holm als Balken modelliert und analysiert. Mit diesen Lasten wird der Holm unter Zuhilfenahme der VDI 2013 ausgelegt. Darauffolgend findet eine Überprüfung der Beulsicherheit und Dimensionierung der Klebeverbindungen und Bolzen statt. Als letzter Schritt der analytischen Berechnung, wird der Schubmittelpunkt und die Verdrillung ermittelt.\\
Um die analytischen Berechnungen zu verifizieren soll eine numerische Berechnung mittels eines FEM-Programms stattfinden. Dafür wird ein CAD-Modell erstellt, welches dann in ein FEM-Programm importiert und analysiert wird.\\
Zuletzt soll ein Vergleich zu anderen Lösungen der Aufgabenstellung gezogen werden und mögliche Optimierungsmöglichkeiten der Auslegung gesucht werden.\\

\noindent Um eine Einführung in die Berechnungen bereitzustellen, sollen anfangs wichtige Grundlagen erläutert werden. Weitergehende Berechnungsmethoden und Theorien werden inhaltlich jedoch erst im Zuge der eigentlichen Berechnung dargestellt, um einen direkten Bezug zwischen Theorie und Anwendung vermitteln zu können. 
