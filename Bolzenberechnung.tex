

\subsubsection{Bolzenberechnung (H.G.)} \label{Bolzenberechnung}
Zunächst muss eine Auslegung für die Flächenpressung erfolgen. Dabei ist für die Buche $\sigma_{p,zul}=60\frac{N}{mm^2}$.
Die projizierte Fläche ist $A=a*d$ , d ist hierbei der Durchmesser des Bolzens und a die Länge des Lochs im Steg.\\
\noindent
Die Flächenpressung ist nun: 
\begin{equation}
	p=\frac{F}{A}=\frac{5085,5\mathrm{N}}{a*8mm}
\end{equation}
Mit der zulässigen Flächenpressung für Buchenholz lässt sich die Gleichung nun nach a auflösen (siehe Glg. \ref{Fpressung}).
\begin{equation}
\label{Fpressung}
	a=\frac{F}{\sigma_{p,zul}*d}=\frac{5085,5\mathrm{N}}{60\frac{N}{mm^{2}}*8mm}=10,59mm
\end{equation}
Somit muss die Holzverstärkung des Stegs an der Stelle der Lager mindestens eine Breite von 10,59$\mathrm{mm}$ aufweisen. Im Weiteren wird eine Breite von 11$\mathrm{mm}$ angenommen.\\
\noindent
Nun werden die Bolzen, die den Holm im für den Versuchsaufbau vorgegebenen U-Profil fixieren, ausgelegt. Die Bolzen sind auf Biegung belastet, wodurch Gleichung \ref{Biegespannung1} angenommen wird.
\begin{equation}
\label{Biegespannung1}
	\sigma_{b}=\frac{M_{b}}{W} 
\end{equation}
 
 \begin{equation}
 \label{WKreis}
 	W_{Kreis}=\frac{\pi*d^{3}}{32}
 \end{equation}
  
 \begin{equation}
 \label{Moment}
 	M_{b}=F*l
 \end{equation}
 Mit Gleichung \ref{WKreis} und Gleichung \ref{Moment} ergibt sich die Biegespannung zu Gleichung \ref{Biegespannung}.
 \begin{equation}
 \label{Biegespannung}
 	\sigma_{b}=F*\frac{32*l}{\pi*d^{3}}
 \end{equation}
Die zu ertragende Kraft ist $5085,5 \mathrm{N}$. Diese lässt sich jetzt noch halbieren, da mit der oben genannten Annahme nur die Hälfte des Bolzens betrachtet wird. Damit ergibt sich die gesuchte Kraft (siehe Glg. \ref{KraftB}).
 \begin{equation}
 \label{KraftB}
 	F_{bel}=\frac{F}{2} =2542,75 \mathrm{N}
 \end{equation}
 Nun kann mit $d=8mm$ und $l=8,559mm$ ein passendes Material gesucht werden. Für Stahl gilt $\sigma_{b,F}\approx1,2*R_{e}$, somit wird $R_{e}$ wie in Gleichung \ref{RE} berechnet. 
 \begin{equation}
 \label{RE}
 	F*\frac{32*l}{1,2*\pi*d^{3}}=360,8\mathrm{MPa}\leq
R_{e} \end{equation}  
\noindent
Mit den getroffenen Annahmen ist S620Q mit einer Streckgrenze von $R_{e}=620\frac{N}{mm^{2}}$ ein geeigneter Stahl, mit dem eine ausreichende Sicherheit gegeben ist.\\
\noindent
Für die Querkraftbolzen wird ebenfalls S620Q verwendet. Die aufzunehmenden Kräfte lassen sich aus dem Kräftegleichgewicht ermitteln. $Q_{1}$ ist die Kraft des Bolzens rechts vom Holm und $Q_{2}$ die Kraft des Bolzens links vom Holm. Damit ergeben sich bei einer Prüfkraft von 500N folgende Kräfte:
$$Q_{1}=409,85N$$
$$Q_{2}=90,15N$$
Für den Benötigten Durchmesser lässt sich nun Gleichung \ref{Biegespannung} nach d umstellen.
\begin{equation}
d=\sqrt[3]{\frac{32*F*l}{1,2*\pi*R_{e}}}
\end{equation}
Mit einem Hebelarm l von 10mm ergibt sich:
\begin{equation}
d=3,82mm
\end{equation}
Da die Querkraftbolzen auf keinen Fall versagen dürfen und in der Massenbilanz nicht relevant sind, werden sie mit einem Durchmesser von 8mm bemessen.\cite{item6}\\
 
 
 
 
  
 
