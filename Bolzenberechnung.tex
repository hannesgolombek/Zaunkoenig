

\subsection{Bolzenberechnung}
Zunächst muss eine Auslegung für die Flächenpressung erfolgen. Dabei ist für Buche $\sigma_{p,zul}=60\frac{N}{mm^2}$
Die projizierte Fläche ist $A=a*d$ , d ist hierbei der Durchmesser des Bolzens und a die Länge des Lochs im Steg.\\
Die Flächenpressung ist nun: 
\begin{equation}
	p=\frac{F}{A}=\frac{5085,5\mathrm{N}}{a*8mm}
\end{equation}
Mit der zulässigen Flächenpressung für Buchenholz lässt sich die Gleichung nun nach a auflösen:
\begin{equation}
	a=\frac{F}{\sigma_{p,zul}*d}=\frac{5085,5\mathrm{N}}{60\frac{N}{mm^{2}}*8mm}=10,59mm
\end{equation}
Somit muss die Holzverstärkung des Stegs an der Stelle der Lager mindestens eine Breite von 10,59$\mathrm{mm}$ aufweisen. Im Weiteren wird eine Breite von 11$\mathrm{mm}$ angenommen.\\
 
Nun werden die Bolzen, die den Holm im für den Versuchsaufbau vorgegebenen U-Profil fixieren, ausgelegt. Die Bolzen sind auf Biegung belastet, wodurch sich die Gleichung:
\begin{equation}
	\sigma_{b}=\frac{M_{b}}{W} 
\end{equation}
,ergibt.\\
 Mit: 
 \begin{equation}
 	W_{Kreis}=\frac{\pi*d^{3}}{32}
 \end{equation}
 und 
 \begin{equation}
 	M_{b}=F*l
 \end{equation}
 ergibt sich die Biegespannung zu
 \begin{equation}
 	\sigma_{b}=F*\frac{32*l}{\pi*d^{3}}
 \end{equation}
Die zu ertragende Kraft ist $5085,5 \mathrm{N}$. Diese lässt sich jetzt noch halbieren, da mit der oben genannten Annahme nur die Hälfte des Bolzens betrachtet wird. Damit ergibt sich:
 \begin{equation}
 	F_{bel}=\frac{F}{2} =2542,75 \mathrm{N}
 \end{equation}
 Nun kann mit $d=8mm$ und $l=8,559mm$ ein passendes Material gesucht werden. Für Stahl gilt $\sigma_{b,F}\approx1,2*R_{e}$, somit muss 
 \begin{equation}
 	F*\frac{32*l}{1,2*\pi*d^{3}}=360,8\mathrm{MPa}\leq
R_{e} \end{equation}  sein.\\

 Mit den getroffenden Annahmen ist S620Q (anderer) mit einer Streckgrenze von $R_{e}=620\frac{N}{mm^{2}}$ ein geeigneter Stahl.\cite{item6}\\
 
 
 
 
  
 
