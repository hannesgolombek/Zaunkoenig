
\subsection{Bolzenberechnung}
Im Folgenden werden die Bolzen, die den Holm im für den Versuchsaufbau vorgegebenen U-Profil fixieren ausgelegt. Die Bolzen sind auf Biegung belastet, wodurch sich die Gleichung:
$$\sigma_{b}=\frac{M_{b}}{W} $$
 ,ergibt.
 Mit: $$W_{Kreis}=\frac{\pi*d^{3}}{32}$$
 und $$M_{b}=F*l$$
 ergibt sich die Biegespannung zu
 $$\sigma_{b}=F*\frac{32*l}{\pi*d^{3}}$$
Die zu ertragende Kraft ist $5085,5 \mathrm{N}$. Diese lässt sich jetzt noch halbieren, da mit der oben genannten Annahme nur die Hälfte des Bolzens betrachtet wird. Damit ergibt sich:
 $$F_{bel}=\frac{F}{2} =2542,75 \mathrm{N}$$
 Nun kann mit $d=8mm$ und $l=14,059mm$ ein passendes Material gesucht werden. Für Stahl gilt $\sigma_{b,F}\approx1,2*R_{e}$, somit muss $$R_{e}\geq F*\frac{32*l}{1,2*\pi*d^{3}}=592,27\mathrm{MPa}$$ sein.\\
 Mit den getroffenden Annahmen ist S620Q mit einer Streckgrenze von $R_{e}=620\frac{N}{mm^{2}}$ ein geeigneter Stahl.\\
Nun muss noch eine Auslegung für die Flächenpressung erfolgen.Dabei ist $p_{zul}=\frac{R_{e}}{1,2}$
Die projizierte Fläche ist $A=a*d$ , d ist hierbei der Durchmesser des Bolzens und a die Länge des Lochs im Steg.\\
Die Flächenpressung ist nun: $$p=\frac{F}{A}=\frac{5085,5\mathrm{N}}{a*8mm}$$
Mit der zulässigen Flächenpressung für S620Q lässt sich die Gleichung nun nach a auflösen:
$$a=\frac{1,2*F}{R_{e}*d}=\frac{1,2*5085,5\mathrm{N}}{620\frac{N}{mm^{2}}*8mm}=1,23mm$$
Somit muss der Steg an der Stelle der Lager mindestens eine Breite von 1,23$\mathrm{mm}$ aufweisen.\cite{item6}
 
 
 
 
  
 
