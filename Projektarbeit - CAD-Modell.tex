Auf Basis der verfeinerten Dimensionierung des Holmes mithilfe von ELAMX und der Beulabschätzung, soll nun ein CAD-Modell des Flügels erstellt werden. Als Grundlage dient eine unvollständige technische Zeichnung der Profilkontur, aus der exakt entnommen werden kann, dass das Profil ohne die Hochauftriebselemente oder Querruder $ 172mm $ tief ist und eine Profildicke von $ 37,5mm $ aufweist. Aus den bekannten Längenangangaben kann der Maßstab der gedruckten Zeichnung zu $ 1:1,039 $ berechnet werden. Mithilfe eines Rechtecks, das die Kontur gerade umschließt, können weitere Punkte auf der Kontur des Profils ermittelt werden. Im CAD-Programm werden Tangentenbögen von Punkt zu Punkt gelegt, um die Kontur hinreichend glatt anzunähern.\\


\noindent In den Bereichen oberhalb und unterhalb des Holms soll die Haut nicht in Sandwich-Bauweise ausgeführt sein. Für die Auslegung des Holms wurde davon ausgegangen, dass eine Dicke des Verbundmaterials der Haut von $ 0,75mm $ ausreichend ist. Zunächst wird davon ausgegangen, dass für die Haut das Gewebe Interglas 90070 verwendet wird, das ein Flächengewicht von $ 80\frac{g}{m^{2}} $ aufweist. Nach Gleichung ~\ref{gurtlagen} entsprechen $ 9 $ Lagen dieses Gewebes der angenommenen Hautdicke. Dies erscheint ausreichend. Sollten weniger Lagen für die Haut benötigt werden, kann der entstehende Freiraum zwischen den Gurten und der Haut aufgefüllt werden. Um die Hautdicke von $ 0,75mm $ im Bereich der Gurte zu berücksichtigen, wird ein Offset von dieser Breite nach innen gerichtet.\\
\noindent Der zu Beginn des Abschnitts ~\ref{GurtDim} dimensionierte Holm mit rechteckigen Gurtquerschnitten, $ b=28mm $ und $ h_{a}=36mm $ wird nun so auf die Kontur des Profils gelegt, dass die Überdeckung der Gurte mit der umgebenden Haut möglichst gering ausfällt. Dann wird die Höhe $ h_{a} $ an den örtlichen inneren Abstand der oberen und unteren Haut auf $ \tilde{h_{a}}=35,8mm $ angepasst. Der rechteckige Querschnitt der Gurte wird mithilfe eines Offsets von $ \tilde{h}=1,941mm $ der Kontur der Haut angepasst. Diese Anpassungsmaßnahmen senken das Flächenträgheitsmoment leicht. Das resultierende Flächenträgheitsmoment $ \tilde{I_{x}} $ lässt sich aufgrund der komplexen Querschnittsgeometrie der Gurte mit dem CAD-Programm bestimmen. Der Vergleich mit dem erforderlichen Flächenträgheitsmoment zeigt, dass die angepasste Geometrie der Gurte die Steifigkeitsbedingung erfüllt.
