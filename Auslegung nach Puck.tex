\subsubsection{Beispielrechnung nach Klassischer Laminattheorie}
Nachdem im Kapitel . . .  eine Einführung in die klassische Laminattheorie gegeben wurde, wird im Folgenden eine Beispielrechnung, beginnen mit Materialdaten, bis zu den Ingenieurskonstanten eines Lagenaufbaus vorgestellt. Auf diese Berechnungsmethoden bezieht sich die Software $eLamX^{2}$, vorgestellt im folgendem Unterkapitel \ref{elamx}.

Für eine repräsentative Vorstellung soll ein Lagenaufbau 


\subsubsection{$eLamX^{2}$ (T.B.)}\label{elamx}
$eLamX^{2}$ ist ein Laminat-Berechnungsprogramm der TU Dresden, Es kann anhand der klassischen Laminattheorie mit unterschiedlichen Versagenskriterien berechnen, inwiefern ein gewählter Lagenaufbau den Festigkeitskriterien gerecht wird. Zusätzlich sind weitere Funktionen, wie zum Beispiel Beulberechnungen, Optimierungen etc. nutzbar und andere Einflüssse, z.B. Temperatureinflüsse anwendbar, jedoch für diese Auslegung irrelevant.

\noindent Vorerst wurden die gegeben Materialeigenschaften der Aufgabenstellung als Fasermaterial, Matrixmaterial und  Materialeigenschaften definiert.\\

\begin{tabular}{ll|ll|ll}
	\multicolumn{2}{c}{Fasermaterial} &\multicolumn{2}{c}{Matrixmaterial}  &\multicolumn{2}{c}{Materialeigenschaften} \\
	\hline
	$\rho_{f}$ & $2,55 \frac{g}{cm^{3}}$  & $\rho_{m}$ & $1,18 \frac{g}{cm^{3}}$  & $R_{\parallel}^{+}$ & $597,9MPa$ \\
	\hline
	$E_{f,11}$ & $74000MPa$  & $E_{M}$ & $3300MPa$  & $R_{\parallel}^{-}$ & $650,0MPa$\\
	\hline
	$E_{f,22}$ & $74000MPa$  &  &   & $R_{\perp}^{+}$ & $37,7MPa$\\
	\hline
	$G_{f,12}$ & $30800MPa$ & $G_{M}$ & $1222MPa$ & $R_{\perp}^{+-}$ & $130,0MPa$\\
	\hline
	$\nu_{f,21}$ & $0,2$  &$\nu_{M}$ &  $0,35$  & $R_{\parallel\perp}$ & $37,5MPa$\\
\end{tabular}\\

\noindent Mit einem Faservolumenanteil $\varphi=0,4$ ergeben sich folgende weitere Materialeigenschaften:\\

\begin{tabular}{ll}
	$\rho$ & $1,728 \frac{g}{cm^{3}}$ \\
	\hline
	$E_{\parallel}$ & $31580 MPa$\\
	\hline
	$E_{\perp}$ & $5341,2MPa$\\
	\hline
	$\nu_{\parallel\perp}$ & $0,29$\\
	\hline
	$G_{\parallel\perp}$ & $1984,5 MPa$\\
\end{tabular}\\

\noindent Anschließend werden die nach Kapitel ~\ref{VDI2013} berechneten Laminate bzw. Lagenzusammensetzungen aus mehreren Gewebe-Lagen zusammengesetzt. Eine Gewebelage wird dabei durch zwei einzelne Materiallagen mit einem Winkel von $90^{\circ}$ zueinander simuliert, sodass sich die doppelte Anzahl des Materials gegenüber der Gewebeanzahl ergibt. Für die Holmgurte ergibt sich ein Lagenaufbau nach Abbildung ~\ref{fig:Lagenaufbau Holmgurte} und für den dünnen Steg nach Abbildung ~\ref{fig:Lagenaufbau Steg dünn}. Für den dicken Steg ergibt sich der gleiche Aufbau wie bei dem dünnen Steg, allerdings mit 24 Materiallagen (siehe Abb. ~\ref{fig:Lagenaufbau Steg dick}). Weshalb sich die Anzahl der Gewebelagen gegenüber der Auslegung nach VDI 2013 unterscheidet, wird im Weiteren beschrieben.\\ 

\noindent Anschließend werden die nach VDI 2013 im Kapitel \ref{VDI} errechneten Normalkraft- und Schubflüsse bzw. Dehnungen eingegeben, sodass nun die Sicherheiten nach den Versagenskriterien von Puck ermittelt werden können. Dabei handelt es sich um leicht abweichende Beträge gegenüber der Realität. Diese Werte stammen aus der Auslegung nach VDI 2013, in der teilweise die Lagenanzahl, wie bereits beschrieben, geringer ist. Zu beachten ist außerdem, dass die Flüsse und Dehnungen auf das allgemeine Koordinatensystem und nicht auf die der einzelnen Lagen bezogen werden. Es wird automatisch die niedrigste Sicherheit mit dem jeweiligen Versagenskriterium ausgegeben. Auch muss berücksichtigt werden, dass für die komplette Überprüfung der Auslegung ebenfalls $\epsilon_{x}$ negativ angenommen werden muss, um den zweiten Gurt mit Druckbelastungen zu betrachten. Dabei ergeben sich jedoch stets höhere Sicherheiten.\\

\noindent Um immer eine Sicherheit von $j>1$ zu garantieren, müssen im Steg noch weitere Lagen zu der ursprünglich berechneten Anzahl hinzugefügt werden, sodass sich eine Gesamt-Lagenanzahl von 4 und 24 statt 2 und 20 ergibt. Diese Abweichung kann u.a. daran liegen, dass die VDI 2013 mit Konstanten rechnet, die nur beispielhaft an einem Gewebe ermittelt wurden und somit nicht exakt für die gegebenen Gewebe der Aufgabenstellung gelten können.
Dargestellt sind die Rechnungen in Abb. ~\ref{fig:Berechnung Holmgurte}, Abb.~\ref{fig:Berechnung Steg dünn} und Abb.~\ref{fig:Berechnung Steg dick}.\\

\noindent Als nächstes können die Ingenieurskonstanten (E-Moduln des gesamten Laminats in Hauptachsenrichtungen) ermittelt werden, welche für die spätere Stabilitätsabschätzung benötigt werden (siehe Abb. ~\ref{fig:Ingenieurskonstanten Holmgurte}, Abb. \ref{fig:Ingenieurskonstanten Steg dick} und Abb.~\ref{fig:Ingenieurskonstanten Steg dünn}).\\

\noindent Der Übersichtlichkeit halber werden die Abbildung im Kapitel \ref{Abbildungen} dargestellt.